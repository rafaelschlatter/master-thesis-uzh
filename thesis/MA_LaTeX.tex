\documentclass[10pt,twocolumn,oneside,cmyk]{article} 
\usepackage[left=1.5cm,right=1.5cm,top=2.5cm,bottom=1.5cm,
		     bindingoffset=0.9cm,columnsep=0.4cm, a4paper]{geometry}
\usepackage[
  % set width and height to a4 width and height + 6mm
  width=21.6truecm, height=30.3truecm,
  % use any combination of these options to add different cut markings
  %cross,
  % set the type of TeX renderer you use
  pdftex,
  % center the contents
  center
]{crop}

\usepackage{titling}
\usepackage[T1]{fontenc}
\usepackage[bitstream-charter]{mathdesign}
\usepackage{microtype}
\usepackage{sectsty}
\allsectionsfont{\usefont{T1}{qhv}{b}{n}\selectfont}
\usepackage{multicol}
\usepackage{amsmath}
\usepackage{bm}
\usepackage{lipsum}
\usepackage{multicol}
\usepackage{float}
\usepackage{supertabular}
\usepackage{tabularx}
\usepackage{booktabs}
\newcolumntype{Y}{>{\centering\arraybackslash}X}
\usepackage{rotating}
\usepackage{rotfloat}

\usepackage[svgnames,cmyk]{xcolor}
\xdefinecolor{My_Blue}{cmyk}{1,1,0,0}
\usepackage[misc]{ifsym}
\usepackage{mwe}
\usepackage{graphicx}

\usepackage{float}
\setlength{\dblfloatsep}{0.2cm plus 0.5cm minus 0.5cm}

\usepackage[backend=bibtex,style=authoryear,url=false,isbn=false]{biblatex}
\bibliography{MA_LaTeX}
\renewcommand*{\bibfont}{\footnotesize}

\usepackage{fancyhdr}
\fancypagestyle{mypagestyle}{%
    \fancyhf{}
    \fancyhead[OC]{\footnotesize{\textit{R. D. Schlatter / The impact of tax differentials on pre-tax income of Swiss MNEs}}}
    \fancyhead[EC]{\footnotesize{\textit{R. D. Schlatter / The impact of tax differentials on pre-tax income of Swiss MNEs}}}
    \fancyhead[LE,RO]{\footnotesize{\thepage}}
    \renewcommand{\headrulewidth}{0pt}
}

\pagestyle{mypagestyle}

\usepackage[colorlinks=true,allcolors=My_Blue,urlcolor=Maroon]{hyperref}
\usepackage[nameinlink,capitalize]{cleveref}
\crefname{section}{Sect.}{Sect.}
\crefname{subsection}{Subsect.}{Subsect.}
\Crefname{section}{Section}{Section}
\Crefname{subsection}{Subsection}{Subsection}
\Crefname{appendix}{Appx.}{Appx.}
\Crefname{table}{Tab.}{Tab.}

\renewcommand\figurename{Figure}
\usepackage{caption}
\usepackage{subcaption}
\captionsetup[figure]{font=footnotesize,labelfont=footnotesize,labelfont=bf,justification=justified,singlelinecheck=false,skip=0cm}

\renewcommand\tablename{Table}
\captionsetup[table]{font=footnotesize,labelfont=footnotesize,labelfont=bf,justification=justified,singlelinecheck=false,skip=0cm}

\usepackage{tikz}
\usepackage{verbatim}
\usetikzlibrary{arrows,positioning} 
\tikzset{
%Define standard arrow tip
    >=stealth',
%Define style for boxes
    punkt/.style={
           rectangle,
           rounded corners,
           draw=black, thin,
           text width=6em,
           minimum height=1em,
           text centered},
     punkt2/.style={
           rectangle,
           rounded corners,
%draw=black, thin,
           text width=18em,
           minimum height=2em,
           text centered},
    circle2/.style={
           circle,
           rounded corners,
           draw=black, thin,
           text width=3.5em,
           minimum height=1em,
           text centered},
% Define arrow style
    pil/.style={
           ->,
           thin,
           shorten <=1pt,
           shorten >=1pt,},
    pol/.style={
           <-,
           thin,
           shorten <=1pt,
           shorten >=1pt,}
}



\begin{document}

%---First blank page---
\thispagestyle{empty}
\onecolumn
\vspace*{\fill}
\begin{center}
This page is intentionally left blank.
\end{center}
\clearpage

%---Title Page---
\begin{titlingpage}
  \newgeometry{top=3.5cm,bottom=2.5cm,right=4cm,left=4.5cm}
   \onecolumn
    \begin{center}
     \includegraphics[height=24mm]{uzh_logo_e_pos.pdf} 
     \vspace{2cm}
     
     \large Department of Business Administration\\
     {\textit{Chair for Managerial Accounting}}\\
     \vspace{1cm}
      \begingroup
      %\fontfamily{qhv}{b}\selectfont
      %\vspace{1cm}
      %\rule{\textwidth}{1pt}
      \vspace{0cm}
      {\Huge The impact of tax differentials on pre-tax income of Swiss MNEs}
      %\rule{\textwidth}{1pt}
      \vspace{0.5cm}
      
      {\Large Master Thesis}\\
     \endgroup
     \vspace{2cm}
     
     Rafael Daniel Schlatter\footnote{\textit { \Letter} R. D. Schlatter, Weidstrasse 50, 4416 Bubendorf, Switzerland; 11-707-668, Business Administration; e-mail: \href{mailto:rafaelschlatter@gmail.com}{rafaelschlatter@gmail.com}.}
     \vspace{0.5cm}
     
     \begin{tabular}{c c}
      \textit{Supervisor}: Benedikt Bisig &\textit{Expert}: Prof. Dr. Dieter Pfaff\\
     \end{tabular}
     \vspace{0.5cm}
     
     Oslo, Jul 17, 2017
     \vspace{0.5cm}
    \end{center}
    \hrule
    \begin{abstract}
Multinational enterprises may use income shifting techniques such as strategic transfer pricing and debt shifting to reduce their global tax burden. Due to a comparably low corporate taxation, Switzerland is a seemingly suitable location for tax planning strategies. The thesis at hand examines income shifting among multinational enterprises headquartered in Switzerland in a quantitative manner and provides indirect evidence of income shifting. Using a large panel dataset of foreign subsidiaries of Swiss parent firms and employing a fixed-effects regression approach, the estimated semi-elasticity of pre-tax income with respect to the statutory tax rate differential between the parent firm and the subsidiary in the benchmark specification is $-1.458$. This estimate is highly significant and larger than the estimates in comparable papers using European samples. Additionally, this thesis shows that income shifting activities between the parent firm and the subsidiary increase with the parent's ownership share in the subsidiary and the firm size of the subsidiary. Hence, Swiss multinational enterprises preferably shift income using large, wholly-owned subsidiaries.
  \end{abstract}
 \restoregeometry
\end{titlingpage}

%---Dedication---
\onecolumn
  \begin{center}
   For my parents, Trix and Rolf.
  \end{center}
\clearpage

%---Indices---
\tableofcontents
\newpage

\listoffigures
\listoftables
\clearpage

%---List of Symbols---
\section*{List of symbols}
Symbols are listed in the basic form in order of appearance. Some symbols are altered by adding a subscript (time-, subsidiary-, parent-, MNE- or multiple indices) or a superscript.\\

\begin{center}
 \begin{tabular}{l l}
  \toprule
  Symbol &Property \\
  \midrule
  $h$ &parent firm index\\
  $i$ &subsidiary index \\
  $t$ &time index \\
  $r$ &statutory corporate income tax rate \\
  $\Pi$ &total pre-tax income \\
  $\Pi^T$ &true pre-tax income (in the absence of income shifting) \\
  $s$ &shifted income \\
  $c$ &cost function of income shifting \\
  $\gamma$ &factor of proportionality in cost function \\
  $\Pi_{\text{MNE}}$ &total after-tax income of the MNE \\
  $Q$ &production output according to a Cobb-Douglas production function \\
  $w$ &wage costs \\
  $k$ &constant in production function \\
  $A$,$L$,$K$ & technology, labor and capital input\\
  $u$ &random term in production function \\
  $e$ &Euler's number \\
  $\alpha$, $\lambda$, $\phi$ &technology, labor and capital coefficient in production function \\
  $a$, $b$, $x$ &positive numbers \\
  $\beta_1-\beta_4$ &regression coefficients of the basic model \\
  $\tau$ &tax differential between the subsidiary and the parent firm \\
  $I$ &intangible fixed assets \\
  $\text{OW\_51}_{it}$ &$1^{st}$ categorical ownership variable \\
  $\text{OW\_100}_{it}$ &$2^{nd}$ categorical ownership variable \\
  $Case2_{it}$ &categorical shifting direction variable \\
  $\beta_5-\beta_{11}$ &regression coefficients of the extended model \\
  $\bm{T}$ &a row vector, holding the year dummies \\
  $\bm{U}$ &a row vector, holding the industry-year dummies \\
  $\bm{\theta}$ &a column vector, holding coefficients for the year dummies \\
  $\bm{\xi}$ &a column vector, holding coefficients for the industry-year dummies \\
  $\rho$ &subsidiary-fixed effect \\
  $R^2$ &coefficient of determination (within R squared) \\
  $F$ &$F$-distribution \\
  $E$ &expectation \\
  $H_0$, $H_A$ &null and alternative hypothesis \\
  $\chi^2$ &chi-square distribution \\
  \bottomrule
 \end{tabular}
\end{center}
\newpage

%---List of abbreviations---
\section*{List of abbreviations}
\begin{center}
 \begin{tabularx}{\textwidth}{l X}
  BvD &Bureau van Dijk\\
  cat. &Categorical \\
  CFC &Controlled foreign corporation \\
  CHF &Swiss francs \\
  CI &Confidence interval \\
  CITR &Corporate income tax rate \\
  cont. &Continuous \\
  cov &Covariance \\
  DTA &Double taxation agreement \\
  EBIT &Earnings before interest and taxes \\
  ETR &Effective tax rate \\
  FDI &Foreign direct investment \\
  FE &Fixed effects \\
  GDP &Gross domestic product \\
  ID &Identification number \\
  i.i.d. &Independently and identically distributed \\
  ISO &International standard organization \\
  MiDi &Microdatabase foreign investment \\
  MNE &Multinational enterprise \\
  NACE &European classification of economic activities (NACE is the acronym for the French expression "Nomenclature statistique des activites economiques dans la Communaute europeenne" European \textcite[5]{european_commission_nace_2008})\\
  No. &Number \\
  N/A &Not applicable \\
  Obs. &Observation \\
  OECD &Organization for economic co-operation and development \\
  OLS &Ordinary least squares \\
  Perc. &Percentile \\
  PPE &Property, plant and equipment \\
  P/L &Profit/loss \\
  Q-Q &Quantile-quantile \\
  RE &Random effects \\
  ROA &Return on assets \\
  ROS &Return on sales \\
  SE &Standard error \\
  St. dev. &Standard deviation \\
  TP &Transfer pricing \\
  UK &United Kingdom \\
  US &United States of America\\
  UN &United nations \\
  Var &Variance \\
  w.r.t. &With respect to \\
 \end{tabularx}
\end{center}
\clearpage
\twocolumn

%---Section 1---
\section{Introduction} \label{sec:1}
\subsection{Multinational enterprises and income shifting} \label{sec:Multinational enterprises and income shifting}
Firms operating in a multi-jurisdictional environment can use two main income shifting techniques to reduce their global tax burden \parencite[2227]{bartelsman_why_2003}. First, a multinational enterprise (MNE) can make strategic use of transfer pricing (TP). To do so, a MNE may supply intermediate goods or services from an affiliate incorporated in a low-tax (high-tax) jurisdiction to an affiliate incorporated within a high-tax (low-tax) jurisdiction priced above (below) the true price. Secondly, a MNE may grant a loan from an affiliate residing in a low-tax country to an affiliate residing in a high-tax country and benefit from the tax-deductibility of interest payments in the high-tax country. In both cases, income is shifted from a high-tax affiliate to a low-tax affiliate and the global tax bill of the MNE is smaller than it is in the absence of income shifting. The effectivity of these tax planning strategies is partially dependent on the peculiarities of the legal setting. Specific national tax law, as well as international double taxation agreements (DTA) may prevent or deter income shifting of MNEs. In case tax saving potentials are available, it is widely agreed on, that MNEs respond mainly to tax differentials in statutory tax rates \parencites(for example)()[307]{haufler_corporate_2000}[][1499]{becker_corporate_2012} That is, the MNE strategically relocates income to affiliates residing in low-tax countries by means of income shifting techniques.

Prior literature finds empirical, although mostly indirect, evidence that MNEs shift income.\footnote{See \textcite{dharmapala_what_2014} or \textcite{heckemeyer_multinationals_2013} for overviews on evidence of income shifting.} Pre-tax income that is sensitive to changes in tax differentials between affiliates of a MNE is interpreted as indirect evidence for income shifting \parencite[684]{mooij_corporate_2008}. Examples of direct evidence of income shifting are provided by \textcite[23-24]{vicard_profit_2015}, \textcite[1-2]{overesch_transfer_2006} and \textcite[2222]{clausing_tax-motivated_2003}. Recent research is often dedicated to finding specific opportunities to shift income or identifying tax regulations that deter income shifting. The former opportunities, also called drivers of income shifting, are usually based on basic observable firm and country characteristics. These drivers increase the sensitivity of pre-tax income to changes in tax differentials, while tax regulations decrease this sensitivity. Within Europe, especially German authors have taken up on the subject addressing German or European MNEs and the effectiveness of tax legislations. The interest in the German case could stem from the fact that Germany is characterized by high statutory corporate income tax rates (CITR) and income potentially being shifted out of the country \parencite[284, 293]{weichenrieder_profit_2009}. \textcite[1179]{huizinga_international_2008} estimate that 13.6\% of Germany's tax base in 1999 is lost due to income shifting. For the opposite reason, Switzerland is presumably an interesting location to study income shifting.

\subsection{Objective} \label{sec:Objective}
The objective of this thesis is to investigate whether Swiss MNEs shift income or not. Therefore, the sensitivity of pre-tax income of subsidiaries of Swiss MNEs to tax differentials is empirically examined. This sensitivity is to be quantified in terms of a semi-elasticity of the pre-tax income with respect to (w.r.t.) the tax differential. It is further to be verified if firm specific drivers of income shifting affect this semi-elasticity. The results are put in a European context and various implications are discussed. Few studies about income shifting focus on single countries in Europe. Examples are \textcite{langli_taxable_2004} in the case of Norway, \textcite{weichenrieder_profit_2009} in the case of Germany, \textcite{mura_challenging_2013} in the Case of Italy and \textcite{vicard_profit_2015} in the case of France. Up to the current knowledge of the author of this thesis, there exists no comparable research specifically addressing income shifting of Swiss MNEs.\footnote{Swiss firms have been included in previous studies on income shifting within Europe. However, the number of Swiss firms being low, these studies do not allow to draw specific conclusions for Swiss firms.} The thesis at hand attempts to fill this gap in the empirical literature on income shifting.

\subsection{Research approach} \label{Research approach}
\Cref{sec:Literature review} summarizes prior literature. This literature is used in two ways. First, drivers of income shifting based on certain firm and country characteristics suitable for the empirical analysis are identified. Secondly, tax legislation effectively deterring income shifting is analyzed. Tax legislation is not empirically studied, but used to qualitatively infer how effective Swiss tax legislation prevents income shifting. \Cref{sec:Switzerland as a location of income shifting} provides an overview on corporate taxation in Switzerland. \Cref{sec:Theoretical considerations} presents a basic model to study income shifting of Swiss MNEs. This model is theoretically established by \textcite{hines_fiscal_1994} and notably expanded by \textcite{huizinga_international_2008}. A brief description of this model can be found in \textcite[424-427]{dharmapala_what_2014}. The basic model is then extended by incorporating the drivers of income shifting from the literature review. The extended model allows to study how the sensitivity of pre-tax income to changes in tax differentials depends on these drivers of income shifting. The earnings before interest and taxes (EBIT) is employed as a measure of pre-tax income. \Cref{sec:Data and sample} describes the data and variables used. The empirical approach relies mainly on financial firm data retrieved from the ORBIS database. The data is collected at the subsidiary level. Tax and macroeconomic data is obtained from sources otherwise stated. Furthermore, \cref{sec:Data and sample} explains the sample construction process. \Cref{sec:Empirical results} presents and summarizes the empirical results of the basic and extended model. Additionally, robustness tests are used to verify the results. \Cref{sec:Discussion} contains a discussion of the results and provides implications for future research. Specific issues examined are how the income shifting behavior of Swiss MNEs developed over time, and which locations play important roles in income shifting strategies of Swiss MNEs. \Cref{sec:Limitations} addresses the limitations of the thesis and \cref{sec:Conclusion} concludes. Stata 14 \parencite{statacorp._stata_2015} is used for statistical computation, and \LaTeX \parencite{the_tex_users_group_mactex-2017_2017} and R 3.3.3 in conjuncture with ggplot2 \parencites{rcoreteam_2017}{wickham_ggplot2_2016} are used to plot figures.

%---2---
\section{Literature review} \label{sec:Literature review}
The subject of income shifting started to become popular in the United States (US) in the 1990's. An early pioneering work by \textcite{hines_fiscal_1994} influences the literature up to the present. The authors argue that the income of a firm is partly attributable to production factors such as capital and labor, and, if income is shifted, partly to tax rates. They regress the pre-tax income on capital, labor, gross domestic product (GDP) per capita and the effective tax rate (ETR) \parencite[161]{hines_fiscal_1994}. This approach became popular later on and \textcite[423]{dharmapala_what_2014} refers to it as the "Hines and Rice approach". Hines and Rice apply a log-level specification, allowing them to interpret the coefficient estimate of the tax variable as a semielasticity. They use country-aggregated data on US majority-owned affiliates and find that a 1 percentage point increase in the ETR reduces reported income by roughly 3\% \parencite[161-162]{hines_fiscal_1994}. \textcite{huizinga_international_2008} draw upon their approach and use affiliate-level data of European firms from the AMADEUS database and a composite tax variable. The authors argue that income shifting can not only take place between the parent and the subsidiary, but also between subsidiaries of the same parent. They calculate an average tax differential considering all affiliates of a MNE and further argue that for income shifting to take place, the MNE needs an incentive and an opportunity to shift income. The incentive is given by the average tax differential, the opportunity is given by the scale of operations of the affiliate (which they proxy with sales) and the composite tax variable is a combination of the two. Hence, income shifting is large if the average tax differential and the scale of operations are large \parencite[1168-1169]{huizinga_international_2008}. Huizinga and Laeven regress the EBIT on production inputs and the composite tax variable in a log-level specification. Their preferred estimate is -1.766, meaning that an increase in the composite tax variable by 1 percentage point is associated with a decrease in the EBIT of 1.766\%, which is in line with income shifting \parencite[1177]{huizinga_international_2008}.

Numerous authors use the models developed in these two papers to introduce drivers of income shifting, and to explain differences in income shifting behavior. Such studies typically argue, that the opportunities to shift income depend on some observable firm and country characteristics. A standard empirical approach to study drivers of income shifting is to create an interaction term consisting of the driver and the tax incentive variable, and integrating it within the "Hines and Rice approach". The drivers of income shifting considered in this thesis are: the scale of operations, the amount of intangible assets, the ownership share of the parent in the subsidiary and the direction of income shifting.\footnote{Other recently studied drivers are the firm complexity \parencite{beer_profit_2015} or the legal form of the firm \parencite{beuselinck_cross-jurisdictional_2015}.} The legal environment further affects income shifting by reducing possibilities for tax savings in certain countries.

The role of intangible assets within the "Hines and Rice approach" has been studied by \textcite{grubert_intangible_2003}, \textcite{beer_profit_2015} and \textcite{dischinger_corporate_2011}. Because it is difficult to determine arm's length prices for intangibles, the transfer of such provides opportuni- ties for tax savings through income shifting \parencites[2235]{bartelsman_why_2003}[428]{beer_profit_2015}. \textcite[229]{grubert_intangible_2003} examines the effect of intangible assets owned by the parent firm on income shifting and finds that a high intangible asset endowment of the parent firm facilitates income shifting. In contrast hereto, \textcite{beer_profit_2015} use the intangible asset endowment of subsidiaries to explain differences in income shifting behavior. They find that income shifting increases with the subsidiary's amount of intangible assets \parencite[434]{beer_profit_2015}. \textcite[699-700]{dischinger_corporate_2011} find the same qualitative result within a similar framework. Alternative research approaches have further confirmed the use of intangibles in income shifting strategies. \textcite[13]{dischinger_corporate_2008} show that intangible assets within MNEs are located at affiliates facing a low statutory tax rate and \textcite[182]{karkinsky_corporate_2012} find that the tax rate is negatively correlated with the number of patent applications. Both findings suggest intangibles are being used to shift income to low-tax jurisdictions.

\textcite{weichenrieder_profit_2009} studies the impact of the ownership share on the income shifting behavior of German foreign direct investment (FDI). He argues that income shifting to the parent firm is additionally costly due to opposition from other shareholders with conflicting interests if the subsidiary is not wholly-owned. These additional costs do not accrue if income is shifted from the parent firm to the subsidiary \parencite[285]{weichenrieder_profit_2009}. Using the microdatabase direct investment (MiDi), Weichenrieder finds that wholly-owned subsidiaries react stronger to tax rate changes, however, this finding turns insignificant once the capital structure of the subsidiary is accounted for \parencite[295]{weichenrieder_profit_2009}. \textcite[341-343]{desai_costs_2004} find that the profitability of non wholly-owned subsidiaries is less sensitive to tax rates than is that of wholly-owned subsidiaries. This is in line with income shifting and they conclude that a MNE's possibility to minimize taxes is dampened by shared ownership \parencite[341]{desai_costs_2004}. \textcite{dischinger_profit_2008} uses the "Hines and Rice approach" and a continuous variable to measure the ownership share and shows that subsidiaries owned with a higher ownership share shift significantly more income \parencite[17-18]{dischinger_profit_2008}. Further, \textcite[84]{buettner_internal_2013} find that majority-owned subsidiaries use internal debt to shift income to low-tax countries more pronounced than non majority-owned subsidiaries.

\textcite{dischinger_role_2014} focus on the direction of income shifting. The authors distinguish income shifting towards the parent firm from income shifting towards the subsidiary. Due to a headquarter bias, it is more costly to shift income to the subsidiary. Various arguments suggest the existence of a headquarter bias. The authors mention that managers prefer having assets under direct control rather than having them in distant locations, and that funds might be located at the parent firm to avoid potential withholding taxes upon dividend repatriation \parencite[249]{dischinger_role_2014}. Their results imply that the amount of shifted income is by more than 70\% smaller if the income is shifted away from the parent instead of shifted towards the parent \parencite[268]{dischinger_role_2014}.

Various authors have studied the impact of tax and TP legislation on the income shifting behavior of MNEs. Among them are \textcite{lohse_impact_2012}, \textcite{buettner_anti_2017}, \textcite{beuselinck_cross-jurisdictional_2015} and \textcite{ruf_taxation_2012}. \textcite{lohse_impact_2012} assess the impact of TP regulations on the income shifting behavior of European MNEs. They split countries into three categories according to the strictness of TP regulations. Category 1 includes countries with no or very general regulations, but no documentation requirements. Category 2 includes countries with TP regulations including documentation requirements. Category 3 is similar to category 2, but the TP documentation regulations are required to be incorporated into national tax law \parencite[6-7, 19-20]{lohse_impact_2012}. The authors find that income shifting among firms incorporated in countries with binding TP regulations (countries assigned to category 2 and 3) is reduced by approximately 50\% \parencite[10-11]{lohse_impact_2012}. \textcite{buettner_anti_2017} use a similar approach based on a strictness of TP legislation overview provided by \textcite[21-24]{lohse_increasing_2012}, but do not find a significant reduction in income shifting due to stricter TP regulations. However, they find that thin-capitalization rules reduce income shifting \parencite[13-14]{buettner_anti_2017}. \textcite{beuselinck_cross-jurisdictional_2015} use a broader system to quantify the strength of tax enforcement in different countries. Factors they include are the tax audit risk, related party disclosure requirements, existence of tax favorable holding regimes, existence of thin-capitalization rules and DTAs and the possibility to carry-forward losses \parencite[715-716]{beuselinck_cross-jurisdictional_2015}. They find that a weak tax enforcement environment is associated with more income shifting \parencite[729-732]{beuselinck_cross-jurisdictional_2015}. \textcite{ruf_taxation_2012} study the effect of controlled foreign corporation (CFC) rules on passive investment of German MNEs. They find that both, a higher tax rate and binding CFC rules are associated with lower passive investment. Specifically, the passive assets of a foreign subsidiary are reduced by 77\% if that country is affected by the German CFC rules \parencite[1513-1514]{ruf_taxation_2012}. The authors conclude that the German CFC rules effectively reduce tax revenue loss by preventing the outflow of passive investments to low-tax countries \parencite[1527]{ruf_taxation_2012}.

The four drivers of income shifting are incorporated into the basic model by \textcite{huizinga_international_2008} in \cref{sec:Extended model}. The tax legislation with a potentially dampening effect on income shifting is not studied empirically since the sample includes only subsidiaries of Swiss parent firms, which are subject to the same tax system. Nonetheless, the results of these studies give hints about how and to which extent Swiss corporate tax legislation may prevent income shifting among Swiss MNEs. \Cref{app:A} provides a structured classification of the literature reviewed here. A more comprehensive overview on empirical literature can be found for example in \textcite{devereux_impact_2007}.

%---3---
\section{Switzerland as a location of income shifting} \label{sec:Switzerland as a location of income shifting}
The purpose of this section is to evaluate whether Swiss MNEs can possibly reduce their global tax burden using income shifting techniques. To do so, relevant aspects of the corporate tax environment in Switzerland are analyzed. Corporate income taxes in Switzerland are levied on three organizational levels. On the federal level a flat rate of 8.5\% applies (the same rate applies regardless of the amount of income), but differing rates apply on the cantonal level, and rates expressed as percentages of the cantonal rate are levied on the municipal level \parencite[7]{galletta_corporate_2017}. As a consequence, tax rates are heterogeneous in Switzerland. \textcite[18]{lampart_unternehmen_2012} present ETRs for stock companies in Switzerland between 13 and 23.7\% in 2010, depending on the exact location of incorporation. In a European context, statutory CITRs in Switzerland may be considered to be of low to moderate level.\footnote{See \cref{sec:Tax rates and tax differentials} for a comparison of European and worldwide CITRs.} Considering \textcite[307]{haufler_corporate_2000}, who state that income shifting reacts mainly to statutory tax differentials, the low CITRs in Switzerland constitute an incentive to shift income to Switzerland. Two arguments favor the use of statutory tax rates over ETRs. First, previous literature used mainly statutory rates and statutory rates are easy to use, whereas it can be difficult to calculate ETRs due to differing tax favors and deductions in different locations. Secondly, statutory tax rates are determined by governments only and are therefore not under the influence of the firm \parencite[425]{dharmapala_what_2014}. A drawback of statutory tax rates is that they reflect the shifting incentive less accurate, especially in the presence of losses and loss carry-forwards \parencites[16]{overesch_transfer_2006}[74]{buettner_internal_2013}. To comply with previous literature, statutory tax rates are used in this thesis.\footnote{Loss-making subsidiaries are excluded from the sample to mitigate this drawback (see \cref{tab2}.}

Low statutory CITRs are not sufficient to explain income shifting. In case the home country
of the parent firm taxes income of its foreign subsidiaries, no tax saving is obtainable by shifting income among affiliates. Double taxation can occur if the country of residence of the subsidiary also levies taxes on the subsidiaries' domestic income. To counteract double taxation,
Switzerland has concluded DTAs with major industrial countries. The Swiss Confederation
provides and updates a list of all DTAs online, as of March 2017, Switzerland has signed 55
DTAs in accordance with the organization for economic co-operation and development
(OECD) standard, of which 50 are in force \parencite{swiss_confederation_double_2017}. The DTAs apply the exemption method, which excludes income of foreign subsidiaries from taxation in Switzerland \parencites[121]{swiss_global_enterprise_handbook_2016}[36-37]{oecd_model_2014}. Therefore, Swiss MNEs can possibly realize tax savings by shifting income to Switzerland. This reasoning is confirmed by \textcite[16]{heckemeyer_multinationals_2013}. Further, \textcite[8-9, 32-33]{markle_comparison_2016} presents empirical evidence that MNEs subject to a territorial taxation system (countries that generally exempt foreign income from home taxation) shift significantly more income than MNEs subject to a worldwide taxation system. However, a taxation method with an exemption system by itself is not tantamount with income shifting being present. As mentioned in the introduction, certain tax legislations can prevent or dampen income shifting through specific channels such as debt shifting and manipulation of transfer prices. It is thin-capitalization rules in the case of debt shifting \parencite[13-14]{buettner_anti_2017} and TP legislation and documentation requirements in the case of income shifting through strategic TP \parencite[15]{lohse_impact_2012}. The federal tax administration in Switzerland issues minimum and maximum interest rates that are tax-deductible. If the charged rate is outside this interval, it might be adjusted \parencite[963]{pwc_international_2015}. The safe haven debt-to-equity ratio\footnote{Interest payments are granted to be deductible if the debt-to-equity ratio is below the safe haven debt-to-equity ratio \parencite[931]{buettner_impact_2012}.} is 6:1. This ratio is generous in comparison with other countries, meaning that the thin-capitalization rules in Switzerland are rather loose in an international context \parencite[932]{buettner_impact_2012}.\footnote{The comparison dates back to 2005, however, as the authors point out, it occurs seldom that a country abolishes thin-capitalization rules \parencite[932]{buettner_impact_2012}. \textcite[963]{pwc_international_2015} mention that the 6:1 ratio is still applicable in Switzerland in 2015.} Switzerland currently does not apply specific TP legislation or documentation requirements, but follows the OECD guidelines on TP \parencite[962, 964]{pwc_international_2015}. Unsurprisingly, Switzerland ranked low in a worldwide overview on the strictness of national TP regulations \parencite[23-24]{lohse_increasing_2012}.

Thus, the low statutory CITRs within an exemption method taxation system and the absence of strict thin-capitalization rules and TP legislation allow Swiss MNEs to realize tax savings by means of income shifting. More specifically, tax savings can be realized by shifting income to Switzerland, whenever the foreign tax rate is higher than the Swiss rate. Further, according to \textcite[243-244]{gehriger_konzernfinanzierungsgesellschaften_2008} and \textcite[1527-1528]{ruf_taxation_2012}, the absence of CFC rules facilitates income shifting. A broader guide on corporate taxation in Switzerland can be found for example in \textcite[131-135]{feld_impact_2002}.

%---4---
\section{Theoretical considerations} \label{sec:Theoretical considerations}
\subsection{Basic model} \label{sec:Basic model}
The basic model allows to study the relation between tax differentials and pre-tax income of Swiss MNEs. The basic model presented here constitutes an adaptation of the model by \textcite{huizinga_international_2008}. The model is extended in \cref{sec:Extended model} to incorporate the drivers of income shifting. The representative MNE consists of two affiliates, the parent firm h in Switzerland and the subsidiary i in a foreign country. Due to limited access to ownership data, it is not possible to create a comprehensive MNE dataset with groups consisting of more than two affiliates, as for example in \textcite[1169]{huizinga_international_2008} or \textcite[430]{beer_profit_2015}. It is only possible to link subsidiaries to Swiss shareholders. For the purpose of this thesis, a firm is considered a subsidiary if the Swiss parent firm owns at least 10\% of the shares in the subsidiary.\footnote{This might not always constitute a controlling interest. However, firms with an ownership share of at least 10\% are included in the sample to study the effect of the ownership share on income shifting. \textcite[16]{dischinger_profit_2008} applies a minimum ownership of 25\% to study the impact on income shifting. \Cref{sec:5.1} elaborates on the issue of limited ownership information and the 10\% ownership threshold.} It is assumed that both countries exempt foreign income from taxation, but tax any income that has been shifted to the country. This assumption is necessary for income shifting to have a tax-saving effect and is standard in the literature. See for example \textcite[312]{haufler_corporate_2000} or \textcite[1151]{mintz_income_2004}.

%---Figure 1---
\begin{figure}
 \centering \captionsetup{width=0.95\linewidth}
   \begin{tikzpicture}[node distance=0.7cm, auto]
    \linespread{0.7}
    (formidler) {Intermediaries (c) balsdf sdf as asdf asdf asfd asdf };
    \node [circle2] (dummy) {\footnotesize ownership share = $OW_{it}$};
    \node[right=of dummy] [punkt] (F) {\footnotesize Foreign subsidiary $i$, outside of Switzerland, taxed at CITR = $r_{it}$};
    \node[left=of dummy] [punkt] (g) {\footnotesize Parent firm $h$ in Switzerland, taxed at CITR = $r_{ht}$}
     edge[pol, bend right=45] node[below] [punkt2]{\footnotesize \textbf{Case 2:} $r_{it}>r_{ht}$, $s_{it}<0$ income shifting to parent $h$ at costs $c_{it}$} (F)
   edge[pil, bend left=45] node[above] [punkt2]{\footnotesize \textbf{Case 1:} $r_{it}<r_{ht}$, $s_{it}>0$ income shifting to subsidiary $i$ at costs $c_{it}$} (F);
    \draw[dotted] [->, thick] (dummy) -- (F);
    \draw[dotted] [-, thick] (g) -- (dummy);
   \end{tikzpicture}
  \caption[Income shifting scenarios among Swiss multinational enterprises]{Income shifting scenarios among Swiss multinational enterprises. Shifted income is depicted by solid arrows and ownership stakes are depicted by doted arrows. Source: own figure.} \label{fig1}
\end{figure}

The statutory CITRs of the parent and the subsidiary are $r_{ht}$, respectively $r_{it}$. Following \textcite[159]{hines_fiscal_1994}, the total income of subsidiary $i$ in year $t$, $\Pi_{it}$, equals the sum of true but unobserved income (income in the absence of income shifting) and shifted income. True income is denoted by $\Pi_{it}^T$ and shifted income is denoted by $s_{it}$. If $r_{it}<r_{ht}$ $(r_{it}>r_{ht})$, then $s_{it}>0$ $(s_{it}<0)$ and income is shifted to the subsidiary (to the parent). These are cases 1 and 2 in \cref{fig1}. Income shifting is costly and prior literature states a modification of the firms books and real trade and investment flows to justify the income shifting \parencite[1166]{huizinga_international_2008}, or efforts undertaken to conceal the shifting \parencite[313]{haufler_corporate_2000} as explanation for the costs associated with income shifting. The costs of income shifting for subsidiary $i$ in year $t$ are denoted $c_{it}$. \cref{fig1} shows a graphical representation of this setup.

According to \textcite[159]{hines_fiscal_1994} and \textcite[1168-1169]{huizinga_international_2008}, the total pre-tax income of subsidiary $i$ in year $t$ can be written as\footnote{As a result of the aforementioned data restriction, only subsidiaries but not parents can be studied. The sole use of the subsidiary index $i$ reflects this. In fact, the only known information about the parent firm is that it is located in Switzerland.}

%---Equation 1---
\begin{equation}\label{eqone}
 \Pi_{it}=\Pi_{it}^T+s_{it}.
\end{equation}

In the following, the components of the total income are described, starting with the shifted income. To derive the optimal amount of income shifting $s_{it}$, an income function for the MNE is set up. The income of the MNE consists of the true income of both affiliates, plus the tax effects from income shifting, minus the costs of income shifting \parencites[285]{weichenrieder_profit_2009}[250]{dischinger_role_2014}. The cost function specification is taken from \textcite[159]{hines_fiscal_1994}. The cost function reflects that the costs of income shifting are proportional with factor $\gamma$ to the ratio of shifted income over true income. Hence, the costs of income shifting are lower if the true income of the subsidiary is high. This is also emphasized by \textcite[319]{haufler_corporate_2000}, who argue that income shifting to a given country is less costly when the true income or the level of investment in that country is high. The cost function specification is given by

%---Equation 2---
\begin{equation}\label{eqtwo}
 c_{it}(s_{it})=\frac{\gamma}{2} \cdot \frac{s_{it}^2}{\Pi_{it}^T},
\end{equation}

which is a convex function applying equally to positive and negative values of $s_{it}$. The costs of
shifting income are assumed to be non-tax deductible \parencite[250]{dischinger_role_2014}. For simplicity, it is further assumed that these costs are solely borne by the parent firm in Switzerland. Using the cost function above, the total after-tax income of the representative MNE in year $t$ can be expressed as

%---Equation 3---
\begin{equation}\label{eqthree}
 \begin{split}
  \Pi_{MNE,t} &=\underbrace{(1-r_{ht})(\Pi_{ht}^T-s_{it})-c_{it}({s_{it}})}_{\text{after-tax parent income}}\\
	&+\underbrace{(1-r_{it})(\Pi_{it}^T+s_{it})}_{\text{after-tax subsidiary income}},
 \end{split}
\end{equation}

where $\Pi_{ht}^T$ is the true income of the parent $h$. Note that income shifting by itself does not create additional income, since income shifted to one affiliate is equal to the income shifted away from the other affiliate \parencite[159]{hines_fiscal_1994}. Differentiating \cref{eqthree} w.r.t. $s_{it}$, substituting the cost function and solving for sit yields the optimal amount of income shifting\footnote{It might seem that not the straight forward solution has been chosen when solving for the optimal amount of $s_{it}$. This is done to ensure comparability to recent literature, which usually defines the tax differential as subsidiary tax rate minus the parent tax rate \parencite[for example][259]{dischinger_role_2014}.}

%---Equation 4---
\begin{equation}\label{eqfour}
 \begin{split}
  \partial \Pi_{\text{MNE,$t$}}/\partial s_{it} &=(r_{it}-r_{ht})-c'_{it}=0\\
	& \Leftrightarrow (r_{it}-r_{ht})=-\gamma \cdot \frac{s_{it}}{\Pi_{it}^T}\\
	& \Leftrightarrow s_{it}=-\frac{1}{\gamma} \cdot \Pi_{it}^T(r_{it}-r_{ht}).
 \end{split}
\end{equation}

Substituting the third row from \cref{eqfour} into \cref{eqone} and rearranging results in

%---Equation 5---
\begin{equation}\label{eqfive}
 \begin{split}
  \Pi_{it} &=\Pi_{it}^T-\frac{1}{\gamma}\cdot\Pi_{it}^T(r_{it}-r_{ht})\\
	& = \Pi_{it}^T \cdot \left[1-\frac{1}{\gamma} \cdot (r_{it}-r_{ht}) \right],
 \end{split}
\end{equation}

taking natural logarithms and approximating the second line of \cref{eqfive} yields\footnote{The approximation used is $\ln (1+x) \approx x$, if $x$ is close to 0 and is taken from \textcite[1169]{huizinga_international_2008}. Note that $\ln(a\cdot b)=\ln(a)+\ln(b)$ has been applied before using the approximation. Here $x$ is equal to $1/\gamma \cdot (r_{it}-r_{ht})$. With moderate tax differentials, this expression is reasonably close to 0.}

%---Equation 6---
\begin{equation}\label{eqsix}
 \ln \Pi_{it} = \ln \Pi_{it}^T-\frac{1}{\gamma}\cdot(r_{it}-r_{ht}).
\end{equation}

In line with \textcite[160-161]{hines_fiscal_1994} and \textcite[1169]{huizinga_international_2008}, the true income is assumed to be the output $Q_{it}$ produced according to a Cobb-Douglas production function, minus the wage costs $w_{it}$. The above authors propose the following specification,
$Q=kA_{it}^\alpha L_{it}^\lambda K_{it}^\phi e^{u_{it}}$, where $k$ is a constant term, $A_{it}$, $L_{it}$ and $K_{it}$ are the technology, labor and capital input of subsidiary $i$ in year $t$, $u_{it}$ is a random term and $e$ is Euler's number. The wage costs $w_{it}$ are equal to the partial derivative of $Q_{it}$ w.r.t. $L_{it}$, which gives $w_{it}=k\lambda A_{it}^\alpha L_{it}^{\lambda-1} K_{it}^\phi e^{u_{it}}$. Subtracting the wage costs $w_{it}$ from the true income $\Pi_{it}^T$ is therefore equal to

%---Equation 7---
\begin{equation}\label{eqseven}
 \Pi_{it}^T=Q_{it}-w_{it}=k(1-\lambda) A_{it}^\alpha L_{it}^\lambda K_{it}^\phi e^{u_{it}},
\end{equation}

and taking natural logarithms gives

%---Equation 8---
\begin{equation}\label{eqeight}
 \begin{split}
  \ln \Pi_{it}^T &= \ln k+\ln (1-\lambda)+\alpha \cdot \ln A_{it} +\lambda \cdot \ln L_{it}\\
	&+ \phi \cdot \ln K_{it} + u_{it}.
 \end{split}
\end{equation}

Replacing $\ln \Pi_{it}^T$ in \cref{eqsix} with the expression given in \cref{eqeight} and introducing empirically customary notation, yields the following basic model

%---Equation 9---
\begin{equation}\label{eqnine}
 \begin{split}
  \ln \Pi_{it} &= \underbrace{\ln k + \ln(1-\lambda)}_{\text{$\equiv \beta_0$}}+\alpha \cdot \ln A_{it} +\lambda \cdot \ln L_{it}\\
	&+ \phi \cdot \ln K_{it} + u_{it}-\frac{1}{\gamma}\cdot \underbrace{(r_{it}-r_{ht})}_{\text{$\equiv \tau_{it}$}}\\
	&= \beta_0+\beta_1 \cdot \ln A_{it}+\beta_2 \cdot \ln L_{it}\\
	&+ \beta_3 \cdot \ln K_{it} - \beta_4 \cdot \tau_{it} +u_{it},
 \end{split}
\end{equation}

where $\alpha=\beta_1$, $\lambda=\beta_2$, $\phi=\beta_3$, $1/\gamma=\beta_4$ and the tax differential is depicted by $\tau_{it}$. The model given in \cref{eqnine} corresponds to the estimation equation from \textcite[1169]{huizinga_international_2008}. Their composite tax variable captures the incentive to shift income among any affiliates of the MNE (the average tax differential) and the opportunity to shift income (the scale of operations, proxied by sales). In contrast thereto, the tax differential $\tau_{it}$ only captures the incentive to shift income between the subsidiary and the parent, which is a direct consequence of the earlier mentioned limited access to ownership data, which makes it impossible to calculate an average tax differential.

\subsection{Extended model} \label{sec:Extended model}
The model extensions are motivated by the research presented in \cref{sec:Literature review} and allow to study income shifting in greater detail. These studies identify individual opportunities to shift income based on observable firm and country characteristics. Other than \textcite{huizinga_international_2008}, who assume the scale of operations of an affiliate in a given country is the sole driver of income shifting, additional factors are considered here. Four drivers are examined. These drivers are the scale of operation \textcite{huizinga_international_2008}, the amount of intangible assets \parencites{beer_profit_2015}{dischinger_corporate_2011}, the ownership share \parencites{weichenrieder_profit_2009}{dischinger_profit_2008} and the direction of income shifting \parencite{dischinger_role_2014}. These drivers extent the basic model in \cref{eqnine} by introducing interaction terms consisting of the drivers and the tax differential. This procedure is standard in the literature and is used by all authors mentioned above, however, an in-depth analysis of the interaction terms and corresponding marginal effects is often lacking. Therefore, the recommendations of \textcite[64]{brambor_understanding_2006} and \textcite[660]{berry_improving_2012} are followed. Their main recommendations are to include all variables that constitute an interaction term, calculate marginal effects and corresponding standard errors for a substantive range of the variables involved, and present the results thereof in an informative way.\footnote{The authors come from the field of politics, however, their recommendations apply irrespective of the subject. The authors find that their main recommendations are largely ignored in a survey of political studies. The articles presented in \cref{sec:Literature review} almost always include all variables constituting an interaction (an exception are \textcite[1169]{huizinga_international_2008} who state that their composite tax variable is the product of two terms, but include only the product and not the single terms). But, interaction terms are rarely analyzed at the level of detail proposed by \textcite[73-77]{brambor_understanding_2006} (an exception are \textcite[444]{beer_profit_2015} who include a figure of a marginal effect of an interaction term).}

\textcite[1167-1168]{huizinga_international_2008} argue that the amount of income shifting depends on the affiliate's scale of operations in a given country. A firm with a large scale of operations finds it easier to shift income than a firm with a small scale of operations. To account for this possibility, an interaction term consisting of the scale of operations and the tax differential is added to the model in \cref{eqnine}. Other than in \textcite[1172]{huizinga_international_2008}, the capital input $K_{it}$ is used instead of sales to proxy for the scale of operations. This is based on \textcite[319]{haufler_corporate_2000} and has the advantage of presumably being a more stable measure and being less distorted by income shifting \parencite[1174]{huizinga_international_2008}. It is further convenient as no additional variable has to be included when interacting the tax differential with the capital input. The standalone terms are kept in the model. \textcite[428]{beer_profit_2015} argue that intangible assets such as trademarks, patents and copyrights are difficult to value and provide opportunities for tax savings. Thus, a strategic transfer price of an intangible asset deviating from the arm's length price is less costly to conceal from the tax authority and income shifting increases with the amount of intangibles. This argument is followed and an interaction term consisting of the amount of intangibles, $I_{it}$, and the tax differential is introduced. Both, the interaction term and $I_{it}$ as a standalone term are added to \cref{eqnine}. A similar argument, with income shifting depending on the ownership share of the parent, is made by \textcite[5]{dischinger_corporate_2008}. He argues that income shifting strategies with subsidiary $i$ are hard to implement if the ownership share of the parent in this subsidiary is small and hence, tax savings from income shifting are hard to realize. Dischinger measures the ownership with a continuous variable, whereas \textcite[292]{weichenrieder_profit_2009} uses a categorical variable to distinguish wholly-owned subsidiaries from non wholly-owned subsidiaries. The later approach is followed here, and two categorical ownership variables are introduced and each interacted with $\tau_{it}$. $\text{OW\_51}_{it}$  is equal to 1 if the ownership share is between 51 and 99.99\% and 0 otherwise, and $\text{OW\_100}_{it}$ is equal to 1 if the subsidiary is wholly-owned and 0 otherwise. Since time-invariant variables cannot be used in a fixed effects (FE) model, solely the interaction terms are added to \cref{eqnine}.\footnote{The ORBIS database reports the ownership for the last available year only \textcite[430]{dharmapala_what_2014}. For the purpose of the empirical analysis, it is assumed that the ownership share remained unchanged as it was in the year 2015. \textcite[9]{dischinger_corporate_2008} mentions that making this assumption does not constitute a serious threat to the validity of his results. But making this assumption results in having a time-invariant ownership variable. As a consequence, only the interaction terms but not the standalone terms can be added to \cref{eqnine}. \Cref{sec:Limitations} elaborates and mentions that the coefficient estimates of such a procedure are possibly biased.} \textcite[249, 251]{dischinger_role_2014} provide several reasons why income shifting to the parent is less costly than income shifting to the subsidiary. Among them are withholding taxes upon repatriation of foreign income as dividends and managers preferring having funds under control rather then having them overseas. To implement this argument, the case distinction from \cref{fig1} is applied to the model. In case 1, income is shifted to the subsidiary $(r_{it} < r_{ht}$, $s_{it} > 0)$, and in case 2, income is shifted to the parent $(r_{it} > r_{ht}$, $s_{it} < 0)$. The variable $Case2_{it}$ equal to 1 if the shifting direction is towards the parent (in case 2) and 0 otherwise (in case 1), is introduced and interacted with $\tau_{it}$. Both the interaction and the standalone term are added to the model in \cref{eqnine}. The extended model is constructed by combining the second line of \cref{eqnine} and the additional standalone and interaction terms mentioned above, and is given for subsidiary $i$ in year $t$ by

%---Equation 10---
\begin{equation}\label{eqten}
 \begin{split}
  \ln \Pi_{it} & = \underbrace{\beta_0 + \beta_1 \cdot \ln A_{it} + \beta_2 \cdot \ln K_{it} + \beta_4 \cdot \tau_{it} + u_{it}}_{\text{basic model from \cref{eqnine}}}\\
	& - \underbrace{\beta_5 \cdot \tau_{it} \times \ln K_{it}}_{\text{capital interaction}} + \beta_6 \cdot \ln I_{it} -\underbrace{\beta_7 \cdot \tau_{it} \times \ln I_{it}}_{\text{intang. interact.}}\\
	& - \underbrace{\beta_8 \cdot \tau_{it} \times \text{OW\_51}_{it}}_{\text{$1^{st}$ ownership interact.}} - \underbrace{\beta_9 \cdot \tau_{it} \times \text{OW\_100}_{it}}_{\text{$2^{nd}$ ownership interact.}}\\
	& + \beta_{10} \cdot Case2_{it} - \underbrace{\beta_{11} \cdot \tau_{it} \times Case2_{it}}_{\text{direction interaction}}.
 \end{split}
\end{equation}

\cref{eqten} constitutes the extended model. Note that the interaction terms enter with a negative sign. This is due to the definition of the tax differential in \cref{eqfour} and the deliberate coding of the categorical variables included in the interaction terms.\footnote{The categorical variables $\text{OW\_51}_{it}$, $\text{OW\_100}_{it}$ and $Case2_{it}$ are coded in the same way as in the original papers. Doing so facilitates the comparison of the results.} The extended model allows to compute and interpret specific marginal effects, although this advantage comes at the price of increased model complexity. The subsequent subsection is intended to give an idea on model interpretation and how the extended model can be used to gain a more detailed insight into income shifting activities of Swiss MNEs.

\subsection{Model interpretation and marginal effects} \label{sec:Model interpretation and marginal effects}
A negative sign of $\beta_4$ is in line with income shifting \parencite[447]{becker_cross-border_2012}. If $\tau_{it}$ is negative (case 1, $r_{it} < r_{ht}$, $s_{it} > 0$), income is shifted to the subsidiary. As a consequence, the subsidiary's total income $\Pi_{it}$ increases. A negative sign of $\beta_4$ in \cref{eqnine,eqten} reflects case 1 correctly. If $\tau_{it}$ is positive (case 2, $r_{it} > r_{ht}$, $s_{it} < 0$), income is shifted to the parent. As a consequence, the subsidiary's total income $\Pi_{it}$ decreases. A negative sign of $\beta_4$ also reflects case 2 correctly. Moreover, the marginal effect of $\tau_{it}$ on $\Pi_{it}$ in the basic model is given by the partial derivative $\partial \ln \Pi_{it}/\partial \tau_{it}=-\beta_4$ \textcite[11]{buettner_anti_2017}. Since the basic model is written in a log-level specification, the marginal effect can be interpreted as a semi-elasticity \parencite[43-46]{wooldridge_introductory_2009}. That is the percentage change in $\Pi_{it}$ associated with a 1 percentage point change in $\tau_{it}$ \parencite[429]{dharmapala_what_2014}. An estimate of -1.5 translates into a 1.5\% decrease in $\Pi_{it}$ due to a 1 percentage point increase in $\tau_{it}$, where the increase in $\tau_{it}$ is either a result of an increase in $r_{it}$ or a decrease in $r_{ht}$.

The extended model in \cref{eqten} is harder to interpret because the influence of $\tau_{it}$ depends on the drivers of income shifting. Partially differentiating \cref{eqten} w.r.t. $\tau_{it}$ yields

%---Equation 11
\begin{equation}\label{eqeleven}
 \begin{split}
  \frac{\partial \ln \Pi_{it}}{\partial \tau_{it}} &= -\beta_4 -\beta_5 \times \ln K_{it} -\beta_7 \times \ln I_{it}\\
	&- \beta_8 \times \text{OW\_51}_{it} - \beta_9 \times \text{OW\_100}_{it}\\
	&-\beta_{11} \times Case2_{it},
 \end{split}
\end{equation}

which varies with the capital input, the amount of intangibles and the categorical ownership and shifting direction variables. The marginal effect in \cref{eqeleven} still represents a semi-elasticity, and the coefficient estimates of the interaction terms from \cref{eqten} depict how much the semi-elasticity changes w.r.t. the drivers of income shifting. For example, an estimate of $\beta_5$ of $0.5$ results in a decrease (an absolute increase) of the marginal effect by $-1.5$, given $\ln K_{it}=3$, an ownership share between 10 and 50.99\% and shifting direction to the subsidiary (the three last terms in \cref{eqeleven} fall away). Given these characteristics, $\beta_4$ now depicts the marginal effect for a subsidiary with no capital. This example highlights two important points when it comes to interaction terms. First, marginal effects according to \cref{eqeleven} reflect firm specific opportunities to shift income. For example, in the above setting the marginal effect for a subsidiary with capital of $\ln K_{it} = 3$ is $-3$, which is twice as large as the marginal effect for a subsidiary with no capital. Thus, the extended model allows to study income shifting behavior among Swiss MNEs in greater detail. Secondly, it highlights the importance of calculating meaningful marginal effects and standard errors as proposed by \textcite[74]{brambor_understanding_2006}. A typical regression table reports coefficient estimates only. However, the estimate of $\beta_4$ is of little use in the extended model, since the sample includes no observations with zero capital. Although a regression table allows to calculate marginal effects for various combinations and levels of the regressors, there is no way of calculating correct standard errors and therefore, the significance of the marginal effect can not be assessed. Consider again the above example. The marginal effect for a subsidiary with the given characteristics is $\partial \ln \Pi_{it}/\partial \tau_{it}=\beta_4 - \beta_5 \times \ln K_{it}$, and the standard error (SE) is given by

%---Equation 12---
\begin{equation}\label{eqtwelve}
 \begin{split}
  \text{SE}\left(\frac{\partial \ln \Pi_{it}}{\partial \tau_{it}}\right) &= [\text{var}\beta_4 + (\ln K_{it})^2 \cdot\text{var}\beta_5\\
&+2 \cdot \ln K_{it} \cdot \text{cov}(\beta_4 \beta_5)]^{1/2}
 \end{split}
\end{equation}

The formula can be found in \textcite[70]{brambor_understanding_2006} and \textcite[16]{aiken_multiple_1991}, who also provide a variety of standard errors for common interaction models.\footnote{The formulas can be assessed on Golder's website \parencite{golder_interactions_2017}. The formulas allow to calculate standard errors for all marginal effects presented in \cref{sec:Basic model results,sec:Extended model results including single interactions,sec:Extended model results including multiple interactions}.} Typically, this quantity can not be assessed by the reader, as covariance terms are seldom reported. Therefore, a strong focus is put on meaningful marginal effects and their graphical representation in \cref{sec:Empirical results} when presenting the estimation results.

\subsection{Estimation approach} \label{sec:Estimation approach}
\textcite[1172-1173]{huizinga_international_2008} estimate their model by ordinary least squares (OLS). This is reasonable since their data is a cross-section from 1999. The panel dataset in this thesis allows to control for unobserved heterogeneity among subsidiaries. As a consequence, FE estimation is used here.\footnote{FE estimation is preferred over random effects (RE) estimation, based on the results of a Hausman specification test. The result of this test is provided in \cref{app:B2}.} A vector of yearly categorical variables to model a time trend, and a vector of categorical variables depicting the industry-affiliation interacted with the yearly categorical variables are added to the model in \cref{eqten}. The European classification of economic activities (NACE) rev. 2 main sector codes are used to distinguish industries \parencite[57]{european_commission_nace_2008}. The sample is limited to firms in the manufacturing and wholesale and retail industries, where a Cobb-Douglas production function to describe output presumably seems appropriate \parencite[for example][1172]{huizinga_international_2008}. This restriction is relaxed in robustness tests in \cref{sec:Robustness}. The FE estimation approach is fully described below in \cref{eqthirteen}. Since the basic model from \cref{eqnine} is nested within the extended model from \cref{eqten}, this is done for the extended model only.

%---Equation 13---
\begin{equation}\label{eqthirteen}
 \begin{split}
  \ln \Pi_{it} & = \beta_0 + \beta_1 \cdot \ln A_{it} + \beta_2 \cdot \ln K_{it} + \beta_4 \cdot \tau_{it} \\
	& - \beta_5 \cdot \tau_{it} \times \ln K_{it} + \beta_6 \cdot \ln I_{it} -\beta_7 \cdot \tau_{it} \times \ln I_{it}\\
	& - \beta_8 \cdot \tau_{it} \times \text{OW\_51}_{it}- \beta_9 \cdot \tau_{it} \times \text{OW\_100}_{it}\\
	& + \beta_{10} \cdot Case2_{it} - \beta_{11} \cdot \tau_{it} \times Case2_{it}+ \bm{T_{it} \theta_{it}}\\
	& +\bm{U_{it}\xi_{it}} + \rho_i+ u_{it},
 \end{split}
\end{equation}

where
\vspace{0.5cm}

\begin{supertabular}{p{0.07\textwidth} p{0.37\textwidth}}
 $i$ & is the subsidiary index,\\
 $t$&is the time index ranging from 2007 to 2015,\\
 $\Pi_{it}$&is the total pre-tax income, measured as EBIT,\\
 $A_{it}$&is the technology input, proxied by the GDP per capita in local currency,\\
 $L_{it}$&is the labor input, measured as costs of employees,\\
 $K_{it}$&is the capital input, measured as fixed assets,\\
 $\tau_{it}$&is the tax differential, calculated as $\tau_{it}=(r_{it}-r_{ht})$\\
 $I_{it}$& are fixed intangible assets,\\
 $\text{OW\_51}_{it}$ &is a categorical variable equal to 1 if the subsidiary is owned by a Swiss parent firm with a share between 51 and 99.99\% and 0 otherwise,\\
 $\text{OW\_100}_{it}$ &is a categorical variable equal to 1 if the subsidiary is wholly-owned by a Swiss parent firm and 0 otherwise,\\
 $Case2_{it}$&is a categorical variable equal to 1 if the income shifting direction is from the subsidiary to the parent firm in Switzerland and 0 otherwise,\\
 $\bm{T_{it}}$&is a vector of dimensionality $(1\times8)$, indicating which year the observation falls into, $\bm{T_{it}}=$ ($T_{08}$ $T_{09}$ $T_{10}$ $T_{11}$ $T_{12}$ $T_{13}$ $T_{14}$ $T_{15}$), the year 2007 being the reference category, and each of the categorical variables $\bm{T_t}$ being equal to 1 if the observation falls into year $t$ and 0 otherwise,\\
 $\bm{\theta_{it}}$&is a vector of dimensionality $(8\times1)$, holding the coefficients for each year except 2007, $\bm{\theta_{it}}'=$ ($\theta_{08}$ $\theta_{09}$ $\theta_{10}$ $\theta_{11}$ $\theta_{12}$ $\theta_{13}$ $\theta_{14}$ $\theta_{15} $)\\
 $\bm{U_{it}}$&is a vector of dimensionality $(1\times8)$, holding industry-year categorical variables, each being equal to 1 if the observation falls into that industry in that year and 0 otherwise,\\
 $\bm{\xi_{it}}$&is a vector of dimensionality $(8\times1)$, holding the coefficients for the industry-year categorical variables\footnote{See Appendix B.1 for additional comments on matrix algebra and the NACE rev. 2 industry classification. The dimensionality of the vectors Uit and ?it depends on the number of industries included in the analysis.}\\
 $\rho_i$&is the subsidiary-fixed effect, and\\
 $u_{it}$& is the error term.\\
\end{supertabular}
\vspace{0.5cm}

Directly estimating \cref{eqthirteen} is one way of studying income shifting among Swiss MNEs. In order to get a more complete picture of the underlying income shifting patterns, various adaptations of \cref{eqthirteen} are estimated. In a first step, the basic model excluding all interaction terms is estimated. In a second step, \cref{eqthirteen} including single drivers of income shifting separately is estimated. In a third step, all drivers of income shifting are studied within the same model, which is done by directly estimating \cref{eqthirteen} as stated above. The following section describes the data used in estimation and \cref{sec:Empirical results} presents the empirical estimation results.

%---5---
\section{Data and sample} \label{sec:Data and sample}
\subsection{Financial data and country statistics} \label{sec:5.1}

%---Table 1---
\begin{table*}[t]
\footnotesize
 \begin{center}
  \caption{ORBIS search strategy and data restrictions}\label{tab1}
   \begin{tabularx}{\textwidth}{p{0.665\textwidth}r r r r}
\toprule
   Search step / restriction &Subsidiaries &Percentage &Last step\\
   \midrule
   All active companies and companies with unknown situation &177'064'116 &100.000\% &100.000\%\\
   Worldwide companies &174'596'526 &98.606\% &98.606\%\\
   Subsidiaries with a non-missing data value in at least one year between 2007 and $2015^a$ &6'968'497 &3.936\% &3.991\%\\
   ubsidiaries owned by at least one shareholder located in Switzerland, owning between 10 and 100\% of the shares &10'070 &.006\% &.145\%\\
   Exclusion of subsidiaries with no tax or GDP data available& 10'066 &.006\% &99.960\%\\
   Exclusion of subsidiaries located in Switzerland &9'812 &.006\% &97.438\%\\
   Subsidiaries in manufacturing and wholesale and retail industry where a Cobb-Douglass production function is appropriate &5'414 &.003\% &55.177\%\\
   Exclusion of observations with insufficient data$^b$ &4'862 &.003\% &89.804\%\\
\bottomrule
\end{tabularx}
\caption*{\footnotesize{\textit{Notes}. $^a$This search step has been performed separately on the variables fixed assets, cost of employees and EBIT. These steps are included to reduce the number of empty cells downloaded. $^b$Observations with negative EBIT, fixed assets and cost of employees are excluded because logarithms are not defined for negative numbers. Further, the incentive to shift income is mitigated or inexistent for loss-making subsidiaries. The data was downloaded on Mar 27, 2017. Percentage shows the search result in percent of the ORBIS database and last step shows the result in percentage of the previous step. Source: own table.}}
 \end{center}
\end{table*}

%---Table 2---
\begin{table*}[!]
\footnotesize
 \begin{center}
  \caption{Summary statistics}\label{tab2}
   \begin{tabularx}{\textwidth}{l *{7}{Y} r}
   \toprule
   Variable& Minimum &$25^{th}$ Perc. &Median &Mean &$75^{th}$ Perc. &Maximum &Stan. dev. &No. of obs.\\
   \midrule
   ln EBIT, ($\Pi_{it}$) &3.741 &12.054 &13.484 &13.457 &14.486 &22.415 &2.162 &26'869\\
   ln P/L before tax &2.499 &11.946 &13.448 &13.389 &14.854 &22.486 &2.246 &25'423\\
   ln total assets &9.218 &14.831 &16.085 &16.121 &17.305 &24.417 &1.918 &26'869\\
   ln fixed assets, ($K_{it}$) &1.756 &12.031 &13.980 &13.942 &15.812 &24.063 &2.752 &26'869\\
   ln intangible fixed assets, ($I_{it}$) &-4.560 &8.853 &10.890 &10.921 &12.915 &23.233 &3.171 &17'897\\
   ln cost of employees, ($L_{it}$) &0.902 &13.373 &14.624 &14.564 &15.785 &22.467 &1.945 &26'869\\
   ln number of employees, (\textit{L\_N}$_{it}$) &0 &2.639 &3.738 &3.817 &4.977 &12.000 &1.763 &22'193\\
   Leverage$^a$, ($LEV_{it}$) &-7.959 &-0.862 &-0.453 &-0.615 &-0.204 &4.018 &0.623 &26'577\\
   ln GDP per capita, ($A_{it}$) &8.481 &10.204 &10.359 &10.833 &10.549 &17.618 &1.362 &26'869\\
   Local tax rate, ($r_{it}$) &0 &0.220 &0.294 &0.268 &0.314 &0.550 &0.064 &26'869\\
   Tax differential, ($\tau_{it}$) &-0.206 &0.031 &0.105 &0.082 &0.133 &0.360 &0.063 &26'869\\
   $\text{OW\_51}_{it}$ &0 &0 &0 &0.125 &0 &1 &0.331 &26'869\\
   $\text{OW\_100}_{it}$ &0 &0 &1 &0.699 &1 &1 &0.459 &26'869\\ 
   $Case2_{it}$ &0 &1 &1 &0.901 &1 &1 &0.298 &26'869\\   
   \bottomrule
  \end{tabularx}
 \caption*{\footnotesize{\textit{Notes}. $^a$Leverage is calculated as ln debt over ln total assets, therefore ratios below 1 are negative after taking natural logarithms. The maximum GDP per capita is below the maximum EBIT and P/L before tax. This seems unreasonable, but it should be kept in mind that the GDP per capita is measured in local currency units, whereas all other financial data is measured in Swiss Francs (CHF). Source: own table.}}
 \end{center}
\end{table*}

The sample period covers 9 years, ranging from 2007 to 2015. The financial firm data is retrieved from the ORBIS database provided by the Bureau Van Dijk (BvD), which also provides access to the AMADEUS database, the European subset of ORBIS.\footnote{A short overview on the ORBIS database can be found in \textcite{ribeiro_oecd_2010}.} \textcite[14]{kalemli-ozcan_how_2015} mention that the ORBIS and AMADEUS databases do not overlap in 100\% of the cases and the coverage being higher in the ORBIS database. Therefore, the ORBIS database is used. The main variables downloaded include: EBIT, profit/loss (P/L) before tax, total assets, fixed assets, tangible fixed assets, intangible fixed assets, debt, equity, the number and costs of employees, the country's international standard organization (ISO) code, the BvD identification number (ID) and the NACE rev. 2 main sectors. The GDP data is taken from the World Bank Databank world-development-indicators \parencite{world_bank_databank_2017}. Subsidiaries included in the sample are firms with a Swiss parent firm holding at least 10\% of the shares. The sample firms are allowed to have subsidiaries themselves. The BvD ID number is used as the subsidiary identifier within the dataset. The data is downloaded through the BvD web interface and access is provided by the University of Zurich. A major advantage of the ORBIS database is the possibility to link ownership data with accounting data to create MNE-panel datasets \parencite[for example][430]{beer_profit_2015}. Thus, it is possible to link parent firms with all available subsidiaries. Unfortunately, the access provided by the University of Zurich does not include the ownership data. It is therefore not possible to link parent firms to subsidiaries and vice versa. However, another feature of the database can be exploited to retrieve some ownership information. To compute the tax differential $\tau_{it}$, it is not necessary to know which firm is the owner of a foreign subsidiary. It is sufficient to select subsidiaries hold by a Swiss parent firm and download the countries of residence of these subsidiaries. By downloading firms with a Swiss shareholder, a range for the ownership share can be specified, however, the dataset will not contain the specific ownership shares of the individual firms. This method is used to download firms that are owned by a Swiss parent firm by at least 10\%.\footnote{To ensure that the subsidiaries are part of a MNE and not owned by an individual person, the following types of shareholders are selected: banks and financial companies, insurance companies, industrial companies, private equity firms, hedge funds, venture capital, mutual and pension funds, nominees, trusts, trustees, foundations, research institutes, public authorities, states and governments.} \Cref{tab1} summarizes the search strategy and data trimming procedures used to retrieve the data from ORBIS.

In a second step, the search strategy from \Cref{tab1} is reused to download subsidiaries with an ownership share between 51 and 99.99\%. In a third step, firms with an ownership of 100\% are downloaded. Matching the three datasets on the BvD ID numbers and the calendar year allows to distinct three categories. The first category includes subsidiaries with an ownership share between 10 and 50.99\%. The second category includes subsidiaries with an ownership share between 51 and 99.99\%, for which the categorical variable $\text{OW\_51}_{it}$ equals 1. The third category includes wholly-owned subsidiaries for which the variable $\text{OW\_100}_{it}$ equals 1. The lower bound of the ownership is chosen as low as 10\% for the following reasons. First, a shareholder can reach a controlling interest with less than 51\% in case different share classes with differing voting rights exist and secondly, affiliated shareholders can reach a controlling interest combining their voting rights. The low boundary ensures to detect potential income shifting in these cases. If these cases are irrelevant, the above approach is likely to understate the extent of income shifting, which is considered less severe than overstating.

%---Table 3---
\begin{table*}[t]
\footnotesize
 \begin{center}
  \captionsetup{width=0.72\textwidth}
   \caption{Pairwise correlations of main variables}\label{tab3}
    \begin{tabularx}{0.72\textwidth}{l *{6}{Y}}
     \toprule
     Variable &$\Pi_{it}$ &$K_{it}$ &$L_{it}$ &$A_{it}$ &$I_{it}$ &$LEV_{it}$\\
     \midrule
     ln fixed assets, ($K_{it}$) &0.703 & & & &\\
     &(26'869) & & & &\\
     ln costs of employees, ($L_{it}$) &0.770 &0.736 & & &\\
     &(26'869) &(26'869) & & &\\
     ln GDP per capita, ($A_{it}$) & 0.038 &-0.001 &-0.011 & &\\
     &(26'869) &(26'869) &(26'869) & &\\
     ln intangible assets, ($I_{it}$)& 0.510 &0.640 &0.571 &0.013 &\\
     &(17'897) &(17'897) &(17'897) &(17'897)\\
     Leverage, ($LEV_{it}$) &-0.091 &-0.059 &0.014 &-0.059 &0.036\\
     &(26'577) &(26'577) &(26'577) &(17'680) &(26'577)\\
     Tax differential, ($\tau_{it}$) &0.047 &0.023 &0.166 &-0.277 &0.124 &0.129\\
     &(26'869) &(26'869) &(26'869) &(26'869) &(17'897) &(26'577)\\
    \bottomrule
   \end{tabularx}
  \caption*{\footnotesize{\textit{Notes}. The number of observations is given in parentheses. Source: own table.}}
 \end{center}
\end{table*}

%---Figure 2---
\begin{figure*}
 \centering \captionsetup{width=0.95\textwidth}
  \includegraphics[scale=1]{fig2.pdf} 
 \caption[Spatial distribution of subsidiaries]{Spatial distribution of subsidiaries. Countries with no subsidiaries are blank. The number of subsidiaries is presented in \cref{tab4}. The number of subsidiaries have been log-transformed to get a meaningful color scale. A detailed map of Europe is provided in \cref{app:C3}. Source: own figure.} \label{fig2}
\end{figure*}

The raw dataset downloaded from ORBIS includes financial data on 10'070 subsidiaries, of which 4'862 are suitable for the empirical analysis. The average subsidiary is observed over 5.53 years, resulting in a total sample size of 26'869 observations. The data restrictions shown in \Cref{tab1} represent the sample that is used to estimate the basic model and most of the variations of the extended model. The sample size is reduced when intangibles are included in the model. The sample statistics provided in \Cref{tab2} refer to the sample described in \Cref{tab1}. Additionally, the distributions of some of these variables are shown in \Cref{app:C1}. The regression results in \cref{sec:Empirical results} state the sample size for each model specification. The continuous variables are highly skewed to the right. Natural logarithms are taken to counteract. Looking at means, medians, 25 and 75 percentiles of the continuous variables, taking natural logarithms seems to result in useful distributions. The row of intangibles in \Cref{tab2} shows that the sample size is reduced to 17'897 observations when intangible assets are included. The tax differential $\tau_{it}$ is mostly positive, indicating that only few observations in the dataset have a smaller CITR than Switzerland, which is tantamount with most subsidiaries being faced with an incentive to shift income to Switzerland. This is confirmed by the mean of $Case2_{it}$, which shows that 90.01\% of the observations have an incentive to shift income to Switzerland. Tax rates and tax differentials are discussed in more detail in \cref{sec:Tax rates and tax differentials}. The means of the categorical variables $\text{OW\_51}_{it}$ and $\text{OW\_100}_{it}$ show that 12.5\% of the subsidiaries are owned with an ownership share between 51 and 99.99\%, 69.9\% of the subsidiaries are wholly-owned and 17.6\% of the subsidiaries are owned with an ownership share between 10 and 50.99\%.

\cref{tab3} shows the pairwise correlation coefficients among the main variables used. Continuous variables have been log-transformed before calculations. The inputs to the Cobb-Douglas production function, namely fixed assets, costs of employees and the GDP per capita are all positively related to the EBIT. A positive coefficient is observed for the correlation between the EBIT and the tax differential. A negative coefficient is in line with income shifting. Hence, the data from \Cref{tab3} does not suggest income shifting being present. However, the coefficients caption only pairwise correlations, thus neglecting other, more complicated interdependencies in the dataset. A multi-variate analysis is essential.

The sample contains financial data of subsidiaries from 63 countries. \cref{tab4} shows the number of subsidiaries per country. A vast majority of subsidiaries is residing in Europe. Subsidiaries from France, Germany, Italy, Spain and the United Kingdom (UK) make up roughly half of the sample size. A lot of subsidiaries furthermore reside in Eastern European countries such as the Czech Republic, Hungary, Poland and Romania. It is striking how few subsidiaries from Asia, America and Africa are included in the sample. Further, almost no observation from tax havens are included in the sample. European subsidiaries might be numerous for two reasons. First, Swiss MNEs could be less likely to have subsidiaries in distant locations, and secondly, different regions might be unequally covered in ORBIS. 

Various authors, among them \textcite[14]{cobham_international_2014}, \textcite[908]{clausing_effect_2016} and \textcite[18]{fuest_tax_2010} mention that the coverage for developing countries, especially countries in Africa, and tax havens is lower than for European countries.\footnote{\textcite[529-530]{desai_demand_2006} find that tax havens are used by US MNEs to avoid taxes. In case Swiss MNEs do so, the results presented in \cref{sec:Basic model results,sec:Extended model results including single interactions,sec:Extended model results including multiple interactions} might be understated. \Cref{sec:Limitations} elaborates on this issue.} However, this does not explain why there are few observations in developed countries outside of Europe, such as the US, Canada, Japan or Australia. \cref{fig2} presents a spatial distribution of the subsidiaries of Swiss MNEs across the world, amplifying the insights derived from \cref{tab4}.

%---Table 4---
\begin{table}
\begin{center}
 \captionof{table}{Subsidiaries per country}\label{tab4}
 \footnotesize
  \begin{tabularx}{\linewidth}{p{0.15\textwidth}r r r r r}
   \toprule
   Country &Subs. &Obs. &Subs.(\%) &Obs.(\%)\\
   \midrule
   Algeria &7 &(10) &0.14 &(0.04)\\
   Argentina &1 &(3) &0.02 &(0.01)\\
   Australia &1 &(2) &0.02 &(0.01)\\
   Austria &139 &(790) &2.86 &(2.94)\\
   Belgium &151 &(997) &3.11 &(3.71)\\
   Bermuda &2 &(13) &0.04 &(0.05)\\
   \scriptsize{Bosnia \& Herzegovina} &21 &(110) &0.43 &(0.05)\\
   Bulgaria &48 &(266) &0.99 &(0.99)\\
   Canada &1 &(1) &0.02 &(0.00)\\
   Chili &1 &(1) &0.02 &(0.00)\\
   Costa Rica &1 &(5) &0.02 &(0.02)\\
   Croatia &30 &(186) &0.62 &(0.69)\\
   Czech Republic &230 &(1'419) &4.73 &(5.28)\\
   Denmark &102 &(396) &2.10 &(1.47)\\
   Ecuador &2 &(5) &0.04 &(0.02)\\
   Estonia &23 &(132) &0.47 &(0.49)\\
   Finland &70 &(427) &1.44 &(1.59)\\
   France$a$ &543 &(3'155) &11.17 &(11.74)\\
   Germany &703 &(3'630) &14.46 &(13.51)\\
   Greece &1 &(9) &0.02 &(0.03)\\
   Hong Kong &2 &(15) &0.04 &(0.06)\\
   Hungary &87 &(556) &1.79 &(2.07)\\
   Iceland &1 &(3) &0.02 &(0.01)\\
   India &71 &(207) &1.46 &(0.77)\\
   Indonesia &4 &(34) &0.08 &(0.13)\\
   Ireland &24 &(133) &0.49 &(0.49)\\
   Israel &1 &(1) &0.02 &(0.00)\\
   Italy &741 &(4'192) &15.24 &(15.60)\\
   Japan &10 &(30) &0.21 &(0.11)\\
   Jamaica &1 &(4) &0.02 &(0.01)\\
   Kenya &3 &(18) &0.06 &(0.07)\\
   Kuwait &1 &(7) &0.02 &(0.03)\\
   Latvia &1 &(3) &0.02 &(0.01)\\
   Luxembourg &12 &(54) &0.25 &(0.20)\\
   Macedonia &16 &(34) &0.33 &(0.13)\\
   Malaysia &15 &(87) &0.31 &(0.32)\\
   Malta &2 &(8) &0.04 &(0.03)\\
   Montenegro &4 &(14) &0.08 &(0.05)\\
   Morocco &10 &(16) &0.21 &(0.06)\\
   Netherlands &101 &(431) &2.08 &(1.60)\\
   New Zealand &38 &(198) &0.78 &(0.74)\\
   Nigeria &2 &(18) &0.04 &(0.07)\\
   Norway &64 &(424) &1.32 &(1.58)\\
   Pakistan &3 &(19) &0.06 &(0.07)\\
   Philippines &1 &(9) &0.02 &(0.03)\\
   Poland &235 &(1'306) &4.83 &(4.86)\\
   Portugal &92 &(551) &1.89 &(2.05)\\
   Republic of Korea &52 &(350) &1.07 &(1.30)\\
   Romania &173 &(948) &3.56 &(3.53)\\
   Serbia &82 &(387) &1.69 &(1.44)\\
   Singapore &2 &(10) &0.04 &(0.04)\\
   Slovakia &91 &(517) &1.87 &(1.92)\\
   Slovenia &52 &(326) &1.07 &(1.21)\\
   South Africa &1 &(6) &0.02 &(0.02)\\
   Spain$^b$ &283 &(1'612) &5.82 &(6.00)\\
   Sri Lanka &1 &(9) &0.02 &(0.03)\\
   Sweden &145 &(866) &2.98 &(3.22)\\
   Thailand &14 &(31) &0.29 &(0.12)\\
   Ukraine &24 &(112) &0.49 &(0.42)\\
   \scriptsize{United Arab Emirates} &1 &(2) &0.02 &(0.01)\\
   UK &316 &(1'751) &6.50 &(6.52)\\
   US &2 &(8) &0.04 &(0.03)\\
   Uruguay&4 &(4) &0.08 &(0.01)\\
   \midrule
   Africa & &(68) & &(0.25)\\
   Americas & &(45) & &(0.17)\\
   Asia & &(811) & &(3.02)\\
   Oceania & &(200) & &(0.74)\\
   Europe & &(25'745) & &(95.82)\\
   World &4'862 &(26'869) &100.00 &(100.00)\\
   \bottomrule
  \end{tabularx}
 \captionof*{table}{\textit{Notes.} Numbers in parentheses represent observations numbers and the corresponding percentages. $^a$One firm (8 observations) residing in La Reunion is treated as residing in France. $^b$Two firms (17 observations) residing on the Canary Islands are treated as residing in Spain. The United Nations (UN) region classification used to assign countries to geographic regions can be found in \cref{app:C2}. Source: own table.}
 \end{center}
\end{table}

The world map suggests that the distance between Switzerland and the foreign country is correlated with the number of subsidiaries in that foreign country. \textcite[716]{clausing_effect_2016} finds that the distance between US subsidiaries and the parent firm and the income of these subsidiaries are negatively correlated. Her results favor the first argument why the sample contains few subsidiaries outside of Europe. However, it is not possible to exclude restrictions on data availability outside Europe as the reason for this supposed correlation. The data is used for the empirical analysis as presented here.

\subsection{Tax rates and tax differentials} \label{sec:Tax rates and tax differentials}
National tax rates are available at KPMG's corporate tax rates table website \parencite{kpmg_corporate_2017}.\footnote{KPMG does not provide an export function. Therefore the data is downloaded from Damodaran's website \parencite{damodaran_corporate_2017}. According to his wish, credit is given to KPMG.} KPMG provides yearly CITRs for countries worldwide. The tax rates are matched with the financial firm data using the ISO 2 country codes and the calendar year as matching variables, using the Stata package "kountry" \parencite{raciborski_kountry:_2008}. Aggregated summary statistics for tax rates and tax differentials are displayed in \cref{tab5}. The computation of national tax rates differs among countries and might affect the empirical analysis.\footnote{See \cref{app:C4} for peculiarities in national tax rate calculations potentially affecting the analysis.} Tax rates for Switzerland are calculated as the average of the tax rates in the capital cities of the cantons \parencite{kpmg_corporate_2017}.

%---Table 5---
\begin{table*}[t]
\footnotesize
 \begin{center}
  \caption{Tax rate statistics}\label{tab5}
   \begin{tabularx}{\textwidth}{l *{7}{Y} r}
    \toprule
    UN geographic region &Minimum &$25^{th}$ Perc. &Median &Mean &$75^{th}$ Perc. &Maximum &Stan. dev. &No. of Obs.\\
    \midrule
    \textit{Panel A}: Local tax rates, ($t_{it}$) \\
    \midrule
    Africa &0.000 &0.000 &0.280 &0.187 &0.300 &0.369 &0.150 &45\\
    Americas &0.000 &0.225 &0.265 &0.259 &0.333 &0.400 &0.111 &81\\
    Asia &0.000 &0.240 &0.280 &0.291 &0.350 &0.550 &0.108 &126\\
    Europe (whole continent) &0.090 &0.160 &0.220 &0.219 &0.280 &0.384 &0.075 &297\\
    Eastern Europe &0.100 &0.160 &0.190 &0.179 &0.190 &0.250 &0.040 &63\\
    Northern Europe &0.125 &0.180 &0.220 &0.216 &0.260 &0.300 &0.053 &81\\
    Southern Europe &0.090 &0.100 &0.200 &0.206 &0.300 &0.373 &0.092 &99\\
    Western Europe &0.250 &0.250 &0.294 &0.295 &0.333 &0.384 &0.037 &54\\
    Oceania &0.280 &0.280 &0.300 &0.296 &0.300 &0.330 &0.012 &18\\
    World &0.000 &0.190 &0.250 &0.241 &0.300 &0.550 &0.101 &567\\
    \midrule
    \textit{Panel B}: Tax differentials, ($\tau_{it}$)\\
    \midrule
    Africa &-0.206 &-0.188 &0.101 &0.000 &0.121 &0.165 &0.153 &45\\
    Americas &-0.206 &0.044 &0.086 &0.073 &0.146 &0.221 &0.111 &81\\
    Asia &-0.179 &0.052 &0.092 &0.105 &0.158 &0.371 &0.107 &126\\
    Europe (whole continent) &-0.116 &-0.021 &0.031 &0.033 &0.090 &0.177 &0.075 &297\\
    Eastern Europe &-0.106 &-0.028 &0.003 &-0.007 &0.011 &0.067 &0.041 &63\\
    Northern Europe &-0.081 &-0.026 &0.040 &0.030 &0.073 &0.108 &0.053 &81\\
    Southern Europe &-0.116 &-0.081 &0.019 &0.019 &0.111 &0.171 &0.092 &99\\
    Western Europe &0.044 &0.070 &0.109 &0.109 &0.148 &0.177 &0.037 &54\\
    Oceania &0.094 &0.101 &0.110 &0.110 &0.119 &0.124 &0.009 &18\\
    World &-0.206 &-0.002 &0.067 &0.054 &0.119 &0.371 &0.101 &567\\
    \bottomrule
   \end{tabularx}
  \caption*{\footnotesize{\textit{Notes}. The tax data is from KPMG (2017). Countries are assigned to geographic regions based on \textcite{united_nations_methodology_2017}, see \cref{app:C2}. All calculations are unweighted. The number of observations is equal to the number of countries per region multiplied by 9, since data is downloaded for 2007-2015. Source: own table.}}
 \end{center}
\end{table*}

%---Figure 3---
\begin{figure*}[!]
 \centering \captionsetup{width=0.95\textwidth}
   \includegraphics[scale=1]{fig3.pdf}
 \caption[Corporate tax rates across the world]{Corporate tax rates across the world. Solid lines represent unweighted mean tax rates, dashed lines depict minimum and maximum tax rates and the shaded area shows the mean tax rate $\pm1$ standard deviation. The red line depicts the Swiss tax rate. A detailed version of this figure is shown in \cref{app:C5}. Tax data is taken from \textcite{kpmg_corporate_2017}. Countries are assigned to geographic regions based on \textcite{united_nations_methodology_2017}, see \cref{app:C2}. Source: own figure.} \label{fig3}
\end{figure*}

%---Figure 4---
\begin{figure*}[!]
 \centering \captionsetup{width=0.95\textwidth}
   \includegraphics[scale=1]{fig4.pdf}
 \caption[Corporate tax rates across Europe]{Corporate tax rates across Europe. Solid lines represent unweighted mean tax rates, dashed lines depict minimum and maximum tax rates and the shaded area shows the mean tax rate $\pm1$ standard deviation. The red line depicts the Swiss tax rate. Tax data is taken from \textcite{kpmg_corporate_2017}. Countries are assigned to geographic regions based on \textcite{united_nations_methodology_2017}, see \cref{app:C2}. Source: own figure.} \label{fig4}
\end{figure*}

%---Figure 5---
\begin{figure*}[t]
 \centering \captionsetup{width=0.95\linewidth}
   \includegraphics[scale=1]{fig5.pdf}
 \caption[Differences in tax differentials in 2015 in Europe]{Differences in tax differentials in 2015 in Europe. The left map shows tax differentials for subsidiaries with incentive to shift to the parent firm $(r_{it} > r_{ht})$ and the right map shows tax differentials for subsidiaries with incentive to shift away from the parent firm $(r_{it} < r_{ht})$. Source: own figure.} \label{fig5}
\end{figure*}

As described in \cref{sec:Basic model}, the tax differential represents the incentive to shift income. Other factors possibly affecting the shifting incentive are neglected.\footnote{Another factor potentially affecting the incentive to shift income are withholding taxes. \textcite[15]{markle_comparison_2016} uses withholding taxes as part of his tax incentive variable. However, \textcite[289]{scholes_taxes_1992} note that the use of TP techniques typically avoids withholding taxes entirely. In this thesis, withholding taxes are potentially influential only when income is shifted out of Switzerland but is intended to be repatriated later. Since this could be the case for 10\% of the sample at most (see \cref{tab2}), withholding taxes are neglected.} Positive values of $\tau_{it}$ in \cref{tab5} represent an incentives to shift income to Switzerland. The mean tax differentials show that the Swiss tax rate is on average below the tax rates in almost all world regions. The mean tax differentials between Switzerland and Asia, the Americas and Western Europe are particularly high, indicating large tax saving opportunities. \cref{fig3,fig4} show mean tax rates over time. The Swiss tax rate is below the mean tax rate in the Americas, Asia, Europe and Oceania during all sample years. The mean African tax rate is below the Swiss tax rate during the first years of the sample period, but is well above the Swiss rate at the end of the sample period.

Since a vast majority of the sample subsidiaries is located in Europe, a closer comparison of European CITRs is desirable. \cref{fig4} allows to compare European tax rates in greater detail. The mean tax rate in Western Europe is higher than in the other regions during all sample years. The mean tax rates in Northern and Southern Europe show a slightly decreasing trend over the sample period, whereas the mean tax rates in Eastern and Western Europe show a more stable trend. The Swiss tax rate fell continuously over the sample period. Clearly, Swiss MNEs with subsidiaries in Western Europe have an incentive to shift income to Switzerland. Mean tax rates in Southern and Northern Europe are higher than the Swiss tax rate, although the difference is not as obvious as it is between Western Europe and Switzerland. Eastern European tax rates are comparable to the Swiss tax rate, leaving the country-specific incentives to shift income small or ambiguous.

Finally, \cref{fig5} provides a snapshot of the tax differentials between Switzerland and European countries for the most recent year 2015. The left map shows tax differentials $> 0$ (shifting incentive to the parent in Switzerland), and the right map shows tax differentials $< 0$ (shifting incentive away from the parent). Comparing the two maps indicates that most European countries are faced with an incentive to shift income to Switzerland in 2015. \cref{fig5} visually underlines the insights derived within this subsection. A complete overview on all variables and datasources used is provided in \cref{app:C6}.

%---6---
\section{Empirical results} \label{sec:Empirical results}
\subsection{Basic model results} \label{sec:Basic model results}

%---Table 6---
\begin{table*}[t]
\footnotesize
  \begin{center}
  \captionsetup{width=0.75\textwidth}
   \caption{Basic model results}\label{tab6}
    \begin{tabularx}{0.75\textwidth}{l *{4}{Y}}
    \toprule
    \multicolumn{5}{l}{Subsidiary-fixed effects, panel 2007-2015, dependent variable: ln EBIT, $\Pi_{it}$}\\
    \midrule
    Regression &Basic &Benchmark & Add. controls &Quadratic\\
    \cmidrule(lr){1-1}
    \cmidrule(lr){2-5}
    Explanatory variables &(1) &(2) &(3) &(4)\\
    \midrule
ln GDP per capita, $(A_{it})$			&$0.257^*$              &$0.235^*$           &$0.218$         &$0.237$   \\
                  		 				&$(1.926)$        &$(1.762)$      &$(1.630)$         &$(1.770)^*$  \\
ln fixed assets, $(K_{it})$      			&$0.064^{***}$ &$0.063^{***}$  &$0.066^{***}$  &$0.063^{***}$\\
                 						&$(5.162)$   &$(5.096)$    &$(5.283)$  &$(5.096)$\\
ln cost of employees, $(L_{it})$        		&$0.465^{***}$ &$0.465^{***}$ &$0.467^{***}$ &$0.465^{***}$ \\
                 						&$(13.549)$ &$(13.553)$ &$(13.453)$ &$(13.554)$ \\
Leverage, ln debt over            			&                           &        	                &$-0.254^{***}$        &         \\
     ln total assets, $(LEV_{it})$                  	&                      	    &	 			&$(-8.253)$      &         \\ 
GDP growth, $(GDP\_G_{it})$    		&                   &                  &$0.018^{***}$                 &  \\
                         				 		&                 &       		&$(4.646)$             &\\
Tax differential, $(\tau_{it})$            		&$-1.526^{***}$ &$-1.458^{***}$&$-1.139^{***}$&$-1.504^{***}$\\
                	      						&$(-3.564)$   &$(-3.390)$   &$(-2.671)$ &$(-2.445)$ \\ 
Tax differential squared, $((\tau_{it})^2)$   	&                  &                 &			&0.363   \\
                         						&              &            &			&(0.117)   \\
Year dummies &$\sqrt{}$ &$\sqrt{}$ &$\sqrt{}$ &$\sqrt{}$\\
Industry-year dummies & &$\sqrt{}$ &$\sqrt{}$ &$\sqrt{}$\\
No. of observations                   &26'869           &26'869           &26'577           &26'869   \\
No. of subsidiaries             &4'862        &4'862      &4'818       & 4'862   \\
Within $R^2$                &0.066           &0.066           &0.074           &0.066   \\
Overall $F$-test                 &51.143       &   31.386      &    32.912    &      29.889\\				
    \bottomrule
   \end{tabularx}
 \caption*{\footnotesize{\textit{Notes}. Regressions (1) and (2) are the basic and benchmark regression excluding and including a set of industry-year categorical variables. Regression (3) includes leverage and GDP growth as additional control variables. Regression (4) includes a quadratic term. $^*$, $^{**}$ and $^{***}$ denote significance on the 10, 5 and 1\% significance level. $t$-statistics are reported in parenthesis and standard errors are clustered at the subsidiary level to control for heteroscedasticity and autocorrelation \parencite[285]{hoechle_robust_2007}. Source: own table.}}
 \end{center}
\end{table*}

\cref{tab6} presents the results from estimating the basic model in \cref{eqnine} using the sample described in \cref{tab1}. Regressions (1) and (2) include a set of year, respectively a set of year and a set of industry-year categorical variables. Regression (2) constitutes the benchmark regression.\footnote{Regression diagnostics for the benchmark regression are presented in \cref{app:D1}.} Regression (3) includes frequently used control variables \parencites[293]{weichenrieder_profit_2009}[8]{lohse_impact_2012}. Regression (4) includes a quadratic term of the tax differential, allowing to check for a curvature in the relationship between the EBIT and the tax differential \parencite[162-163]{hines_fiscal_1994}. Thus, it can be verified whether the proposed model specification in \cref{eqnine} is appropriate or not.

The inputs of the Cobb-Douglas production function are positive and significant (with the exception of GDP per capita in regression (3)), meaning that higher inputs lead to higher output. These estimates can be interpreted as elasticities \parencite[45-46]{wooldridge_introductory_2009}. Specifically in regression (1), increasing the GDP per capita, the fixed assets or the costs of employees by 1\%, corresponds to an increase in the EBIT of 0.257\%, 0.064\% or 0.465\%, respectively. The tax differential enters significantly negative in all specifications, thus providing indirect evidence of income shifting among Swiss MNEs. The semi-elasticity of the EBIT w.r.t the tax differential in the benchmark regression (2) is $-1.458$, indicating that an increase in the tax differential by 1 percentage point is associated with a decrease in the EBIT of 1.458\%. It is irrelevant whether this increase in $\tau_{it}$ is due to an increase in the subsidiary's CITR or a decrease in the Swiss CITR. The semi-elasticities in regressions (1) and (3) and subsequent semi-elasticities (or marginal effects, the two terms are used interchangeably when appropriate), can be interpreted accordingly. The leverage and the GDP growth enter significantly negative, respectively significantly positive. The coefficients of leverage and GDP growth can be interpreted as semi-elasticities.

%---Figure 6---
\begin{figure}[t]
 \centering \captionsetup{width=0.95\linewidth}
  \includegraphics[scale=1]{fig6.pdf} 
 \caption[Marginal effect of the tax differential in the quadratic regression (4) in \cref{tab6}]{Marginal effect of the tax differential in the quadratic regression (4) in \cref{tab6}. The solid line shows the marginal effect according to the partial derivative of regression (4) in \cref{tab6}, and the shaded area represents the 90\% confidence interval. The grey bar indicates the range of the middle 90\% of the distribution of tax differentials (observations between the $5^{th}$ and $95^{th}$ percentile). Source: own figure, based on \textcite[661]{berry_improving_2012}} \label{fig6}
\end{figure}

The estimate of the tax differential in regression (4) cannot be interpreted without considering the estimate of the squared tax differential simultaneously. The marginal effect of the tax differential is given by $\partial \ln \Pi_{it}/\partial \tau_{it}=-1.504 + 0.363\times 2 \cdot \tau_{it}$.\footnote{Interpreting the coefficient of the tax differential as a conditional semi-elasticity when the tax differential is 0 is theoretically pointless since it corresponds to having no incentive to shift income.} \cref{fig6} plots the marginal effect over the complete range of tax differentials. The marginal effect is significant whenever the confidence intervals (CI) do not cross the zero-line. This is the case for more than 90\% of the tax differentials. However, the slope of the marginal effect in \cref{fig6} is close to 0, indicating that the interaction is economically meaningless. \textcite[662]{berry_improving_2012} describe such a marginal effect (with a slope close to 0) as evidence against a theory presuming an interaction effect. Hence, the model specification in \cref{eqnine} seems to be appropriate and a quadratic term is not included in subsequent analysis. An economic argument against the use of a quadratic term is the following. A flat tax rate implies that the tax saving for a subsidiary engaging in income shifting activities is proportional to the change in the tax differential. Whether the tax rate change happens at low or high tax differentials is irrelevant. However, \textcite[163]{hines_fiscal_1994} as well as \textcite[6, 8]{dowd_profit_2017} find that a quadratic term is statistically and economically meaningful in their analysis.

\subsection{Extended model results including single interactions} \label{sec:Extended model results including single interactions}

%---Table 7---
\begin{table*}[t]
\footnotesize
 \begin{center}
  \caption{Extended model results including single interactions}\label{tab7}
   \begin{tabularx}{\textwidth}{l c c c c c c c}
    \toprule
    \multicolumn{8}{l}{Subsidiary-fixed effects, panel 2007-2015, dependent variable: ln EBIT, $\Pi_{it}$}\\
    \midrule
    Income shifting driver &\multicolumn{2}{c}{Capital interaction} &\multicolumn{2}{c}{Intangibles interaction} & \multicolumn{2}{c}{Ownership interactions} &Direction int.\\
    \cmidrule(lr){1-1}
    \cmidrule(lr){2-3}
    \cmidrule(lr){4-5}
    \cmidrule(lr){6-7}
    \cmidrule(lr){8-8}
    Explanatory variables &(1) &(2) &(3) &(4) &(5) &(6) &(7)\\
    \midrule
    ln GDP per capita, $(A_{it})$ &$0.220^*$ &$0.228^*$ &$0.228$ &$0.226$ &$0.236^*$ &$0.236^*$ &$0.226^*$\\
     &$(1.651)$ &$(1.706)$ &$(1.344)$ &$(1.329)$ &$(1.777)$ &$(1.776)$ &$(1.651)$\\
    ln fixed assets, $(K_{it})$ &$0.089^{***}$ & &$0.081^{***}$ &$0.079^{***}$ &$0.063^{***}$ &$0.063^{***}$ &$0.063^{***}$\\
     &$(5.742)$ &$$ &$(4.028)$ &$(3.998)$ &$(5.087)$ &$(5.087)$ &$(5.094)$\\
    ln cost of employees, $(L_{it})$ &$0.464^{***}$ &$0.489^{***}$ &$0.460^{***}$ &$0.462^{***}$ &$0.465^{***}$ &$0.465^{***}$ &$0.465^{***}$\\
     &$(13.601)$ &$(14.463)$ &$(10.720)$ &$(10.718)$ &$(13.552)$ &$(13.551)$ &$(13.553)$\\
    Tax differential, $(\tau_{it})$ &$3.637*$ &$-0.396$ &$0.939$ &$-0.639$ &$-0.263$ &$-0.233$ &$-1.330$\\
     &$(1.950)$ &$(-0.689)$ &$(0.804)$ &$(-1.091)$ &$(-0.393)$ &$(-0.286)$ &$(-0.961)$\\
    Capital interaction, (cont., $\tau_{it}\times K_{it}$) &$-0.346^{***}$ & & & & & &\\
     &$(-2.816)$ & & & & & &\\
    ln fixed assets, (cat., $K\_d_{it}$) & &$0.221^{***}$ & & & & &\\
     & &$(3.750)$ & & & & &\\
    Capital interaction, (cat., $\tau_{it}\times K\_d_{it}$) & &$-1.690^{***}$ & & & & &\\
     & &$(-3.191)$ & & & & &\\
    ln intangible assets, $(I_{it})$ & & &$0.007$ & & & &\\
     & & &$(0.645)$ & & & &\\
    Intangible interaction, (cont., $\tau_{it}\times I_{it}$) & & &$-0.160^*$ & & & &\\
     & & &$(-1.816)$ & & & &\\
    ln intangible assets, (cat., $I\_d_{it}$) & & & &$-0.003$ & & &\\
     & & & &$(-0.054)$ & & &\\
    Intangibles interaction, (cat., $\tau_{it}\times I\_d_{it}$)& & & &$(-0.485)$ & & &\\
     & & & &$(-1.073)$ & & &\\
    $2^{nd}$ ownership interaction, ($\text{OW\_100}_{it}$) & & & & &$-1.839^{**}$ &$-1.868^*$\\
     & & & & &$(-2.205)$ &$(-1.946)$\\
    $1^{st}$ ownership interaction, ($\text{OW\_51}_{it}$) & & & & & &$-0.085$ &\\
     & & & & & &$(-0.058)$\\
    Shifting direction, $(Case2_{it})$ & & & & & & &$0.007$\\
     & & & & & & &$(0.115)$ \\
    Direction interaction, $(\tau_{it}\times Case2_{it})$ & & & & & & &$-0.179$\\
     & & & & & & &$(-0.117)$\\
    Year dummies &$\sqrt{}$ &$\sqrt{}$ &$\sqrt{}$ &$\sqrt{}$ &$\sqrt{}$ &$\sqrt{}$ &$\sqrt{}$ \\
    Industry-year dummies &$\sqrt{}$ &$\sqrt{}$ &$\sqrt{}$ &$\sqrt{}$ &$\sqrt{}$ &$\sqrt{}$ &$\sqrt{}$ \\
    No. of observations &26'869 &26'869 &17'897 &17'897 &26'869 &26'869 &26'869\\
    No. of subsidiaries &4'862 &4'862 &3'698 &3'698 &4'862 &4'862 &4'862\\
    Within $R^2$ &0.067 &0.065 &0.064 &0.064 &0.066 &0.066 &0.066\\
    Overall $F$-test &30.365 &28.078 &19.271 &19.017 &30.032 &28.670 &28.537\\
    \bottomrule
   \end{tabularx}
  \caption*{\footnotesize{\textit{Notes}. Regressions (1) and (2) include the capital interaction, (1) includes a continuous specification and (2) a categorical specification. Regressions (3) and (4) include the intangibles interaction, (3) includes a continuous specification and (4) a categorical specification. Regressions (5) and (6) include the ownership interactions, (5) includes only one interaction and (6) includes both interactions. Regression (7) includes the shifting direction interaction. $^*$, $^{**}$, $^{***}$ denote significance on the 10, 5, 1\% significance level. $t$-statistics are reported in parenthesis and standard errors are clustered at the subsidiary level to control for heteroscedasticity and autocorrelation \parencite[285]{hoechle_robust_2007}. Source: own table.}}
 \end{center}
\end{table*}

The results of estimating the extended model including single interaction terms are presented in \cref{tab7}. Each interaction term representing a driver of income shifting is analyzed separately. Regressions (1) and (2) test the theory that income shifting depends on the scale of operations, as proposed by \textcite{huizinga_international_2008}. Regression (1) includes an interaction term consisting of the tax differential and the capital input, measured as ln fixed assets. The interaction term in regression (2) uses the categorical variable $K\_d_{it}$, equal to 1 if the ln fixed assets are above mean and 0 otherwise. Regression (1) shows a significant positive coefficient of the tax differential and a significant negative coefficient of the interaction term. Since the inputs of the production function have been interpreted in the preceding section, the focus is put on the tax differential and the interaction terms. The coefficient of the tax differential represents the semi-elasticity of the EBIT w.r.t. the tax differential for subsidiaries with zero ln fixed assets, thus it is not meaningful to interpret this coefficient in isolation. The coefficient of the interaction term depicts how much the above mentioned semi-elasticity changes when ln fixed assets are increased. The marginal effect of the tax differential is given by the partial derivative and is equal to $\partial \ln \Pi_{it} / \partial \tau_{it}=3.637-0.346\times \ln K_{it}$, which is visualized in \cref{fig7}. Evaluation at the sample mean of ln fixed assets yields a significant effect of $-1.187^{***}$.\footnote{See \cref{tab2} in \cref{sec:5.1} for sample statistics.}

The minimum amount of ln fixed assets required for significant income shifting to be present is given by the point of intersection between the zero-line and the upper limit of the 90\% CI. \cref{fig7} shows that this is the case once the ln fixed assets have reached roughly 13. The slope of the marginal effect is given by the coefficient estimate of the interaction term and is equal to $-0.346$. The results from regression (1) imply that income shifting increases with fixed assets once a certain threshold of ln fixed assets is reached. A possible explanation for this result are fixed costs associated with income shifting. A subsidiary shifts income only if the benefits from shifting exceed the costs. The negative coefficient of the capital interaction indicates that the opportunities to shift income, and hence the benefits from income shifting, increase with the scale of operations. The marginal effect suggests that the benefits from in- come shifting exceed the costs once the ln fixed assets are larger than roughly 13. \textcite[423-424]{dharmapala_what_2014} mentions the possibility of fixed costs associated with income shifting and regression (1) supports this presumption. Regression (2) shows the same qualitative result as regression (1). The marginal effect of the tax differential is $-0.396$ $(-2.086^{***})$ for subsidiaries with below (above) mean ln fixed assets and the coefficient of the interaction term is $-1.690$, which is equal to the difference between the marginal effects. The marginal effect is significant for subsidiaries with above mean ln fixed assets only, which supports the insight regarding fixed costs derived from the results of regression (1).

%---Figure 7---
\begin{figure*}[t]
 \centering \captionsetup{width=0.95\textwidth}
   \includegraphics[scale=1]{fig7.pdf}
 \caption[Marginal effect of the tax differential in regressions (1) and (3) in \cref{tab7}]{Marginal effect of the tax differential in regressions (1) and (3) in \cref{tab7}. The solid line shows the marginal effect according to the partial derivative of regression (4) in \cref{tab6}, and the shaded area represents the 90\% confidence interval. The grey bar indicates the range of the middle 90\% of the distribution of tax differentials (observations between the $5^{th}$ and $95^{th}$ percentile). Source: own figure, based on \textcite[661]{berry_improving_2012}} \label{fig7}
\end{figure*}

Regressions (3) and (4) test whether income shifting depends on the intangible asset endowment of the subsidiary. Regression (3) includes an interaction term consisting of the tax differential and the intangible assets of the subsidiary, measured as ln intangible fixed assets. Regression (4) includes an interaction term consisting of the tax differential and the categorical variable $I\_d_{it}$, equal to 1 if the ln intangible fixed assets are above mean and 0 otherwise. The right graph of \cref{fig7} shows the marginal effect of the tax differential in regression (3), which is given by $\partial \ln \Pi_{it}/\partial\tau_{it}=0.939-0.160\times \ln I_{it}$. Evaluation at the sample mean of ln intangible fixed assets results in an insignificant marginal effect of $-0.808$. \cref{fig7} supports the presumption that income shifting involves fixed costs, suggesting that firms with ln intangible fixed assets larger than roughly 11.5 significantly shift income. Regression (4) conveys the same qualitative result, the marginal effect of the tax differential is $-0.639$ $(-1.124^{**})$ for subsidiaries with below (above) mean ln intangible fixed assets. The difference between the two effects is not significant as indicated by the interaction term of $-0.485$ with a low $t$-statistic.\footnote{See \cref{app:D2} for additional comments on this result.} Regressions (3) and (4) provide moderate support for the theory that income shifting increases with the amount of intangible assets.

Regressions (5) and (6) test the theory that income shifting increases with the ownership share. Regression (5) includes the categorical ownership variable $\text{OW\_100}_{it}$ interacted with the tax differential. This procedure is comparable to \textcite[285]{weichenrieder_profit_2009}, with the exception that it is applied to both shifting directions. Regression (6) additionally includes the ownership variable $\text{OW\_51}_{it}$ interacted with the tax differential. Note that the ownership variables can not be included as standalone variables since they do not vary over time. The marginal effects of the tax differential in regressions (5) and (6) are

%---Eq14---
\begin{equation}\label{eq14}
\footnotesize
 \begin{split}
  & \frac{\partial \ln \Pi_{it}}{\partial \tau_{it}} =
   \begin{cases}
    \beta_4 = 0.263, & \text{if}\ \text{OW\_100}_{it}=0 \\
    \beta_4 + \beta_9 = -2.102^{***}, & \text{if}\ \text{OW\_100}_{it}=1
   \end{cases}\\ 
  &\frac{\partial \ln \Pi_{it}}{\partial \tau_{it}} =
   \begin{cases}
    \beta_4 = -0.233, & \text{if}\ \text{OW\_51}_{it}=\text{OW\_100}_{it}=0 \\
    \beta_4 + \beta_8 = -0.318, & \text{if}\ \text{OW\_51}_{it}=1, \text{OW\_100}_{it}=0 \\
    \beta_4 + \beta_9 = -2.101^{***}, & \text{if}\ \text{OW\_51}_{it}=0, \text{OW\_100}_{it}=1
   \end{cases}
 \end{split}
\end{equation}

The coefficient estimates of the interaction term $\tau_{it} \times \text{OW\_100}_{it}$ in regressions (5) and (6) are significant and negative, indicating that wholly-owned subsidiaries shift more income than subsidiaries with an ownership share between 10 and 50.99\%. $\tau_{it} \times \text{OW\_51}_{it}$ enters regression (6) insignificant. The coefficients of the interaction terms give estimates and significances for the differences in income shifting between the three ownership categories of subsidiaries. The marginal effects show that only wholly-owned subsidiaries are engaged in significant income shifting. Both marginal effects for subsidiaries with an ownership share between 10 and 50.99\% and subsidiaries with an ownership share between 51 and 99.99\% are not significant. Regressions (5) and (6) not only support the theory that a higher ownership share is associated with a higher amount of income shifting, but moreover suggest that only wholly-owned subsidiaries shift income. One potential problem with the approach in regression (5) is the following. In Subsection 5.1, it is argued that subsidiaries with less than 51\% ownership can possibly shift income. If this is not justified, the variable $\text{OW\_100}_{it}$ is impractical since it measures the difference in the extent of income shifting between wholly-owned subsidiaries and (at least some) subsidiaries that have, other than assumed, no possibility to shift income (subsidiaries with an ownership share below 51\%). Thus, the variable $\text{OW\_100}_{it}$ is trivial and unsurprisingly shows a significantly negative coefficient. Regression (5) is therefore rerun on a subsample of subsidiaries with an ownership share of at least 51\% (22'084 observations).\footnote{This variation of regression (5) in \cref{fig7} is described verbally only. A results table is not shown.} The coefficient of the interaction term $\tau_{it} \times \text{OW\_100}_{it}$ is equal to $-2.122^*$ and significant on the 10\% confidence level. The marginal effects are $-0.012$ $(-2.134^{***})$ for non wholly-owned (wholly-owned) subsidiaries, of which the latter is significant on the 1\% confidence level. These results are similar to regression (5), confirming that the coefficient of the interaction term in regression (5) is not significant due to trivial reasons.

%---Figure 8---
\begin{figure*}[t]
 \centering \captionsetup{width=0.95\textwidth}
   \includegraphics[scale=0.46]{fig8.pdf}
 \caption[3D Marginal effect of the tax differential in regression (1) in \cref{tab8}]{3D Marginal effect of the tax differential in regression (1) in \cref{tab8}. The marginal effect is plotted for wholly-owned subsidiaries with shifting direction to the parent firm. The plane represents the marginal effect according to the partial derivative of regression (1) in \cref{tab8}. Source: own figure.} \label{fig8}
\end{figure*}

Regression (7) includes an interaction term consisting of the tax differential and the shifting direction, depicted by $Case2_{it}$. $Case2_{it}$ is equal to 1 if the shifting direction is to the parent $(r_{it} > r_{ht}$, $s_{it} < 0)$, and 0 otherwise $(r_{it} < r_{ht}$, $s_{it} > 0)$. The coefficient estimate is negative but insignificant and the marginal effects of the tax differential are given by

%---Eq15---
\begin{equation}\label{eq15}
 \frac{\partial \ln \Pi_{it}}{\partial \tau_{it}} =
   \begin{cases}
    \beta_4 = -1.330, & \text{if}\ Case2_{it}=0 \\
    \beta_4 + \beta_{11} = -1.509^{***}, & \text{if}\ Case2_{it}=1
   \end{cases}
\end{equation}
\vspace{0.1cm}

indicating that income shifting to the parent firm is significant, whereas no income is shifted to the subsidiaries. It should be borne in mind that the coefficient estimate and the marginal effects test different hypotheses, and that comparing marginal effects does not allow to judge on the significance of the difference between them. To gain a deeper insight, regression (7) is rerun separately on the subsample of manufacturing subsidiaries and on the subsample of subsidiaries in the wholesale and retail sector. The marginal effects of the tax differential are $-2.124^{***}$ $(1.591)$ for manufacturing subsidiaries with shifting direction to the parent firm (to the subsidiary), and $-1.026$ $(-7.623^{***})$ for wholesale and retail subsidiaries with shifting direction to the parent firm (to the subsidiary). The coefficient of the direction interaction is significant for both industries, but with differing signs.\footnote{The regression results and additional comments are shown in \cref{tab20} and in \cref{app:D3}.} Hence, manufacturing subsidiaries shift mainly income to the parent in Switzerland and wholesale and retail subsidiaries receive income shifted away from their parent firms in Switzerland. The somewhat unclear result from regression (7) in \cref{tab7} might stem from offsetting industry-dependent shifting behavior.

The analysis of single drivers of income shifting allows to draw the following conclusions. Once a certain size threshold has been reached, income shifting increases with the scale of operations, suggesting that income shifting gives rise to fixed costs. The same argument applies to the amount of intangible assets. Wholly-owned subsidiaries engage in income shifting activities, but non wholly-owned subsidiaries do not. Moreover, the analysis suggests that once below 100\%, the ownership share is irrelevant.\footnote{Interpretation of the results including ownership interactions are only valid under the proviso that the coefficient estimates from regression (5) and (6) in \cref{tab7} are unbiased. \cref{sec:Limitations} elaborates.} On average, Swiss MNEs only shift income to the parent firm in Switzerland, but not to foreign subsidiaries, however, the detailed analysis of the shifting direction interaction suggests that there are differences in income shifting patterns across industries. The conclusions drawn so far are only valid within the respective framework of analysis. How the different interaction terms affect each other is not possible to assess with the results from \cref{tab7}. The following subsection is intended to shed light on how the different drivers of income shifting influence each other.

%---6.3---
\subsection{Extended model results including multiple interactions} \label{sec:Extended model results including multiple interactions}

%---Table 8---
\begin{table*}[t]
\footnotesize
 \begin{center}
 \captionsetup{width=0.75\textwidth}
  \caption{Extended model results including multiple interactions}\label{tab8}
   \begin{tabularx}{0.75\textwidth}{l *{4}{Y}}
    \toprule
     \multicolumn{5}{l}{Subsidiary-fixed effects, panel 2007-2015, dependent variable: ln EBIT, $\Pi_{it}$}\\
     \midrule
     Specification & \multicolumn{2}{c}{Continuous interactions} & \multicolumn{2}{c}{Categorical interactions}\\
     \cmidrule(lr){1-1}
     \cmidrule(lr){2-3}
     \cmidrule(lr){4-5}
     Explanatory variables &(1) &(2) &(3) &(4)\\
     \midrule
     ln GDP per capita, $(A_{it})$ &0.219 &$0.219^*$ &0.238 &$0.228^*$\\
     &(1.240) &(1.651) &(1.349) &(1.714)\\
    ln fixed assets, $(K_{it})$ &$0.136^{***}$ &$0.094^{***}$\\
     &$(5.159)$ &$(5.987)$\\
    ln cost of employees, $(L_{it})$ &$0.458^{***}$ &$0.464^{***}$ &$0.494^{***}$ &$0.489^{***}$\\
    &$(10.768)$ &$(13.605)$ &$(11.525)$ &$(14.461)$\\
    Tax differential, $(\tau_{it})$ &$9.788^{***}$ &$6.318^{***}$ &$0.882$ &$1.104$\\
    &$(3.039)$ &$(2.842)$ &$(0.564)$ &$(1.145)$\\
    Capital interaction, (cont., $\tau_{it}\times K_{it}$) &$-0.612^{***}$ &$-0.407^{***}$\\
    &$(-3.228)$ &$(-3.299)$\\
    ln fixed assets, (cat., $K\_d_{it}$) & & &$0.286^{***}$ &$0.227^{***}$\\
    & & &$(3.533)$ &$(3.828)$\\
    Capital interaction, (cat., $\tau_{it}\times K\_d_{it}$) & & &$-2.273^{***}$ &$-1.765^{***}$\\
    & & &$(-3.336)$ &$(-3.316)$\\
    ln intangible assets, $(I_{it})$ &$-0.004$\\
    &$(-0.390)$\\
    Intangible interaction, (cont., $\tau_{it}\times I_{it}$) &$-0.027$\\
    &$(-0.273)$\\
    ln intangible assets, (cat., $I\_d_{it}$) & & &$0.002$\\
    & & &$(0.043)$ \\
    Intangibles interaction, (cat., $\tau_{it}\times I\_d_{it}$)& & &$-0.251$\\
    & & &$(-0.549)$\\
    $2^{nd}$ ownership interaction, ($\text{OW\_100}_{it}$) &$-3.613^{***}$ &$-2.639^{***}$ &$-3.070^{***}$ &$-2.173^{***}$\\
    &$(-3.349)$ &$(-2.590)$ &$(-2.885)$ &$(-2.202)$\\
    $1^{st}$ ownership interaction, ($\text{OW\_51}_{it}$) &$0.314$ &$-0.536$ &$0.425$ &$-0.316$\\
    &$(0.190)$ &$(-0.345)$ &$(0.273)$ &$(-0.211)$\\
    Shifting direction, $(Case2_{it})$ &$0.051$ & &$0.063$\\
    &$(0.630)$ & &$(0.773)$\\
    Direction interaction, $(\tau_{it}\times Case2_{it})$ &$1.333$ & &$1.980$\\
    &$(0.704)$ & &$(1.048)$\\
    Year dummies &$\sqrt{}$ &$\sqrt{}$ &$\sqrt{}$ &$\sqrt{}$\\
    Industry-year dummies &$\sqrt{}$ &$\sqrt{}$ &$\sqrt{}$ &$\sqrt{}$ \\
    No. of observations &17'897 &26'869 &17'897 &26'869\\
    No. of subsidiaries &3'698 &4'862 &3'698 &4'862\\
    Within $R^2$ &0.066 &0.067 &0.064 &0.065\\
    Overall $F$-test &16.752 &28.072 &15.433 &25.823\\
    \bottomrule
   \end{tabularx}
  \caption*{\footnotesize{\textit{Notes}. Regressions (1) and (2) include continuous capital and intangibles interactions. Regression (1) includes all interactions and (2) includes only significant interactions. Regressions (3) and (4) include categorical capital and intangibles interactions. (3) includes all interactions and (4) includes only significant interactions. Regression (1) is the preferred regression. $^*$, $^{**}$, $^{***}$ denote significance on the 10, 5, 1\% significance level. $t$-statistics are reported in parenthesis and standard errors are clustered at the subsidiary level to control for heteroscedasticity and autocorrelation \parencite[285]{hoechle_robust_2007}. Source: own table.}}
 \end{center}
\end{table*}

\cref{tab8} shows the results of estimating the extended model including multiple interactions. Regressions (1) and (2) include the capital and intangibles interaction continuously specified, and regressions (3) and (4) include all interaction terms categorically specified. Regression (1) and (3) each include all interaction terms, whereas regressions (2) and (4) only include interaction terms that entered significantly in regressions (1) and (3), respectively. Regression (1) shows the results of estimating equation (13) on the sample from \cref{tab1} and constitutes the preferred regression. The coefficient of the tax differential is significantly positive while the coefficient of the capital interaction is significantly negative. Hence, the theory of fixed costs associated with income shifting is supported. The coefficient of the intangibles interaction is not significant anymore and the significance of the coefficients of the two ownership interaction terms remains unchanged. The coefficient of the direction interaction remains insignificant. Marginal effects for this extended model can be calculated using \cref{eqeleven}. Since this model allows to calculate numerous marginal effects for subsidiaries with different characteristics, it is mainly relied on graphics to interpret the shifting behavior of Swiss MNEs. \cref{fig8} shows the marginal effect of the tax differential for wholly-owned subsidiaries with shifting direction to the parent. Note that the variables $\text{OW\_51}_{it}$, $\text{OW\_100}_{it}$ and $Case2_{it}$ shift the plane along the vertical axis, depending on the value they take. For example, switching $\text{OW\_100}_{it}$ from 0 to 1 lowers the plane by 3.613 units ceteris paribus, which is equal to the coefficient estimate of $\text{OW\_100}_{it}$. Thus, the marginal effect is in absolute terms by 3.613 larger for wholly-owned subsidiaries than it is for subsidiaries with an ownership share between 10 and 50.99\%. \cref{fig8} shows that intangible assets are less important in explaining differences in income shifting patterns than is the scale of operations. The marginal effect barely varies with the amount of intangibles, possibly because the informational value of intangibles is nested within fixed assets. This finding contradicts several recent studies examining the effect of intangibles on income shifting.\footnote{Additional comments on the implications of the intangibles interaction in \cref{tab8} are deferred to \cref{sec:Implications for future research}} Among the papers reviewed in \cref{sec:Literature review}, only \textcite[436-438]{beer_profit_2015} take into account that a size effect potentially affects the extent of income shifting. When including an interaction of the tax differential and the logarithm of the MNE's total assets, their intangibles interaction remains significant. However, the intangible assets are measured at the subsidiary level while the size is measured at the MNE level, thus the authors lack consistency of measurement across the variables.

%---Figure 9---
\begin{figure*}[t]
 \centering \captionsetup{width=0.95\textwidth}
   \includegraphics[scale=1]{fig9.pdf}
 \caption[2D Marginal effect of the tax differential in regression (1) in \cref{tab8}]{2D Marginal effect of the tax differential in regression (1) in \cref{tab8}. ln intangible fixed assets are fixed at 11. The solid line shows the marginal effect according to the partial derivative of regression (1) from \cref{tab8}, and the shaded area represents the 90\% confidence interval. The grey bar indicates the range of the middle 90\% of the distribution of ln fixed assets (observations between the $5^{th}$ and $95^{th}$ percentile). Source: own figure, based on \textcite[661]{berry_improving_2012}.} \label{fig9}
\end{figure*}

%---Figure 10---
\begin{figure*}[!]
 \centering \captionsetup{width=0.95\textwidth}
   \includegraphics[scale=1]{fig10}
 \caption[Marginal effect of the tax differential in regression (3) in \cref{tab8}]{Marginal effect of the tax differential in regression (3) in \cref{tab8}. The symbols $\bullet$, $\blacktriangle$, $\blacksquare$ and $+$ represent the marginal effect according to the partial derivative of regression (3) in Table 8, and the grey error bars represent 90\% confidence intervals. Percentages indicate the approximate fraction of the sample belonging to each category. Source: own figure.} \label{fig10}
\end{figure*}

\cref{fig9} highlights how ownership and the shifting direction affect the marginal effect of the tax differential in regression (1). The top and bottom row show the marginal effect for subsidiaries with $Case2_{it} = 1$, respectively with $Case2_{it} = 0$. The intangible asset endowment is fixed at ln intangible fixed assets = 11 (roughly the sample mean) for all effects. Significant income shifting is present whenever the upper bound of the CI is below the zero-line. The marginal effects are practically relevant over the range indicated by the grey bar \parencite[661]{berry_improving_2012}. Both conditions are met only for wholly-owned subsidiaries with sufficiently large ln fixed assets (the top and bottom left graphs). \cref{fig9} suggests that the ownership share explains more variability in income shifting patterns than the direction of income shifting (the difference between semi-elasticities is larger across the columns of the graphs than it is across the rows). This is not surprising given the coefficient estimates and their significance in regression (1) in \cref{tab8}. The results from regression (1) in \cref{tab8} support the theories that the scale of operations and the ownership share influence the extent of income shifting, but reject that intangible assets and the direction of income shifting affect the extent of income shifting significantly. Regression (2) in \cref{tab8} excludes the insignificant interactions and associated standalone terms. Doing so changes the magnitude of the remaining interaction terms slightly, but does not affect signs and significances. In regression (2), the marginal effects of the tax differential for subsidiaries with mean ln fixed assets are $-1.995^{***}$ for wholly-owned subsidiaries, 0.108 for subsidiaries with an ownership share between 51 and 99.99\% and 0.644 for subsidiaries with an ownership share between 10 and 50.99\%.

%---Table 9---
\begin{table*}[t]
\footnotesize
 \begin{center}
  \caption{Summary of semi-elasticities from \cref{sec:Basic model results,sec:Extended model results including single interactions,sec:Extended model results including multiple interactions}}\label{tab9}
   \begin{tabularx}{\textwidth}{l l *{7}{Y}}
    \toprule
    Number &Subsidiary characteristics &$\tau_{it}$ &$K_{it}$/$K\_d_{it}$ &$I_{it}$ &$\text{OW\_51}_{it}$ &$\text{OW\_100}_{it}$ &$Case2_{it}$ &Estimate\\
    \midrule
    \multicolumn{9}{l}{\textit{Panel A}: Basic model, \cref{tab6}}\\
    \midrule
    (A.1) &Regression (2), benchmark &- &- &- &- &- &- &$-1.458^{***}$\\
    (A.2) &Regression (4) & $25^{th}$ perc. &- &- &- &- &- &$-1.481^{***}$\\
    (A.3) &Regression (4) &Mean &- &- &- &- &- &$-1.444^{***}$\\
    (A.4) &Regression (4) &$75^{th}$ perc. &- &- &- &- &- &$-1.407^{*}$\\
    \midrule
    \multicolumn{9}{l}{\textit{Panel B}: Extended model including single drivers of income shifting, \cref{tab7}}\\
    \midrule
    (B.1) &Regression (1) &- &$25^{th}$ perc. &- &- &- &- &$-0.525$\\
    (B.2) &Regression (1) &- &Mean &- &- &- &- &$-1.187^{***}$\\
    (B.3) &Regression (1) &- &$75^{th}$ perc. &- &- &- &- &$-1.834^{***}$\\
    (B.4) &Regression (3) &- &- &$25^{th}$ perc. &- &- &- &$-0.477$\\
    (B.5) &Regression (3) &- &- &Mean &- &- &- &$-0.808$\\
    (B.6) &Regression (3) &- &- &$75^{th}$ perc. &- &- &- &$-1.127^{**}$\\
    (B.7) &Regression (6) &- &- &- &0 &0 &- &$-0.233$\\
    (B.8) &Regression (6) &- &- &- &1 &0 &- &$-0.318$\\
    (B.9) &Regression (6) &- &- &- &0 &1 &- &$-2.101^{***}$\\
    (B.10) &Regression (7) &- &- &- &- &- &0 &$-1.330$\\
    (B.11) &Regression (7) &- &- &- &- &- &1 &$-1.509^{***}$\\
    \midrule
    \multicolumn{9}{l}{\textit{Panel C}: Extended model including multiple drivers of income shifting, \cref{tab8}}\\
    \midrule
    (C.1) &Regression (1), preferred &- &Mean &Mean &0 &1 &1 &$-1.322^*$\\
    (C.2) &Regression (2) &- &Mean &- &0 &0 &- &$0.645$\\
    (C.3) &Regression (2) &- &Mean &- &1 &0 &- &$0.108$\\
    (C.4) &Regression (2) &- &Mean &- &0 &1 &- &$-1995^{***}$\\
    (C.5) &Regression (4) &- &1 &- &0 &0 &- &$-0.661$\\
    (C.6) &Regression (4) &- &1 &- &1 &0 &- &$-0.977$\\
    (C.7) &Regression (4) &- &1 &- &0 &1 &- &$-2.834^{***}$\\
    (C.8) &Regression (4) &- &0 &- &0 &0 &- &$1.104$\\
    (C.9) &Regression (4) &- &0 &- &1 &0 &- &$-0.788$\\
    (C.10) &Regression (4) &- &0 &- &0 &1 &- &$-1.069^*$\\
    \midrule
    \multicolumn{9}{l}{\textit{Panel D}: Comparison with prior literature from \cref{sec:Literature review} and consensus estimate, \cref{tab8}}\\
    \midrule
    (A.1) &\multicolumn{7}{l}{\textcite[163]{hines_fiscal_1994}, Table 2, column (2), OLS cross-section 1982} &$-2.250^{***}$\\
    (A.1) &\multicolumn{7}{l}{\textcite[22-23]{heckemeyer_multinationals_2013}, consensus estimate} &$-0.820^a$\\
    (B.2) &\multicolumn{7}{l}{\textcite[1177]{huizinga_international_2008}, Table 6, column (1), OLS cross-section 1999} &$-1.766$\\
    (B.5) &\multicolumn{7}{l}{\textcite[435]{beer_profit_2015}, Table 2, column (2), mean intangibles, FE panel 2003-2011} &$-0.980^{***}$\\
    (B.9) &\multicolumn{7}{l}{\textcite[26]{dischinger_profit_2008}, Table 6, column (5), wholly-owned subsidiaries, OLS cross-section 2004} &$-1.551^a$\\
    (B.11) &\multicolumn{7}{l}{\textcite[258]{dischinger_role_2014}, Table 3, column (2), shifting to parent, FE panel 1995-2007} &$-1.148$\\
    \bottomrule
   \end{tabularx}
  \caption*{\footnotesize{\textit{Notes}. The semi-elasticities are calculated according to the partial derivative of the regression outputs in \cref{tab6,tab7,tab8}, respectively the mentioned papers. $^a$The significance can not be assessed. Sample statistics can be found in \cref{tab2}. $^*$, $^{**}$, $^{***}$ denote significance on the 10, 5, 1\% significance level. Source: own table.}}
 \end{center}
\end{table*}


Regressions (3) and (4) replicate the analysis from regressions (1) and (2) using categorical instead of continuous specifications for the interaction terms, analogous to the analysis in \cref{tab7}. All previous conclusions drawn remain valid using categorical variables for the capital and intangibles interaction. Using categorical specifications simplifies interpretation and visualization of marginal effects, at the costs of a reduced level of detail. The marginal effects from regression (3) are shown in \cref{fig10}. The marginal effects are significant whenever the error bar does not cross the zero-line. This is the case for wholly-owned subsidiaries with above mean ln fixed assets, regardless of the value of the other variables. Additionally, subsidiaries with below mean ln fixed assets, but above mean ln intangible fixed assets and shifting direction to the parent show borderline significance. \cref{fig10} confirms the insight from \cref{fig8,fig9}, that the ownership share and the scale of operations explain most variation in income shifting behavior of Swiss MNEs. Especially the amount of intangible assets is largely irrelevant for explaining differences in income shifting patterns (the difference between the marginal effects depicted by $\bullet$ and $\blacksquare$ is minute). Further, these conclusions remain valid after excluding the insignificant interaction terms in regression (4).

The main results are equal across all regressions in \cref{tab8}. Particularly, using continuous and categorical specifications for the interaction terms does not lead to differing conclusions. The scale of operations and the ownership share play a dominant role in explaining differences in income shifting behavior of Swiss MNEs.\footnote{The same comments as in footnote 31 apply. Interpreting the ownership interactions is only valid given that the estimates in \cref{tab8} are not biased because of the exclusion of the ownership variables as standalone terms. See \cref{sec:Limitations} for additional comments.} Specifically, the results from the preferred regression (1) indicate that the marginal effect changes by $-0.612$ when ln fixed assets are increased by 1, and that the marginal effect is by $-3.613$ larger (in absolute terms) for wholly-owned subsidiaries compared to subsidiaries with an ownership share between 10 and 50.99\%. The corresponding semi-elasticity from regression (1) for wholly-owned subsidiaries with mean ln fixed and intangible fixed assets and shifting direction to the parent is $-1.322^*$. Evaluating the marginal effect from regression (2) at the sample mean of ln fixed assets yields $-1.995^{***}$ for wholly-owned subsidiaries. All regressions support the idea of fixed costs and suggest that the amount of intangible assets is of minor importance once the scale of operations is considered. Almost only wholly-owned subsidiaries shift income. The impact of the shifting direction is modest as suggested by \cref{fig9,fig10}, however, the result from \cref{sec:Extended model results including single interactions} concerning the shifting direction and industry-specific shifting behavior should be kept in mind.

\subsection{Summary of marginal effects} \label{sec:Summary of marginal effects}


%---Table 10---
\begin{table*}[t]
\footnotesize
 \begin{center}
 \captionsetup{width=0.95\textwidth}
  \caption{Robustness tests for the basic model}\label{tab10}
   \begin{tabularx}{0.95\textwidth}{l *{5}{Y}}
    \toprule
     \multicolumn{6}{l}{Subsidiary-fixed effects, panel 2007-2015, dependent variable: ln EBIT, $\Pi_{it}$, except (1): P/L before tax}\\
     \midrule
     Variation & P/L before tax &Prod. inputs &Industries &Majority &Tax Rate\\
     \cmidrule(lr){1-1}
     \cmidrule(lr){2-6}
     Explanatory variables &(1) &(2) &(3) &(4) &(5)\\
     \midrule
     ln GDP per capita, $(A_{it})$ &$0.279^*$ &$0.596^{***}$ &$0.250^{**}$ &$0.360^{**}$ &$0.235^*$\\
     &$(1.907)$ &$(4.025)$ &$(1.981)$ &$(2.401)$ &$(1.762)$\\
     ln fixed assets, $(K_{it})$ &$0.067^{***}$ & &$0.067^{***}$ &$0.061^{***}$ &$0.063^{***}$\\
     &$(4.863)$ & &$(5.747)$ &$(4.663)$ &$(5.096)$\\
     ln cost of employees, $(L_{it})$ &$0.465^{***}$ & &$0.419^{***}$ &$0.456^{***}$ &$0.465^{***}$\\
     &$(12.876)$ & &$(14.198)$ &$(11.840)$ &$(13.553)$\\
     ln tangible fixed assets, $(TK_{it})$ & &$0.097^{***}$\\
     & &$(5.912)$\\
     ln number of employees, $(L\_N_{it})$ & &$0.406$\\
     & &$(11.146)$\\
    Tax differential, $(\tau_{it})$ &$-1.267^{***}$ &$-1.782^{***}$ &$-1.263^{***}$ &$-1.797^{***}$\\
     &$(-2.612)$ &$(-3.778)$ &$(-3.132)$ &$(-3.626)$\\
     Local tax rate, $(r_{it})$ & & & & &$-1.458^{***}$\\
     & & & & &$(-3.390)$\\
     Year dummies &$\sqrt{}$ &$\sqrt{}$ &$\sqrt{}$ &$\sqrt{}$ &$\sqrt{}$ \\
     Industry-year dummies &$\sqrt{}$ &$\sqrt{}$ &$\sqrt{}$ &$\sqrt{}$ &$\sqrt{}$ \\
     No. of observations &25'919 &22'188 &31'164 &22'084 &26'869\\
     No. of subsidiaries &4'813 &4'327 &5'749 &4'044 &40862\\
     Within $R^2$ &0.057 &0.057 &0.063 &0.066 &0.066\\
     Overall $F$-test &29.639 &29.050 &10.822 &25.714 &31.386\\
    \bottomrule
   \end{tabularx}
  \caption*{\footnotesize{\textit{Notes}. All regressions are based on regression (2) from \cref{tab6} with the following modifications. Regression (1) uses P/L before tax as the dependent variable. Regression (2) uses different production inputs. Regression (3) expands the sample to NACE industries A-I (see \cref{app:B1}). Regression (4) includes only majority-owned subsidiaries. Regression (5) uses the local tax rate instead of the tax differential. $^*$, $^{**}$, $^{***}$ denote significance on the 10, 5, 1\% significance level. $t$-statistics are reported in parenthesis and standard errors are clustered at the subsidiary level to control for heteroscedasticity and autocorrelation \parencite[285]{hoechle_robust_2007}. Source: own table.}}
 \end{center}
\end{table*}

\cref{tab9} presents an overview of semi-elasticities from the different models presented throughout \cref{tab6,tab7,tab8}. Panel A shows the semi-elasticities estimated using different variations of the basic model. Panels B and C show the semi-elasticities estimated using the extended model including single and multiple interactions, and Panel D shows semi-elasticities from related studies presented in the literature review. The benchmark estimate A.1 is smaller than the comparison from \textcite{hines_fiscal_1994}, which is not surprising since various authors mention that using affiliate-level data rather than aggregated data results in lower estimates of the semi-elasticity \parencites[15, 18]{heckemeyer_multinationals_2013}[431]{dharmapala_what_2014}. Other than in \textcite[163]{hines_fiscal_1994}, the quadratic tax term is economically irrelevant here (the estimates A.2-A.4 are similar). The semi-elasticity estimated by \textcite{huizinga_international_2008} is larger than the estimate B.2, however, the tax incentive variable they use is a product of two terms, and hence the difference in the estimates could stem from different model specifications. \textcite[18]{heckemeyer_multinationals_2013} find that the semi-elasticities decrease over time, possibly as a result of the introduction of specific tax law deterring income shifting. Accordingly, the time difference in the datasets could be responsible for the lower estimate. \textcite{beer_profit_2015} find a slightly larger semi-elasticity than the estimate B.5, but their approach raises concerns about variable measurement.\footnote{The concerns about the intangibles interaction in this thesis are discussed in \cref{sec:Implications for future research}.} \textcite{dischinger_profit_2008}finds a lower semi-elasticity for wholly-owned firms than the estimate B.9. Different estimation methods might give rise to this difference. \textcite{dischinger_role_2014} present a smaller semi-elasticity than B.11 using a sample of European MNEs. Their approach is largely identical to the one applied here, hence the larger estimate probably indicates a larger extent of income shifting among Swiss MNEs. The semi-elasticities from \cref{tab9}, specifically the benchmark estimate of $-1.458^{***}$ and the preferred estimate of $-1.322^*$ are large compared to the discussion in \textcite[431-432]{dharmapala_what_2014} and the consensus estimate of $-0.82$ provided by \textcite[22-23]{heckemeyer_multinationals_2013}, possibly because Swiss MNEs face more tax saving opportunities than otherwise similar European MNEs, and thus shift more income.

\subsection{Robustness} \label{sec:Robustness}
The results presented in \cref{sec:Basic model results,sec:Extended model results including single interactions,sec:Extended model results including multiple interactions} are verified in a series of robustness tests. \cref{tab10} provides robustness tests for the basic model, using the benchmark regression (2) from \cref{tab6} as a reference point. \cref{tab11} provides robustness tests for the extended model including single interaction terms, and \cref{tab12} presents robustness tests for the extended model including multiple interaction terms. In regression (1) of \cref{tab10} the EBIT is replaced by P/L before tax as the dependent variable. The coefficient of the tax differential is lower than in regression (1) of \cref{tab6}, which is against expectations based on \textcite[10]{heckemeyer_multinationals_2013}, who state that the EBIT is affected by income shifting through transfer pricing and royalty payments but not financial shifting techniques. Marques and Pinho (2016, 720?21) are confronted with a similar case but do not provide an explanation.\footnote{Compare columns (1) and (4) as well as column (3) and (6) in Table 9 on page 720.}

%---Table 11---
\begin{table*}[t]
\footnotesize
 \begin{center}
 \captionsetup{width=0.85\textwidth}
  \caption{Robustness tests for the extended model including single interactions}\label{tab11}
   \begin{tabularx}{0.85\textwidth}{l *{4}{Y}}
    \toprule
     \multicolumn{5}{l}{Subsidiary-fixed effects, panel 2007-2015, dependent variable: ln EBIT, $\Pi_{it}$}\\
     \midrule
     Income shifting driver &Capital (cont.) &Intang. (cont.) &Ownership &Direction\\
     \cmidrule(lr){1-1}
     \cmidrule(lr){2-5}
     Explanatory variables &(1) &(2) &(3) &(4)\\
     \midrule
     ln GDP per capita, $(A_{it})$ &$0.202$ &$0.241$ &$0.219$ &$0.207$\\
     &$(1.506)$ &$(1.417)$ &$(1.639)$ &$(1.506)$\\
     ln fixed assets, $(K_{it})$ &$0.094^{***}$ &$0.086^{***}$ &$0.066^{***}$ &$0.066^{***}$\\
     &$(6.069)$ &$(4.331)$ &$(5.276)$ &$(5.281)$\\
     ln cost of employees, $(L_{it})$ &$0.466^{***}$ &$0.454^{***}$ &$0.466^{***}$ &$0.466^{***}$\\
     &$(13.511)$ &$(10.630)$ &$(13.453)$ &$(13.454)$\\
     Leverage, ln debt over ln total assets, $(LEV_{it})$ &$-0.257^{***}$ &$-0.365^{***}$ &$-0.253^{***}$ &$-0.254^{***}$\\
     &$(-8.350)$ &$(-9.378)$ &$(-8.224)$ &$(-8.251)$\\
     GDP growth, $(GDP\_G_{it})$ &$0.019^{***}$ &$0.023^{***}$ &$0.018^{***}$ &$0.018^{***}$\\
     &$(4.707)$ &$(4.689)$ &$(4.578)$ &$(4.634)$\\
     Tax differential, $(\tau_{it})$ &$4.418^{**}$ &$1.475$ &$-0.076$ &$-0.909$\\
     &$(2.382)$ &$(1.283)$ &$(-0.095)$ &$(-0.687)$\\
     Capital interaction, (cont., $\tau_{it}\times K_{it}$) &$-0.377^{***}$\\
     &$(-3.098)$\\
     ln intangible assets, $(I_{it})$ & &$0.012$\\
     & &$(1.246)$\\
     Intangible interaction, (cont., $\tau_{it}\times I_{it}$) & &$-0.177$\\
     & &$(-2.050)$\\
     $2^{nd}$ ownership interaction, ($\text{OW\_100}_{it}$) & & &$-1.583^*$\\
     & & &$(-1.670)$\\
     $1^{st}$ ownership interaction, ($\text{OW\_51}_{it}$) & & &$-0.329$\\
     & & &$(-0.225)$\\
     Shifting direction, $(Case2_{it})$ & & & &$0.006$\\
     & & & &$(0.100)$\\
     Direction interaction, $(\tau_{it}\times Case2_{it})$ & & & &$-0.301$\\
     & & & &$(-0.203)$\\
    Year dummies &$\sqrt{}$ &$\sqrt{}$ &$\sqrt{}$ &$\sqrt{}$\\
    Industry-year dummies &$\sqrt{}$ &$\sqrt{}$ &$\sqrt{}$ &$\sqrt{}$ \\
    No. of observations &26'577 &17'680 &26'577 &26'577\\
    No. of subsidiaries &4'818 &3'659 &4'818 &4'818\\
    Within $R^2$ &0.074 &0.076 &0.074 &0.074\\
    Overall $F$-test &32.160 &22.331 &30.233 &30.171\\
    \bottomrule
   \end{tabularx}
  \caption*{\footnotesize{\textit{Notes}. Regressions (1), (2), (3) and (4) correspond to regressions (1), (3), (6) and (7) from \cref{tab7} and add leverage and GDP growth as additional control variables.. $^*$, $^{**}$, $^{***}$ denote significance on the 10, 5, 1\% significance level. $t$-statistics are reported in parenthesis and standard errors are clustered at the subsidiary level to control for heteroscedasticity and autocorrelation \parencite[285]{hoechle_robust_2007}. Source: own table.}}
 \end{center}
\end{table*}

The result from regression (1) suggests that not all shifting techniques are used to shift income in the same direction, and that the underlying shifting incentive might be more complex than the mere tax differential between countries. The production factors fixed assets and costs of employees are replaced with tangible fixed assets and the number of employees in regression (2), resulting in a larger coefficient estimate of the tax differential of $-1.782$. Expanding the industries to the NACE sectors A-I in regression (3) yields a lower coefficient of $-1.263$. The reduced estimate could be a further indication of industry-specific shifting behavior. Regression (4) considers only subsidiaries with an ownership share of 51\% or more. Unsurprisingly, the coefficient estimate of the tax differential is larger, reflecting that majority-owned subsidiaries engage in income shifting activities to a larger extent than non majority-owned subsidiaries. Regression (5) replaces the tax differential with the local tax rate at the subsidiary's location, showing that with all parent firms facing the same tax rate, it is computationally irrelevant whether the tax differential or the local tax rate of the subsidiary is used. The coefficient estimates and corresponding $t$-statistics are identical to the numbers presented in regression (2) of \cref{tab6}.\footnote{Calculating the tax differential as $\tau_{it}=(r_{it}-r_{ht})$ constitutes a variable transformation which does not affect the coefficient estimates. This is the case because the parent tax rate $r_{ht}$ is the same for all observations for any given year. This argument does not apply to studies using samples with parent firms from multiple countries.} However, the interpretation of a change in the tax incentive variable is slightly different. In case a tax differential is used, a change in the shifting incentive can be attributed to either a change in the tax rate of the subsidiary, the parent firm or both. By using the local tax rate of the subsidiary only, a change in the tax rate of the parent firm is irrelevant for interpretation, thus using the tax rate instead of the tax differential probably reflects reality less accurate. The robustness tests in \cref{tab10} show that the conclusions from the basic model of income shifting remain unchanged for various modifications of the estimation approach.

%---Table 12---
\begin{table*}[t]
\footnotesize
 \begin{center}
 \captionsetup{width=0.85\textwidth}
  \caption{Robustness tests for the extended model including multiple interactions}\label{tab12}
   \begin{tabularx}{0.85\textwidth}{l *{4}{Y}}
    \toprule
     \multicolumn{5}{l}{Subsidiary-fixed effects, panel 2007-2015, dependent variable: ln EBIT, $\Pi_{it}$, except (2): P/L before tax}\\
     \midrule
     Variation &Add. controls &P/L before tax &Prod. inputs &Industries\\
     \cmidrule(lr){1-1}
     \cmidrule(lr){2-5}
     Explanatory variables &(1) &(2) &(3) &(4)\\
     \midrule
     ln GDP per capita, $(A_{it})$ &$0.200$ &$0.263^*$ &$0.595^{***}$ &$0.236^*$\\
     &$(1.501)$ &$(1.812)$ &$(4.053)$ &$(1.873)$\\
     ln fixed assets, $(K_{it})$ &$0.099^{***}$ &$0.101^{***}$ & &$0.098^{***}$\\
     &$(6.276)$ &$(5.958)$ & &$(6.605)$\\
     ln cost of employees, $(L_{it})$ &$0.465^{***}$ &$0.463^{***}$ & &$0.417^{***}$\\
     &$(13.517)$ &$(12.908)$ & &$(14.207)$\\
     Leverage, ln debt over ln total assets, $(LEV_{it})$ &$-0.256^{***}$\\
     &$(-8.326)$\\
     GDP growth, $(GDP\_G_{it})$ &$0.018$\\
     &$(4.624)$\\
     ln tangible fixed assets, $(TK_{it})$ & & &$0.120^{***}$\\
     & & &$(6.011)$\\
     ln number of employees, $(L\_N_{it})$ & & &$0.406^{***}$\\
     & & &$(11.217)$\\
     Tax differential, $(\tau_{it})$ &$6.923^{***}$ &$6.221^{**}$ &$4.384^*$ &$6.156^{**}$\\
     &$(3.119)$ &$(2.537)$ &$(1.727)$ &$(2.885)$\\
     Capital interaction, (cont., $\tau_{it}\times K_{it}$) &$-4.34^{***}$ &$-0.451^{***}$ & &$-0.414^{***}$\\
     &$(3.522)$ &$(-3.245)$ & &$(-3.543)$\\
     $2^{nd}$ ownership interaction, ($\text{OW\_100}_{it}$) &$-2.422^{**}$ &$-1.132$ &$-2.531^*$ &$-2.040^{**}$\\
     &$(-2.407)$ &$(-1.003)$ &$(-2.213)$ &$(-2.092)$\\
     $1^{st}$ ownership interaction, ($\text{OW\_51}_{it}$) &$-0.800$ &$-1.070$ &$0.414$ &$-0.238$\\
     &$(-0.518)$ &$(-0.553)$ &$(0.245)$ &$(-0.178)$\\
     Capital interaction, (cont., $\tau_{it}\times TK_{it}$) & & &$-0.313^{**}$\\
     & & &$(-2.239)$\\
     Year dummies &$\sqrt{}$ &$\sqrt{}$ &$\sqrt{}$ &$\sqrt{}$\\
     Industry-year dummies &$\sqrt{}$ &$\sqrt{}$ &$\sqrt{}$ &$\sqrt{}$ \\
     No. of observations &26'577 &25'919 &22'188 &31'164\\
     No. of subsidiaries &4'818 &4'813 &4'327 &5'749\\
     Within $R^2$ &0.075 &0.057 &0.058 &0.064\\
     Overall $F$-test &29.814 &26.358 &25.770 &10.613\\
    \bottomrule
   \end{tabularx}
  \caption*{\footnotesize{\textit{Notes}. All regressions are based on regression (2) from \cref{tab8} with the following modifications. Regression (1) includes leverage and GDP growth as additional control variables. Regression (2) uses P/L before tax as the dependent variable. Regression (3) uses different production inputs. Regression (4) expands the sample to NACE industries A-I (see \cref{app:B1}). $^*$, $^{**}$, $^{***}$ denote significance on the 10, 5, 1\% significance level. $t$-statistics are reported in parenthesis and standard errors are clustered at the subsidiary level to control for heteroscedasticity and autocorrelation \parencite[285]{hoechle_robust_2007}. Source: own table.}}
 \end{center}
\end{table*}

\cref{tab11} takes up the regressions (1), (3), (6) and (7) from \cref{tab7} and adds frequently used control variables. The additional variables are the leverage of the subsidiary, measured as ln debt over ln total assets, and GDP growth. The leverage enters significantly negative in all regressions, suggesting that a higher leverage is associated with a lower EBIT. This is not intuitive, since the EBIT is independent of interest payments. \textcite[1174]{huizinga_international_2008} find the same result and argue that leverage could make profitable investments more difficult to finance. They also find that leverage has a more pronounced influence when the P/L before tax is used instead of the EBIT, which is a more intuitive result \parencite[1174]{huizinga_international_2008}. The GDP growth enters significantly positive in all regressions, indicating that higher GDP growth is associated with a higher EBIT. The robustness tests in \cref{tab11} do not lead to changes of signs and significances of the coefficient estimates of the tax differential and interaction terms. Hence, all conclusions drawn from the extended model using single interactions in \cref{sec:Extended model results including single interactions} remain valid after controlling for leverage and GDP growth. However, including the additional controls reduces the marginal effect of the tax differential. The semi-elasticity in regression (1) of \cref{tab11} is $-0.838^*$ for subsidiaries with mean ln fixed assets, whereas the same marginal effect is $-1.187^{***}$ in regression (1) of \cref{tab7}. This is in line with \textcite[18]{heckemeyer_multinationals_2013}, who find in a meta-study of 25 income shifting papers, that controlling for leverage reduces the impact of the tax differential.

%---Figure 11---
\begin{figure*}[t]
 \centering \captionsetup{width=0.95\textwidth}
   \includegraphics[scale=1]{fig11.pdf}
 \caption[Income shifting patterns over time]{Income shifting patterns over time. The left and right graphs show the marginal effect of the tax differential using a continuous and a categorical specification. The solid line shows the marginal effect according to the partial derivatives of regressions (1) and (2) in \cref{tab13}, and the shaded area represents the 90\% confidence interval. Percentages indicate the approximate fraction of observations that fall into each year. Source: own figure.}\label{fig11}
\end{figure*}

\cref{tab12} provides results from robustness tests of the extended model including multiple interactions. All four regressions are based on regression (2) in \cref{tab8}. The same variations as in \cref{tab10} are estimated. Regression (1) includes the additional control variables leverage and GDP growth, regression (2) uses the P/L before tax as the dependent variable instead of the EBIT, in regression (3) the fixed assets and the costs of employees are replaced by tangible fixed assets and the number of employees, and finally regression (4) expands the sample to the NACE main sectors A-I. Including the additional control variables does not alter the significance of the tax differential or the interaction terms, but renders the coefficient of the GDP per capita insignificant. This might be the result of collinearity between GDP per capita and GDP growth, in which case it would be sufficient to include one of the two variables. The marginal effect for wholly-owned subsidiaries at the sample mean of ln fixed assets is $-1.550^{***}$. Replacing the EBIT with P/L before tax in regression (2) renders the coefficient of $\text{OW\_100}_{it}$ insignificant. The marginal effect for subsidiaries with the characteristics described above is $-1.199^{**}$, and both effects for non wholly-owned subsidiaries are not significant. This pattern of results is also observed in regressions (4) and (7) of \cref{tab7}, and the comments in \cref{app:D2} apply equivalently. Replacing the production function inputs in regression (3) does not lead to changes in signs or significances of the interaction terms. The marginal effect for subsidiaries with the characteristics above is $-2.366^{***}$. Expanding the industries in regression (4) results in a marginal effect of $-1.656^{***}$ for wholly-owned subsidiaries with mean ln fixed assets. As proposed earlier in this subsection and in \cref{sec:Extended model results including single interactions}, the lower marginal effect in regression (4) could indicate that the income shifting behavior of Swiss MNEs depends on the industry-affiliation. These four semi-elasticities correspond to the marginal effect in regression (2) from \cref{tab8}, which is equal to $-1.995^{***}$ for subsidiaries with the same characteristics. The extended model using multiple income shifting drivers is robust to the estimation variations applied in \cref{tab12}.

\cref{sec:Discussion} contains a discussion about how the income shifting behavior of Swiss MNEs developed over time and which locations play important roles within income shifting strategies of Swiss MNEs. As mentioned in the introduction, recent research is often concerned with the effectiveness of various tax legislations \parencite{beuselinck_cross-jurisdictional_2015}. \textcite[1]{lohse_impact_2012} mention that various countries have recently introduced TP legislation aimed at reducing income shifting among MNEs. Against this background, it seems fruitful to address the above mentioned issues for the following reasons. First, examining how income shifting patterns developed over time could provide insights on the effectiveness of tax legislations, and secondly, examining locations involved in income shifting strategies might provide guidance in designing new tax legislation aimed at deterring the extent of income shifting. For simplicity, the discussion is based on the basic model in the benchmark regression (2) of \cref{tab6}. Additionally, \cref{sec:Implications for future research} summarizes implications for future research.

%---7---
\section{Discussion} \label{sec:Discussion}
\subsection{Did income shifting decrease over time?} \label{sec:Did income shifting decrease over time?}

%---Table 13---
\begin{table}[!]
\footnotesize
 \begin{center}
  \caption{Income shifting patterns over time}\label{tab13}
   \begin{tabularx}{\linewidth}{l *{2}{Y}}
   \toprule
   \multicolumn{3}{l}{Subsidiary-fixed effects, panel 2007-2015, dep. var.: ln EBIT, $\Pi_{it}$}\\
   \midrule
   Time trend &Continuous & Categorical\\
   \cmidrule(lr){1-1}
   \cmidrule(lr){2-2}
   \cmidrule(lr){3-3}
   Explanatory variables &(1) &(2)\\
   \midrule
   Year, (cont., $T\_CY_{it}$) &$-0.033^{***}$ \\
   &$(-4.035)$\\
   ln GDP per capita, ($A_{it}$) &$0.519^{***}$ &$0.320^{**}$\\
   &$(3.606)$ &$(2.178)$\\
   ln fixed assets, ($K_{it}$) &$0.064^{***}$ &$0.062^{***}$ \\
   &$(5.218)$ &$(5.071)$\\
   ln cost of employees, ($L_{it}$) &$0.486^{***}$ &$0.467^{***}$\\
   &$(14.255)$ &$(13.564)$\\
   Tax differential, ($\tau_{it}$) &$-173.913^a$ &$-1.524^{***}$\\
   &$(1.530)$ &$(3.330)$\\
   Time interaction &$0.086$&\\
   (cont., $\tau_{it} \times$ Year) &$(1.520)$\\
   $\tau_{it} \times 2008$ & &$-1.025^{***}$\\
   & &$(-2.858)$\\
   $\tau_{it} \times 2009$ & &$-0.544$\\
   & &$(-1.318)$\\
   $\tau_{it} \times 2010$ & &$0.287$\\
   & &$(0.675)$\\
   $\tau_{it} \times 2011$ & &$0.020$\\
   & &$(0.049)$\\
   $\tau_{it} \times 2012$ & &$0.481$\\
   & &$(1.098)$\\
   $\tau_{it} \times 2013$ & &$0.143$\\
   & &$(0.313)$\\
   $\tau_{it} \times 2014$ & &$0.103$\\
   & &$(0.225)$\\
   $\tau_{it} \times 2015$ & &$0.184$\\
   & &$(0.383)$\\
   Year dummies & &$\sqrt{}$\\
   Industry-year dummies &$\sqrt{}$ &$\sqrt{}$\\
   No. of observations &26'869 &26'869\\
   No. of subsidiaries &4'862 &4'862\\
   Within $R^2$ &0.057 &0.067\\
   Overall $F$-test &56.095 &23.121\\
   \bottomrule
   \end{tabularx} 
  \caption*{\footnotesize{\textit{Notes}. Regression (1) includes a time interaction with a continuous time variable. Regression (2) includes time interactions with yearly categorical variables. $^*$, $^{**}$, $^{***}$ denote significance on the 10, 5, 1\% significance level. $t$-statistics are reported in parenthesis and standard errors are clustered at the subsidiary level to control for heteroscedasticity and autocorrelation \parencite[285]{hoechle_robust_2007}. $^a$The coefficient of the tax differential is much larger than in previous specification. This is the result of including the the time trend as calendar years instead of using integers 1-9 as done in \textcite{lohse_impact_2012}. Since this a linear variable transformation, the marginal effect is unaffected. Source: own table.}}
 \end{center}
\end{table}

%---Figure 12---
\begin{figure*}[t]
 \centering \captionsetup{width=0.95\textwidth}
     \includegraphics[scale=1]{fig12.pdf}
 \caption[Income shifting across geographic regions]{income shifting across geographic regions. Marginal effects of the tax differential across different regions in the world and Europe. Black dots represent the marginal effects according to the partial derivative of regression (1) and (2) in \cref{tab14}, and grey error bars represent 90\% confidence intervals. Percentages indicate the approximate fraction of observations that falls into each region. Source: own figure.}\label{fig12}
\end{figure*}

\textcite[10, 23]{lohse_impact_2012} use a linear time trend to assess changes in income shifting behavior and find that income shifting decreased over their sample period, suggesting that TP regulations indeed reduce the extent of income shifting among European MNEs. However, since negative and positive changes offset each other, modeling a linear time trend provides only limited insights into changes in income shifting patterns. Therefore, an alternative approach using interaction terms consisting of the tax differential and yearly categorical variables is additionally applied. The results of both approaches are presented in \cref{tab13} and the corresponding marginal effects are plotted in \cref{fig11}. Regression (1) includes a linear time trend using calendar years as a continuous variable. The negative coefficient of the time trend implies that the EBIT decreased significantly over the sample period. The marginal effect of the tax differential in the left graph of \cref{fig11} suggests that income shifting decreased slightly over time, moreover, it indicates that Swiss MNEs did not shift income during the last sample years 2014 and 2015. Other than in \textcite[10]{lohse_impact_2012}, the coefficient estimate of the interaction term in regression (1) is not significant. Regression (2) provides a more detailed insight into how income shifting among Swiss MNEs has changed over time. In regression (2), the coefficient of the tax differential depicts the marginal effect for all observations in the year 2007. An isolated interpretation is therefore useful. During the year 2007, an increase in the tax differential by 1 percentage point is associated with a decrease in the EBIT of 1.524\%. The right graph in \cref{fig11} shows how the semi-elasticity developed over time. It dropped significantly to $-2.549^{***}$ in 2008, but is not significantly different from the level in 2007 in later years. Further, the right graph in \cref{fig11} does not suggest that income shifting disappeared in the latest sample years. A possible explanation for the large extent of income shifting in 2008 is the following. Many companies experienced losses during the financial crisis. Income shifted to a loss-making affiliate is taxed at 0\% up to the extent of the losses of that affiliate, and hence the tax incentive to shift income is larger than the tax differential because the ETR of at least part of the shifted income is smaller than the statutory CITR. This argument further implies that loss carry-forwards change the shifting incentive \parencite[9-10]{overesch_effects_2009} (\cref{sec:Limitations} elaborates). To conclude, income shifting among Swiss MNEs did not decrease over time, but rather remained constant in the last sample years.

\subsection{Are income shifting patterns different across the globe?} \label{sec:Are income shifting patterns different across the globe?}

%---Table 14---
\begin{table}[!b]
\footnotesize
 \begin{center}
  \caption{Income shifting across geographic regions}\label{tab14}
   \begin{tabularx}{\linewidth}{l *{2}{Y}}
   \toprule
   \multicolumn{3}{l}{Subsidiary-fixed effects, panel 2007-2015, dep. var.: ln EBIT, $\Pi_{it}$}\\
   \midrule
   Geographic regions &Worldwide &Europe\\
   \cmidrule(lr){1-1}
   \cmidrule(lr){2-2}
   \cmidrule(lr){3-3}
   Explanatory variables &(1) &(2)\\
   \midrule
   ln GDP per capita, ($A_{it}$) &$0.244^*$ &$0.230$\\
   &$(1.794)$ &$(1.381)$\\
   ln fixed assets. ($K_{it}$) &$0.063^{***}$ &$0.061^{***}$\\
   &$(5.086)$ &$(4.915)$\\
   ln cost of employees, ($L_{it}$) &$0.466^{***}$ &$0.465^{***}$\\
   &$(13.553)$ &$(13.235)$\\
   Tax differential, ($\tau_{it}$) &$-0.913$ &$-0.636$\\
   &$(-1.211)$ &$(-0.428)$\\
   $\tau_{it} \times$ Americas$\_d_{it}$ &$-8.624^{***}$\\
   &$(-2.649)$\\
   $\tau_{it} \times$ Asia$\_d_{it}$ &$2.152$\\
   &$(1.524)$\\
   $\tau_{it} \times$ Europe$\_d_{it}$ &$-0.748$\\
   &$(-0.842)$\\
   $\tau_{it} \times$ Oceania$\_d_{it}$ &$-17.028^{***}$\\
   &$(-2.976)$\\
   $\tau_{it} \times$ Northern$\_$Europe$\_d_{it}$ & &$-2.089$\\
   & &$(-1.180)$\\
   $\tau_{it} \times$ Southern$\_$Europe$\_d_{it}$ & &$-0.606$\\
   & &$(-0.328)$\\
   $\tau_{it} \times$ Western$\_$Europe$\_d_{it}$ & &$-0.598$\\
   & &$(-0.364)$\\
   Year dummies &$\sqrt{}$ &$\sqrt{}$\\
   Industry-year dummies &$\sqrt{}$ &$\sqrt{}$\\
   No. of observations &26'869 &25'745\\
   No. of subsidiaries &4'862 &4'607\\
   Within $R^2$ &0.066 &0.067\\
   Overall $F$-test &26.453 &26.949\\
   \bottomrule
   \end{tabularx} 
  \caption*{\footnotesize{\textit{Notes}. Regression (1) includes geographic interactions with four categorical variables. Americas$\_d_{it}$, Asia$\_d_{it}$, Europe$\_d_{it}$ and Oceania$\_d_{it}$, each being equal to 1, if the observation falls into that region and 0 otherwise. Regression (2) includes geographic interactions with three categorical variables Northern$\_$Europe$\_d_{it}$, Southern$\_$Europe$\_d_{it}$, and Western$\_$Europe$\_d_{it}$, each being equal to 1 if the observation falls into that region and 0 otherwise. The United Nations region classification used to assign countries to geographic regions can be found in \cref{app:C2}. $^*$, $^{**}$, $^{***}$ denote significance on the 10, 5, 1\% significance level. $t$-statistics are reported in parenthesis and standard errors are clustered at the subsidiary level to control for heteroscedasticity and autocorrelation \parencite[285]{hoechle_robust_2007}. Source: own table.}}
 \end{center}
\end{table}

\textcite[1173]{huizinga_international_2008} include an interaction term consisting of the weighted tax differential and a categorical variable equal to 1 for firms located in Eastern Europe. Doing so allows the semi-elasticity of the EBIT to vary across geographic regions. This method is adopted in regressions (1) and (2) in \cref{tab14}. Regression (1) includes four categorical variables depicting the world region the subsidiary falls into (Africa, Americas, Asia, Europe, and Oceania). Each categorical variable being equal to 1 if the subsidiary is located in that region and Africa being the reference category. The same approach is used in regression (2) with more detailed regions within Europe (Northern, Southern, Western and Eastern Europe as the reference category). The sample is therefore restricted to European subsidiaries of Swiss MNEs. The results from regression (1) are visualized in \cref{fig12} on the left, suggesting that significant income shifting occurs between the parent firm in Switzerland and subsidiaries located in the Americas, Europe and Oceania. The marginal effects of the tax differential are $-9.537^{***}$ for subsidiaries located in the Americas, $-1.661^{***}$ for European subsidiaries and $-17.941^{***}$ for subsidiaries in Oceania. However, due to the low number of observations in the Americas and Oceania, the estimates are inaccurate (indicated by the wide CIs). Using only observations of European subsidiaries favors the use of smaller geographic regions.

The marginal effect of the tax differential from regression (2) is plotted in the right graph of \cref{fig12}. The semi-elasticities for Northern and Western Europe are $-2.728^{***}$ respectively $-1.234^*$, and both significant. In other words, an increase of 1 percentage point in the tax differential is associated with a decrease in the EBIT of 2.728\% for subsidiaries in northern Europe and 1.234\% for subsidiaries in Western Europe. No income shifting occurs between the Swiss parent firm and subsidiaries located in Eastern or Southern Europe. This is not surprising since the tax rates in Eastern and Southern Europe are lower than in Northern and Western Europe, but very similar to the Swiss CITR (see \cref{fig4} in \cref{sec:Tax rates and tax differentials}).

\subsection{Implications for future research} \label{sec:Implications for future research}
Besides bringing up results to address the thesis objectives in \cref{sec:Objective}, the empirical analysis has brought forward several implications relevant to future research and tax policy. Most important among these is the threshold of fixed assets, that needs to be reached in order for a MNE to benefit from income shifting. \cref{fig7} illustrates this point well. \textcite[445]{dharmapala_what_2014} mentions very generally, that heterogeneity in MNEs' corporate structure, and hence in shifting behavior, can be viewed as consistent with fixed costs associated with income shifting. The results presented in \cref{sec:Extended model results including single interactions,sec:Extended model results including multiple interactions} strongly suggest that Swiss MNEs face fixed costs when shifting income and there is no obvious a priori reason to assume this is not the case for other MNEs. However, the results here provide a mere indication of fixed costs, the size and the cause thereof are to be studied in future research efforts.

An unexpected result is the low impact of intangible assets on income shifting behavior. While \textcite[435]{beer_profit_2015} and \textcite[706]{dischinger_corporate_2011} find that the intangible assets of the subsidiary significantly impact income shifting, the results in \cref{sec:Extended model results including multiple interactions} suggest that intangible assets are unrelated to the extent of income shifting once the capital interaction is included in the model. This is especially surprising considering the large empirical evidence on the relation between tax rates and intangibles, including recent evidence stemming from very different research approaches (for example \textcite[185]{karkinsky_corporate_2012}, who find that MNEs preferably locate patents in low-tax affiliates). Hence, it is unwise to conclude that the results shown here nullify previous research. It is rather stressed to choose a sensible approach when studying how intangible assets and firm size affect income shifting. One such approach is to use the ratio of intangible fixed assets over fixed assets. Since the variable fixed assets used here includes intangible fixed assets, at least part of the significant effect of the capital interaction can be traced back to intangible assets.

The analysis of the shifting direction in \cref{sec:Extended model results including single interactions} suggests, that the income shifting behavior of Swiss MNEs depends on the industry-affiliation. The evidence provided here is of anecdotal nature only and the detailed study of industry-related income shifting behavior is left to future research. Further, the discussion in \cref{sec:Did income shifting decrease over time?} indicates that against expectations, income shifting among Swiss MNEs did not decrease over time. This is surprising since many authors describe a general trend of intensification of tax laws within Europe and as a consequence, report significant reductions in income shifting activities \parencites[for example][23-24]{lohse_increasing_2012}[729]{marques_is_2016}. The income shifting activities of Swiss MNEs would have been expected to decrease since a large fraction of subsidiaries is located within Europe, and thus affected by stricter tax law. This seemingly contradictory finding leads one to assume, that income shifting is most effectively deterred by tightening tax law in the host country of the parent firm. Hence, relevant tax law in Switzerland needs to be adjusted to effectively deter income shifting among Swiss MNEs. This find- ing in conjuncture with industry- and country-specific shifting behavior might provide useful in designing specific tax legislation aimed at preventing income shifting of Swiss MNEs.

Since the number of subsidiaries is low outside of Europe, the CIs in Figure 12 are wide and conclusions based thereon are inherently inaccurate. Therefore, future research on income shifting of Swiss MNEs using the ORBIS database is advised to use a European sample, unless intercontinental data is explicitly needed.

%---8---
\section{Limitations} \label{sec:Limitations}
The data and methodology applied in this thesis come with several deficiencies. The limited access to ownership data has far reaching economic and statistical consequences. Parent firms of subsidiaries can not be identified and hence, affiliates of a MNE cannot be linked to each other to form a corporate group consisting of several affiliates. Thus, calculation of an aver- age tax differential is impossible. Several authors find that MNEs significantly shift income between subsidiaries of the same parent firm \parencites[for example][1174]{huizinga_international_2008}[18-19]{dischinger_corporate_2008}. The thesis at hand missed this potential part of income shifting among Swiss MNEs. Further, parent firms cannot be analyzed as for example in \textcite[260-261]{dischinger_role_2014}. However, this aspect of the ownership issue is outside the influence of the author of this thesis, and by using subsidiaries with a Swiss parent holding a specified ownership share, a satisfactory second-best solution was chosen.

The ownership data from ORBIS is reported for the last available year only (2015 in most cases), and the ownership categorization based on the variables $\text{OW\_51}_{it}$ and $\text{OW\_100}_{it}$ is a snapshot thereof \parencite[430]{dharmapala_what_2014}. By assuming the ownership structure did not change over the sample period, the sample potentially includes subsidiaries with a smaller ownership share than 10\% in earlier years. While \textcite[76]{budd_wages_2005} argue that this measurement error distorts the effect of the tax differential towards 0, other approaches to this problem exist. \textcite[15, 26]{dischinger_profit_2008} uses an OLS cross-section of the last available year to qualitatively verify his results obtained from FE regressions using the whole panel sample. This approach further allows to include the ownership variable as a standalone term and thus avoids possible bias introduced by leaving out terms constituting an interaction term \parencite[66-68]{brambor_understanding_2006}. Another solution is to use more sophisticated statistical methods, such as the estimator proposed by \textcite{hausman_panel_1981}. This approach allows to consistently estimate both time-variant and time-invariant regressors, while still controlling for unobserved heterogeneity among subsidiaries \parencite[325-326]{wooldridge_introductory_2009}. Within this thesis, the estimates of regressions including ownership share interactions should be treated with care since they are possibly biased.

The unequal coverage of different countries and the low coverage of tax havens in the ORBIS database has been mentioned often \parencites[for example][18]{fuest_tax_2010}[906]{clausing_effect_2016}. \textcite[529-530]{desai_demand_2006} find that tax havens are important elements in tax avoidance strategies of US MNEs, and \textcite[906]{clausing_effect_2016} mentions that research using the ORBIS database misses key observations from tax haven countries that drive income shifting behavior.\footnote{However, \textcite[917]{clausing_effect_2016} finds in the same study, that 4 out of 8 (including the top 3) key locations for income shifting of US MNEs in 2012 are industrialized countries in Europe, namely the Netherlands, Ireland, Luxembourg and Switzerland. The coverage in these countries is higher \parencite[7]{kalemli-ozcan_how_2015}, and the criticism is partially mitigated.} While this limitation can be crucial to the validity of the research in these papers, it is presumably of second-order importance for the analysis presented here. Switzerland is defined a tax haven by \textcite[517]{desai_demand_2006}, hence every subsidiary in the sample has the possibility to benefit from tax haven operations (shifting income to the parent firm in Switzerland) as defined by the aforementioned authors. Given the comparably low CITRs in Switzerland, the tax benefits from establishing and shifting income to affiliates in other tax havens is lower for Swiss MNEs than it is for other MNEs facing a higher CITR. Thus, it is assumed that Swiss MNEs are unlikely to have numerous affiliates incorporated in other tax havens.

\textcite[19]{overesch_transfer_2006} finds that firms with loss carry-forwards have a reversed incentive to shift income. Neglecting loss carry-forwards might lead to distorted coefficient estimates. In consequence, it is sensible to exclude subsidiaries with loss carry-forwards, or offer a differing analysis for such firms. \textcite[9-10]{overesch_effects_2009} mentions that firms incorporated in high-tax jurisdictions with loss carry-forwards constitute a suitable control group when studying income shifting. Such firms are faced with a highly positive tax differential, but can offset any profits with former losses. Thus, the shifting incentive is mitigated and applying the basic model from this thesis on a sample of firms with these characteristics should result in an insignificant estimate of the tax differential. Such an analysis could pronounce the indirect evidence of income shifting found in this thesis. Being able to identify a suitable control group further allows to use potentially fruitful approaches, such as difference-in-difference analysis. Alternatively, loss carry-forwards could be used as an explanatory variable, as done for example by \textcite[74]{buettner_internal_2013}.

%---9---
\section{Conclusion} \label{sec:Conclusion}
The main objectives of this thesis were to investigate whether Swiss MNEs shift income or not, and to verify if the income shifting behavior of Swiss MNEs depends on certain firm and country characteristics. The drivers studied in this thesis are the scale of operations, the amount of intangible assets, the ownership share and the direction of shifting. A large body of empirical literature has shown that specific tax law effectively deters income shifting. The absence of such tax law in Switzerland in combination with comparably low statutory tax rates gave rise to the suspicion, that income shifting of Swiss MNEs is large.

The thesis at hand provides indirect evidence of income shifting among Swiss MNEs. The benchmark regression has shown, that a 1 percentage point increase in the tax differential is associated with a decrease of the subsidiary's EBIT by 1.458\%. Moreover, this semi-elasticity is larger than recent results from comparable studies using samples of European firms, possibly because Swiss MNEs face larger income shifting opportunities than otherwise similar European firms. Additionally, this thesis identified the scale of operations and the ownership share as significant determinants of income shifting behavior of Swiss MNEs. The sensitivity of the EBIT to changes in the tax differential increases with the scale of operations and the ownership share. Quantitatively, the preferred regression shows that increasing the ln fixed assets by 1 leads to a decrease (absolute increase) in the semi-elasticity of the EBIT w.r.t. the tax differential of $-0.612$. The difference in semi-elasticities between wholly-owned subsidiaries and subsidiaries with an ownership share between 10 and 50.99\%, is estimated at $-3.613$. Hence, Swiss MNEs shift income using large, wholly-owned subsidiaries. The preferred regression implies that a 1 percentage point increase in the tax differential translates into a 1.322\% decrease in the EBIT of wholly-owned subsidiaries with mean ln fixed assets, mean ln intangible fixed assets and shifting direction to the parent. Intangible assets and the direction of shifting are less important drivers of income shifting of Swiss MNEs.

It was further shown, that income shifting activities of Swiss MNEs are probably different across industries. Against expectations, the extent of income shifting of Swiss MNEs did not decrease over time, but remained stable in recent years. Swiss MNEs shift income mainly from high-tax countries in Northern and Western Europe to Switzerland. It is again stressed, that all results should be considered with respect to the mentioned limitations.
\clearpage

\printbibliography
\clearpage

\appendix
%---A---
\section{Appendix \Cref{sec:Literature review}}\label{app:A}
\cref{tab15} shows a structured overview on the studies included in the literature review in \cref{sec:Literature review}.

\begin{sidewaystable}
 \footnotesize
 \begin{center}
  \captionsetup{width=\textwidth}
  \caption{Literature categorization}\label{tab15}
   \begin{tabularx}{\textwidth}{l l l l l l}
   \toprule
   Identifier &Dependent &Tax variable &Driver/Legislation & Data/Method &Region/Period\\
   \midrule
   \textcite{hines_fiscal_1994} &EBIT &Effective tax rate &None &Country-level cross-section, OLS &US, 1982\\
   \textcite{huizinga_international_2008} &EBIT &Composite tax variable &Scale of operations &Affiliate-level cross-section, OLS &Europe, 1999\\
   \textcite{beer_profit_2015} &EBIT &Tax differential &Intangibles of subsidiary, size of MNE &Affiliate-level panel data, FE &Europe, 2003-11\\
   \textcite{conover_further_2000} &US taxes &N/A, other methodology &Firm size &Group-level panel data, FE &US, 1982-84, 1988-90\\
   \textcite{grubert_intangible_2003} &ROS &Statutory tax rate &Intangibles of parent &Affiliate-level cross-section OLS &US, 1996\\
   \textcite{dischinger_corporate_2011} &P/L before tax &Average tax differential &Intangibles of subsidiary &Affiliate-level panel data, FE &Europe, 1995-2005\\
   \textcite{dischinger_corporate_2008} &Intangibles &Average tax differential &N/A, other methodology &Affiliate-level panel data, FE &Europe 1993-2006\\
   \textcite{karkinsky_corporate_2012} &Number of patent applications &Statutory tax rate &N/A, other methodology &Affiliate-level panel data, FE &Europe, 1995-2003\\
   \textcite{weichenrieder_profit_2009} &ROA &Statutory tax rate &Ownership share &Affiliate-level panel data, FE &Germany, 1996-2003\\
   \textcite{desai_costs_2004} &ROA &Statutory tax rate &Ownership share &Affiliate-level panel data, FE &US, 1982-1997\\
   \textcite{dischinger_profit_2008} &P/L before tax &Tax differential &Ownership share &Affiliate-level cross-section, OLS$^a$ &Europe, 2004$^a$\\
   \textcite{buettner_internal_2013} &Internal loans &Statutory tax rate &Ownership share &Affiliate-level panel data, FE &Germany, 1996-2005\\
   \textcite{dischinger_role_2014} &P/L before taxes &Tax differential &Shifting direction &Affiliate-level panel data, FE &Europe, 1995-2007\\
   \textcite{lohse_impact_2012} &EBIT &Statutory tax rate &TP legislation &Affiliate-level panel data, FE &Europe, 1999-2009\\
   \textcite{buettner_anti_2017} &PPE &Statutory tax rate &Thin-capitalization rules &Affiliate-level panel data, FE &Germany, 1996-2007\\
   \textcite{beuselinck_cross-jurisdictional_2015} &ROS &Tax differential &Tax enforcement &Affiliate-level panel data, FE &Europe, 1998-2009\\
   \textcite{ruf_taxation_2012} &Passive assets &Statutory tax rate &CFC rules &Affiliate-level panel data, FE &Germany, 1996-2006\\
   \bottomrule
   \end{tabularx}
  \caption*{\footnotesize{\textit{Notes}. The classification is based on the main analysis in these papers. Other specifications might be used in robustness checks or alternative analysis. N/A refers to "not appplicable". $^a$\textcite{dischinger_profit_2008} uses panel data and FE estimation to analyze income shifting. To examine the effect of the ownership share however, he uses only the last year of his panel, as ownership information from ORBIS is available for the last year only. ROA, ROS, and PPE stand for return on assets, return on sales and plant, property and equipment. Source: own table.}}
 \end{center}
\end{sidewaystable}


%---B---
\section{Appendix \Cref{sec:Theoretical considerations}}\label{app:B}
\subsection{Matrix algebra and industry classification}\label{app:B1}
The vector $\bm{T_{it}}$ of dimensionality $(1\times8)$ contains the values of the categorical variables $\bm{T_t}$ for subsidiary $i$ in year $t$. The categorical variables $\bm{T_t}$ are equal to 1 if the observation falls into year $t$, and 0 otherwise. The variable for the year 2007 is omitted to avoid perfect multicollinearity. Therefore, each vector $\bm{T_{it}}$ contains exactly one value equal to 1 and seven values equal to 0. The vector $\bm{\theta_{it}}$ of dimensionality $(8\times1)$ contains the coefficients for the categorical variables $\bm{T_t}$. The vectors are given below. The vector $\bm{\theta_{it}}$ has been transposed.

%---Equation 16---
\begin{equation}
 \begin{split}
  \bm{T_{it}} &=(T_{08}\ T_{09}\ T_{10}\ T_{11}\ T_{12}\ T_{13}\ T_{14}\ T_{15})\text{,}\\
  \bm{\theta_{it}'} &= (\theta_{08}\ \theta_{09}\ \theta_{10}\ \theta_{11}\ \theta_{12}\ \theta_{13}\ \theta_{14}\ \theta_{15})\text{.}
 \end{split}
\end{equation}

Applying matrix multiplication to the two vectors as suggested by the econometric models in \cref{sec:Estimation approach} \parencite[790-791]{wooldridge_introductory_2009}, leads to a scalar result. This scalar is equal to the coefficient estimate of the categorical variable indicating the year the observation falls into. An example is given. For subsidiary $i$ in year 2011, the vector $\bm T$ is $\bm{T_{i11}}=(0\ 0\ 0\ 1\ 0\ 0\ 0\ 0)$. Multiplying $\bm{T_{i11}}$ with the corresponding vector $\bm{\theta_{it}}$, $\bm{\theta_{i11}}$, is equal to the coefficient estimate for $\theta_{11}$. The resulting estimation equation is the same as if the categorical variables indicating the year had been added as individual variables.

When FE estimation is used, it is not possible to estimate coefficients for the categorical industry variables as this information is captured by the subsidiary-fixed effect $\rho_i$. Therefore, industry-year dummies are used as a substitute \parencite[for example][259]{dischinger_role_2014}. This is done by interacting the industry dummies with the year dummies. Each industry has now 8 dummy variables equal to 1 if the observation falls into that industry in that year and 0 otherwise. The year 2007 and industry C are dropped because of multicollinearity (only the industries C and G are used in the main analysis). Doing so results in 8 industry-year dummy variables. The vector $\bm{U_{it}}$ of dimensionality $(1\times8)$ contains the industry-year dummy variables and the vector $\bm{\xi_{it}}$ of dimensionality $(8\times1)$ contains the corresponding coefficient estimates. The same comments as for the time dummy variables apply when it comes to estimation. The vectors are shown below and the vector $\bm{\xi_{it}}$ is transposed.

%---Equation 17---
\begin{equation}
 \begin{split}
  \bm{U_{it}} &=(G_{08}\ G_{09}\ G_{10}\ G_{11}\ G_{12}\ G_{13}\ G_{14}\ G_{15})\text{,}\\
  \bm{\xi_{it}'} &= (\xi_{G08}\ \xi_{G09}\ \xi_{G10}\ \xi_{G11}\ \xi_{G12}\ \xi_{G13}\ \xi_{G14}\ \xi_{G15})\text{.}
 \end{split}
\end{equation}

The dimensionality of these vectors changes when more industries are included. When the NACE main sectors A-I are included in robustness tests, the vector $\bm{U_{it}}$ is of dimensionality $(1\times64)$, and the vector $\bm{\xi_{it}}$ is of dimensionality $(64\times1)$. Industry A and the year 2007 being the reference categories for now. The vectors with the new dimensions are shown below.

%---Equation 18---
\begin{equation}
 \begin{split}
  \bm{U_{it}} &=(B_{08} \cdots B_{15}\ C_{08} \cdots C_{15}\  D_{08} \cdots D_{15} \cdots I_{08} \cdots I_{15})\text{,}\\
  \bm{\xi_{it}'} &= (\xi_{B08}\cdots \xi_{B15}\ \xi_{C08}\cdots \xi_{C15}\cdots \xi_{I08}\cdots \xi_{I15})\text{.}
 \end{split}
\end{equation}

\cref{tab16} shows which industries are used at different stages of this thesis. The industry classification is taken from the European \textcite[57]{european_commission_nace_2008}. Industries C and G are used in the main analysis, and industries A-I are used in robustness tests.

%---Table 16---
\begin{table}[t]
\scriptsize
 \begin{center}
  \caption{Structure of NACE rev. 2 main sectors}\label{tab16}
   \begin{tabularx}{\linewidth}{l l l X}
   \toprule
   Letter &Main &Rob. &Description\\
   \midrule
   A &No & Yes &Agriculture, forestry and fishing (acts as the reference category)\\
   B &No & Yes &Mining and quarrying\\
   C &Yes & Yes &Manufacturing\\
   D &No & Yes &Electricity, gas, steam and air conditioning supply\\
   E &No & Yes &Water supply, sewage, waste management and remediation activities\\
   F &No & Yes &Construction\\
   G &No & Yes &Wholesale and retail trade, repair of motor vehicles and motorcycles\\
   H &No & Yes &Transportation and storage\\
   I &No & No &Accommodation and food service activities\\
   J &No & No &Information and communication\\
   K &No & No &Financial and insurance activities\\
   L &No & No &Real estate activities\\
   M &No &No &Professional, scientific and technical activities\\
   N &No & No &Administrative and support service activities\\
   O &No & No &Public administration and defense, compulsory social security\\
   P &No & No &Education\\
   Q &No & No &Human health and social work activities\\
   R &No & No &Arts, entertainment and recreation\\
   S &No & No &Other service activities\\
   T &N/A &N/A &Activities of households as employers, undifferentiated goods- and service-producing activities of households for own use\\
   U &N/A & N/A &Activities of extraterritorial organizations and bodies\\
   \bottomrule
   \end{tabularx}
  \caption*{\footnotesize{Source: own table, the industry classification is from the European \textcite[57]{european_commission_nace_2008}.}}
 \end{center}
\end{table}


\subsection{Hausman specification test}\label{app:B2}
A Hausman specification test is carried out to decide whether FE or RE estimation is appropriate. The test is carried out for regression (2) in \cref{tab6}, and a corresponding RE estimation.\footnote{The test can not be carried out with clustered standard errors, therefore regular standard errors are used.} The basic idea of this test is to compare the FE estimates with the RE estimates. In case the estimates differ, FE is the appropriate estimation method \parencite[288]{wooldridge_econometric_2002}. The Hausman specification test can be interpreted as a test to verify the RE assumption that the subsidiary-fixed effect ($\rho_i$ in this case) is uncorrelated with each of the explanatory variables, i.e. whether $E(\rho_i |X_{it})=0$ holds \parencite[1263]{hausman_specification_1978}. This assumption needs to be made when applying RE estimation \parencite[489]{wooldridge_introductory_2009}. If this assumption is violated, only the FE estimates will be unbiased, whereas if the assumption holds, FE and RE estimates should not differ largely \parencite[1263]{hausman_specification_1978}. To perform the test in Stata, the null and alternative hypothesis are\\

\noindent$H_0$: the RE assumption holds, both the RE and FE estimators are consistent, RE is efficient, and\\
$H_A$: the RE assumption does not hold, RE is inconsistent, but FE is consistent.\\

%---Figure 13---
\begin{figure}[!]
 \centering \captionsetup{width=0.95\linewidth}
   \includegraphics[scale=1]{fig13.pdf}
 \caption[Distribution of main variables]{Distribution of main variables. The blue line represent kernel densities and the red lines represent normal distributions calculated using the empirical mean and standard deviations. Source: own figure.}\label{fig13}
\end{figure}

Thus, if the $H_0$ is rejected, FE estimation is the appropriate method \parencite[940-943]{statacorp._stata_2015-1}. Conducting the Hausman specification test in Stata gives a $\chi^2$ test-statistic of 613.96 and the corresponding $p$-value is 0.00. The $H_0$ is rejected and the Hausman specification test indicates to use FE.
\newpage

%---C---
\section{Appendix \Cref{sec:Data and sample}}\label{app:C}
\subsection{Distribution of main variables}\label{app:C1}
See \cref{fig13} for the distribution of the main variables. This figure is intended to visually convey the summary statistics in \cref{tab2} in \cref{sec:5.1}.

\subsection{United Nations classification of world regions}\label{app:C2}
See \cref{tab17} for the United nations classification of geographic regions \parencite{united_nations_methodology_2017}. Data calculations in \cref{tab4,tab5} as well as \cref{fig2,fig3,fig4} are based on this classification. The ISO alpha-2 country codes are given in parentheses. Subsidiaries from a total of 63 countries are included in the sample.

%---Table 17---
\begin{sidewaystable}[!]
\footnotesize
 \begin{center}
 \captionsetup{width=0.9\linewidth}
  \caption{United Nations geographic regions}\label{tab17}
   \begin{tabularx}{0.9\linewidth}{l l l l l l l l}
   \toprule
   Africa (5) &Americas (9) &Asia (14) &Western Europe (6) &Eastern Europe (7) &Northern Europe (9) &Southern Europe (11) &Oceania (2)\\
   \midrule
   (DZ) Algeria &(AR) Argentina &(HK) Hong Kong &(AT) Austria &(BG) Bulgaria &(DK) Denmark &(BA) Bosnia Herzegovina &(AU) Australia\\
   (KE) Kenya &(BM) Bermuda &(IN) India &(BE) Belgium &(Czech Republic) &(EE) Estonia &(HR) Croatia &(NZ) New Zealand\\
   (MA) Morocco &(CA) Canada &(ID) Indonesia &(FR) France &(HU) Hungary &(FI) Finland &(GR) Greece\\
   (NG) NIgeria &(CL) Chile &(IL) Israel &(DE) Germany &(PL) Poland &(IS) Iceland &(IT) Italy\\
   (ZA) South Africa &(CR) Costa Rica &(JP) Japan &(LU) Luxembourg &(RO) Romania &(IE) Ireland &(MK) Macedonia\\
   &(EC) Ecuador &(KW) KUwait &(NL) Netherlands &(SK) Slovakia &(LV) Latvia &(MT) Malta\\
   &(JM) Jamaica &(MY) Malaysia &(CH) Switzerland$^a$ &(UA) Ukraine &(NO) Norway &(ME) Montenegro\\
   &(US) United States &(PK) Pakistan & & &(SE) Sweden &(PT) Portugal\\
   &(UY) Uruguay &(PH) Philippines & & &(GB) United Kingdom &(RS) Serbia\\
   & &(KR) Republic of Korea & & & &(SI) Slovenia\\
   & &(SG) Singapore & & & &(ES) Spain\\
   & &(LK) Sri Lanka\\
   & &(TH) Thailand\\
   & &(AE) United Arab Emirates\\
   \bottomrule
   \end{tabularx}
  \caption*{\footnotesize{\textit{Notes}. $^a$Switzerland's listing is informational only, Swiss subsidiaries are not included in the sample. Source: own table.}}
 \end{center}
\end{sidewaystable}

\subsection{Spatial distribution of subsidiaries in Europe}\label{app:C3}
See \cref{fig14} for the detailed map of Europe. This map complements the world map in \cref{fig2} in \cref{sec:5.1}. The purpose is to give a more detailed view on Europe, where most of the subsidiaries of Swiss MNEs are incorporated.

%---Figure 14---
\begin{figure}[h]
 \centering \captionsetup{width=0.95\linewidth}
   \includegraphics[scale=1]{fig14.pdf} 
 \caption[Spatial distribution of subsidiaries in Europe]{Spatial distribution of subsidiaries in Europe. Countries with no subsidiaries are blank. The number of subsidiaries is presented in \cref{tab4}. The number of subsidiaries have been log-transformed to get a meaningful color scale. Source: own figure.} \label{fig14}
\end{figure}

\subsection{National tax rate peculiarities}\label{app:C4}
See \cref{tab18} for peculiarities in national tax rates provided by \textcite{kpmg_corporate_2017}. Due to space considerations, only peculiarities for the calculation of European tax rates are shown. It is further described how the calculation of the tax rates potentially affects the shifting incentive. Of special concern are the variations in tax rates in Germany and Switzerland. German subsidiaries make up 13.5\% of the sample (see \cref{tab4} in \cref{sec:5.1}) and a detailed treatment of the tax differential of German subsidiaries might improve the analysis. The tax rate of Switzerland is of greater influence as it affects the tax differentials of all observations. The same argument applies to the other countries listed in \cref{tab18}, due to the low number of subsidiaries the influence is expected to be of smaller extent. However, a tax treatment based on the exact location of subsidiaries and parent firms is outside the scope of this thesis and therefore neglected.

%---Table 18---
\begin{table}[t]
\scriptsize
 \begin{center}
  \caption{National tax rate peculiarities}\label{tab18}
   \begin{tabularx}{\linewidth}{l c X}
   \toprule
   Country & CITR$^a$ &Peculiarities in the tax system potentially affecting the empirical analysis\\
   \midrule
   Austria &25\% &Worldwide taxation in Austria. No tax benefit from income shifting unless appropriate double taxation relief methods are in place.\\
   Bosnia H. &10\% &Worldwide taxation in Bosnia Herzegovina. No tax benefit from income shifting unless appropriate double taxation relief methods are in place.\\
   Croatia &20\% &Various tax favors available, partly depending on the region of incorporation. The tax differential varies across regions and the tax incentive might be over- or understated depending on the exact location of the subsidiary.\\
   Germany &29.72\% &The tax rate consists of an income tax rate of 15\%, a solidarity surcharge of 0.825\% and a local trade tax varying between 7\% and 17.15\%. The tax differential varies across regions and the tax incentive might be over- or understated depending on the exact location of the subsidiary.\\
   Latvia & 15\% &Tax benefits for firms operating in special economic zones. The tax differential might vary across regions and the tax incentive might be over- or understated depending on the exact location of the subsidiary.\\
   Luxembourg &27.08\% &Differing municipal business taxes vary by location. The tax differential varies across regions and the tax incentive might be over- or understated depending on the exact location of the subsidiary.\\
   Macedonia &10\% &Worldwide taxation in Macedonia. No tax benefit from income shifting unless appropriate double taxation relief methods are in place.\\
   Switzerland &17.92\% &Cantons apply different tax rates and municipal taxes vary across regions and time. The tax differential varies across regions and the tax incentive might be over- or understated depending on the exact location of the parent firm.\\
   \bottomrule
   \end{tabularx}
  \caption*{\footnotesize{\textit{Notes}. The information is available in the footnotes to KPMG's corporate tax rate tables online \parencite{kpmg_corporate_2017}. Tax benefits granted depending on the industry-affiliation of the firm are neglected since the main analysis includes only subsidiaries from the manufacturing and wholesale and retail industry, where sector specific tax benefits are rarely granted. $^a$This is the CITR in 2015 that is used in the empirical analysis. Source: own table.}}
 \end{center}
\end{table}

\subsection{Detailed tax rate graphs for all world regions}\label{app:C5}
See \cref{fig15} for the detailed tax rate graphs. \cref{fig15} complements \cref{fig3,fig4} in \cref{sec:5.1}. The European panels are equal to \cref{fig4}. Due to the low number of countries in certain regions, the graphs might not always be useful (e.g. minima and maxima equal to the average tax rates).

%---Figure 15---
\begin{figure*}[!]
 \centering \captionsetup{width=0.95\linewidth}
   \includegraphics[scale=1]{fig15}
 \caption[Corporate tax rates across the world (detailed world regions)]{Corporate tax rates across the world (detailed world regions). Solid black lines represent unweighted mean tax rates, dashed lines depict minimum and maximum tax rates, and the shaded area shows the mean tax rate $\pm$ 1 standard deviation. The red line depicts the Swiss tax rate. Tax data is taken from \textcite{kpmg_corporate_2017}. Countries are assigned to geographic regions based on United \textcite{united_nations_methodology_2017}, see \cref{app:C2}. Source: own figure.} \label{fig15}
\end{figure*}

\subsection{Variables overview and datasources}\label{app:C6}
See \cref{tab19} for the variables and datasources. Interaction terms are constructed using the variables listed below and are therefore not listed.
\newpage

%---Table 19---
\begin{table*}[t]
\footnotesize
 \begin{center}
  \caption{Variables overview and datasources}\label{tab19}
   \begin{tabularx}{\linewidth}{l l X l}
   \toprule
   Variable &Description &Measurement &Datasource\\
   \midrule
   $\Pi_{it}$ &Total income &EBIT, P/L before tax (robustness) &ORBIS\\
   $A_{it}$ &Technology input &GDP per capita (in local currency units) &World Bank\\
   $L_{it}$ &Labor input &Cost of employees, number of employees ($L\_N_{it}$, robustness) &ORBIS\\
   $K_{it}$ &Capital input &Fixed assets, tangible fixed assets ($TK_{it}$, robustness) &ORBIS\\
   $K \_d_{it}$ &Capital input &$K \_d_{it}=1$, if ln fixed assets are above mean; $K \_d_{it}=0$, otherwise &ORBIS\\
   $\tau_{it}$ &Tax differential &Subsidiary tax rate minus parent tax rate, $(r_{it}-r_{it})$ &KPMG$^a$, Damodaran\\
   $I_{it}$ &Intangibles &Intangible fixes assets &ORBIS\\
   $I \_d_{it}$ &Intangibles &$I \_d_{it}=1$, if ln intangible fixed assets are above mean; $I \_d_{it}=0$, otherwise &ORBIS\\
   $Case2_{it}$ &Shifting direction &$Case2_{it}=1$ if the shifting direction is to the parent $(r_{it}>r_{ht})$; and $Case2_{it}=0$, otherwise &KPMG\\
   $\text{OW\_51}_{it}$ &1$^{st}$ ownership variable &$\text{OW\_51}_{it}=1$, if the subsidiary is owned with a share between 51 and 99.99\% and $\text{OW\_51}_{it}=0$, otherwise &ORBIS$^b$\\
   $\text{OW\_100}_{it}$ &2$^{nd}$ ownership variable &$\text{OW\_100}_{it}=1$, if the subsidiary is wholly-owned and $\text{OW\_100}_{it}=0$, otherwise &ORBIS$^b$\\
   $LEV_{it}$ &Leverage &Ratio of ln debt over ln total assets &ORBIS\\
   $GDP\_G_{it}$ &GDP growth &Percentage &World Bank\\
   $T\_CY_{it}$ &Time in years &Calendar year, ranging from 2007 to 2015 &ORBIS\\
   $[a]\_d_{it}$ &World region &Categorical variables indicating the world region, $[a]\_d_{it}=1$, if the observations fall into that region, and 0 otherwise, with $a\in \{\text{Americas, Asia, Europe, Oceania}\}$. &ORBIS, UN\\
   $[b]\_d_{it}$ &Region within Europe &Categorical variables indicating the region within Euroope, $[b]\_d_{it}=1$, if the observations fall into that region, and 0 otherwise, with $b\in \{\text{Northern\_Europe, Southern\_Europe, Western\_Europe}\}$. &ORBIS, UN\\
   \bottomrule
   \end{tabularx}
  \caption*{\footnotesize{\textit{Notes}. $^a$KPMG does not provide an export function. Damodaran's website is used to download the data, \url{http://people.stern.nyu.edu/adamodar/New\_Home\_Page/datafile/countrytaxrate.htm}. $^b$\cref{sec:5.1} explains how the two categorical ownership variables have been constructed. Source: own table.}}
 \end{center}
\end{table*}

%---Figure 16---
\begin{figure*}[!]
 \centering \captionsetup{width=0.95\textwidth}
   \includegraphics[scale=1]{fig16.pdf}
 \caption[Scatterplot of residuals against prediction including fixed effect]{Scatterplot of residuals against prediction including fixed effect. The scatterplot shows the residuals against predicted values of ln EBIT. The prediction is calculated using coefficient estimates from regression (2) in \cref{tab6} including the subsidiary-fixed effect. The top and right plot show histograms for the linear prediction of ln EBIT and the residuals. Blue lines represent the kernel density of the empirical distribution and red lines depict the corresponding normal density. Source: own figure.} \label{fig16}
\end{figure*}

%---D---
\section{Appendix \Cref{sec:Empirical results}}\label{app:D}
\subsection{Regression diagnostics}\label{app:D1}
FE estimation is equivalent to pooled OLS on time-demeaned data \parencite[482]{wooldridge_introductory_2009}. Thus, assessing the appropriateness of a model is similar as with a standard OLS model. The FE regression assumptions as defined by Stock and Watson \textcite[404-406]{stock_introduction_2012} are: the error term $u_{it}$ has conditional mean zero, the observations are independently and identically distributed (i.i.d.), large outliers are unlikely and no perfect multicollinearity is present. The first assumption is given most attention since it ensures unbiasedness of the estimator \parencite[238, 404]{stock_introduction_2012}. All regression diagnostics are based on the benchmark regression (2) from \cref{tab6}.

\cref{fig16} shows the residuals plotted against the predicted values of ln EBIT, including the fixed effect. The greyscale and the hexagon plot form provide helpful and allow to identify where the majority of observations is situated. Each hexagon contains the number of observations as indicated by the scale. Most observations are spread equally across the zero-line, indicating that the residuals $u_{it}$ suit the conditional mean assumption reasonably well \parencite[164]{stock_introduction_2012}. \cref{fig16} further shows that heteroscedasticity appears to be present among the residuals. Residuals corresponding to predictions between 10 and 15 show higher variability than the residuals corresponding to lower and higher predictions. As a consequence, clustered standard errors robust to heteroscedasticity and serial correlation are used for all regression specifications \parencites[404]{stock_introduction_2012}[285]{hoechle_robust_2007}. The histogram of the residuals further shows that the residuals are not normally distributed. The kernel density of $u_{it}$ has less probability mass at the centre, and shows a higher than normal probability of large, negative residuals. The non-normality of the residuals is pronounced in a quantile-quantile (Q-Q) plot in \cref{fig17}.

The Q-Q plot shows a heavy-tailed distribution of the residuals. Normality of the residuals is clearly not given, however, asymptotic approximations can be relied on since the number of observations (26'869) is high and the number of time periods (9) is small \parencite[504]{wooldridge_introductory_2009}. The model is kept in its form as in regression (2) in \cref{tab6}.

\cref{fig18} plots the main independent variables against the dependent variable. The scatterplots allow to judge the linearity of the relationship between the two variables in question. Linearity is not considered a problem. The scatterplot of the tax differential and the EBIT might indicate a slight curvature. However, including a squared term of the tax differential does not improve the econometric model (see regression (4) in \cref{tab6}). The other scatterplots show reasonably linear relationships between the dependent and independent variables.

%---Figure 17---
\begin{figure}[!]
 \centering \captionsetup{width=0.95\linewidth}
   \includegraphics[scale=1]{fig17.pdf} 
 \caption[Quantile-quantile plot of residuals from regression (2) in \cref{tab6}]{Quantile-quantile plot of residuals from regression (2) in \cref{tab6}. Source: own figure.} \label{fig17}
\end{figure}

%---Table 20---
\begin{table}[t]
\footnotesize
 \begin{center}
  \caption{Shifting direction interaction from \cref{tab7} in greater detail}\label{tab20}
   \begin{tabularx}{\linewidth}{l c c}
   \toprule
   \multicolumn{3}{l}{Subsidiary-fixed effects, panel 2007-2015, dep. var.: ln EBIT, $\Pi_{it}$}\\
   \midrule
   Industry &C:manufacturing &G:wholesale, retail\\
   \cmidrule(lr){1-1}
   \cmidrule(lr){2-3}
   Explanatory variables &(1) &(2)\\
   \midrule
   ln GDP per capita, $(A_{it})$ &$0.084$ &$0.545^{**}$\\
   &$(0.498)$ &$(2.305)$\\
   ln fixed assets, $(K_{it})$ &$0.055^{**}$ &$0.062^{***}$\\
   &$(2.520)$ &$(4.235)$\\
   ln cost of employees. $(L_{it})$ &$0.618^{***}$ &$0.368^{***}$\\
   &$(11.695)$ &$(8.717)$\\
   Tax differential, $(\tau_{it})$ &$1.591$ &$-7.623^{***}$\\
   &$(1.056)$ &$(-3.046)$\\
   Shifting direction, $Case2_{it}$ &$0.081$ &$-0.019$\\
   &$(0.807)$ &$(-0.218)$\\
   Direction interaction, &$-3.715^{**}$ &$6.597^{**}$\\
   $\quad (\tau_{it}\times Case2_{it})$ &$(-2.104)$ &$(2.496)$\\
   Year dummies &$\sqrt{}$ &$\sqrt{}$\\
   Industry-year dummies & &\\
   No. of observations &12'356 &14'513\\
   No. of subsidiaries &2'163 &2'699\\
   Within $R^2$ &0.081 &0.059\\
   Overall $F$-test &25.594 &23.210\\
   \bottomrule
   \end{tabularx} 
  \caption*{\footnotesize{\textit{Notes}. Regression (1) and (2) are based on regression (7) in \cref{tab7} with limitations on industries included. Regression (1) includes only subsidiaries from the manufacturing industry and regression (2) includes only subsidiaries from the wholesale and retail industry. $^*$, $^{**}$, $^{***}$ denote significance on the 10, 5, 1\% significance level. $t$-statistics are reported in parenthesis and standard errors are clustered at the subsidiary level to control for heteroscedasticity and autocorrelation \parencite[285]{hoechle_robust_2007}. Source: own table.}}
 \end{center}
\end{table}

%---Figure 18---
\begin{figure*}[t]
 \centering \captionsetup{width=0.95\textwidth}
   \includegraphics[scale=1]{fig18.pdf}
 \caption[Scatterplots of main variables against dependent variable]{Scatterplots of main variables against dependent variable. Source: own figure.} \label{fig18}
\end{figure*}

\subsection{Additional comments to the intangibles interaction in Table 7}\label{app:D2}
These comments concern regression (4) in \cref{tab7}. The marginal effects of the tax differential are $-0.639$ $(-1.124^{**})$ for subsidiaries with below (above) mean ln intangible fixed assets. The insignificant coefficient of $-0.485$ of the interaction term tells that there is no significant difference between the two effects. \textcite[70]{brambor_understanding_2006} mention that this case can occur if the covariance between the interaction term and the tax differential is negative. Thinking of CIs is useful in the case here. The 90\% CI of the interaction term is given by $-0.485\pm1.645\times0.452=[-1.229 , 0.259]$\footnote{The formula for a 90\% confidence interval is given by $\beta_i \pm 1.645\times \text{SE}(\beta_i)$, and can be found for example in \textcite[138]{wooldridge_introductory_2009}.} and includes 0. The 90\% CIs for the marginal effects are given by $[-1.603 ,0.324]$ for subsidiaries with below mean ln intangible fixed assets and by $[-1.989 ,-0.259]$ for subsidiaries with above mean ln intangible fixed assets. The latter of the two does not include 0, meaning the effect is significant. However, the CIs of the two marginal effects overlap, and thus confirm the insignificant difference as suggested by the CI of the interaction term. The CIs illustrate that the interaction term and the marginal effects test different hypotheses. It is therefore entirely possible that they show differences in significance. Further, it should be noted that if a higher cutoff value of ln intangible assets is chosen to separate the subsidiaries into two groups, it is likely that the interaction term would show a significant coefficient. This reasoning is based on the right graph in \cref{fig7}, which shows that only subsidiaries with high intangible asset endowments engage in significant income shifting activities. The arguments based on CIs made here apply equivalently to other regressions showing the same patterns of significance.

\subsection{Additional comments to the shifting direction interaction in Table 7}\label{app:D3}

The pattern of results from regression (7) in \cref{tab7} is equivalent to regression (4) from the same table. While the coefficient estimate of the direction interaction is insignificant, the marginal effect for subsidiaries with shifting direction to the parent is significant. The marginal effect for subsidiaries with shifting direction away from the parent is insignificant. Even though this result is possible and the conclusions valid (see \textcite[70]{brambor_understanding_2006}, and the comments in \cref{app:D2}), a more detailed, industry-specific treatment could bring more clarity. The results of reestimating regression (7) in \cref{tab7}  on the subsamples of manufacturing subsidiaries and subsidiaries in the wholesale and retail industry are shown in \cref{tab20}. The results suggest that income shifting behavior is different across industries. Moreover, the significant shifting direction interaction of one industry probably offsets the reversed significant shifting direction of the other industry when analyzing both industries in one regression. Results from splitting the two subsamples according to $Case2_{it}$ are similar but not reported. However, industry-related questions are outside the scope of this thesis and are left to upcoming research.
\newpage

\onecolumn
\section*{Declaration of independent work}
Der Verfasser, Rafael Daniel Schlatter, erkl\"art an Eides statt, dass er die beiliegende Masterarbeit mit dem Thema "The impact of tax differentials on pre-tax income of Swiss MNEs" (Deutsch: "Der Einfluss unterschiedlicher nationaler Steuers\"atze auf das Vorsteuerergebnis von Schweizer multinationaler Unternehmen") selbst\"andig, ohne fremde Hilfe und ohne Benutzung anderer als der angegebenen Hilfsmittel angefertigt hat. Alle Stellen, die w\"ortlich oder sinngem\"ass aus ver\"offentlichten Quellen entnommen wurden, sind als solche kenntlich gemacht. Die Arbeit hat in dieser oder \"ahnlicher Form oder auszugsweise im Rahmen einer anderen Pr\"ufung noch nicht vorgelegen.
\vspace{2cm}

\begin{tabular}{l p{5cm}l p{4cm}l p{3cm}l}
Place, date: &Oslo, Jul 17, 2017 &Signature &
%\includegraphics[scale=0.1]{signature.pdf}
\\
\cmidrule{4-4}
\end{tabular}
\newpage

%---First blank page---
\thispagestyle{empty}
\vspace*{\fill}
\begin{center}
This page is intentionally left blank.
\end{center}
\clearpage

\twocolumn
\end{document}  